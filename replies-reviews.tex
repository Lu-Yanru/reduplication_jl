%% -*- coding:utf-8 -*-
\documentclass[fleqn,twoside]{article}


\usepackage[
style = langsci-unified,
backend = biber,
natbib = true,
]{biblatex}
\addbibresource{Bib.bib}

\usepackage{xcolor}
\definecolor{lsDOIGray}{cmyk}{0,0,0,0.45}

%\usepackage{csquotes}

%% Language
\usepackage[ngerman, english]{babel}
	

\usepackage{microtype}


%%%
%% Tables, lists, columns
% Text in columns: \begin{}{n} \columnbreak \end{multicols}
\usepackage{tabularx}
\usepackage{multicol}
%	\setlength{\columnsep}{.5cm}	


\usepackage{float}

\usepackage{subcaption}



%%%
%% Linguistic graphs
% tikz
\usepackage{tikz}
\usetikzlibrary{patterns, matrix}

% tree
%\usepackage{forest}
% Needed for the "actual forest version"
%\useforestlibrary{linguistics}
%\forestapplylibrarydefaults{linguistics}
\usepackage{langsci-forest-setup}

% avm
\usepackage{langsci-avm}




	
%% hyperref & url
\usepackage[
bookmarksnumbered, %For numbered bookmarks in PDF
hidelinks, %For links without colored borders
%breaklinks=true,
%pdfborder={0 0 0}
]{hyperref}
%\hypersetup{pdfborder={0 0 0}}
% line break url
%\usepackage{breakurl}


%hyphenation
\usepackage{german}
\selectlanguage{USenglish}
\usepackage{hyphenat}


\usepackage{etoolbox}

% examples and glossing
\usepackage{refcount}
\usepackage{xstring}

\usepackage{gb4e}
\exewidth{(222)}
\judgewidth{**}
\usepackage{cgloss}

\usepackage{jambox}

\usepackage{xspace}

% pinyin
\usepackage{xpinyin}
\xpinyinsetup {pysep = {}}


% disambiguate bib names
\newrobustcmd{\disambiguate}[3]{#2~#3}

% refs
\newcommand{\sectref}[1]{Section~\ref{#1}}
\newcommand{\figref}[1]{Figure~\ref{#1}}
\newcommand{\citegen}[2][]{\citeauthor{#2}'s (\citeyear*[#1]{#2})}

\renewcommand{\subref}[1]{\StrRight{\getrefnumber{#1}}{1}}

%%% Linguistics graphs related
%% indexed box
\newlength{\MyetagLength}
\settowidth{\MyetagLength}{{$\scriptstyle 1$}}

%% indexed box
\newcommand{\iboxt}[1]{{%
        \setlength{\fboxsep}{1.25pt}%
        \fbox{$\scriptstyle #1$}%
}}


\newcommand{\ibox}[1]{%
    %  \iboxt{#1}\,%       why this extra space???? 20.02.2018
    \iboxt{#1}%
}

\newcommand{\iboxb}[1]{(\,\iboxt{#1}\,)}

%% empty box
\newcommand{\etag}{\ibox{\rule{0ex}{1.1ex}\hspace{\MyetagLength}}\xspace}

%% type
\newcommand{\type}[1]{{\normalfont\itshape #1\/}}

%% list of phons
\newcommand{\phonliste}[1]{%
    \mbox{%
        $%
        %
        \left\langle \mbox{\normalfont\itshape#1} \right\rangle%
        $%
        %\\[-1.5mm]
    }%
}

\newcommand{\sliste}[1]{%
    \mbox{%
        $\left\langle\mbox{\upshape #1}\right\rangle$}%
}

\newcommand{\impl}{$\Rightarrow$\xspace}

% does not work properly St.Mü. 2023-06-22
\makeatletter
\def\eas{\ifnum\@xnumdepth=0\begin{exe}[(34)]\else\begin{xlist}[iv.]\fi\ex\begin{tabular}[t]{@{}p{\linewidth}@{}}}
\def\zs{\end{tabular}\ifnum\@xnumdepth=1\end{exe}\else\end{xlist}\fi}
\makeatother


%\def\ea{\ifnum\@xnumdepth=0\begin{exe}[(234)]\else\begin{xlist}[iv.]\fi\ex}
%\def\z{\ifnum\@xnumdepth=1\end{exe}\else\end{xlist}\fi}

% pinyin
\newcommand{\obj}[1]{\pinyin[format = \itshape]{#1}}
\newcommand{\objex}[1]{\pinyin{#1}}



\usepackage{todonotes}

\newcommand{\todostefan}[1]{\todo[color=orange!80]{\footnotesize #1}\xspace}
\newcommand{\todosatz}[1]{\todo[color=red!40]{\footnotesize #1}\xspace}

\newcommand{\inlinetodostefan}[1]{\todo[color=green!40,inline]{\footnotesize #1}\xspace}

\newcommand{\inlinetodoopt}[1]{\todo[color=green!40,inline]{\footnotesize #1}\xspace}
\newcommand{\inlinetodoobl}[1]{\todo[color=red!40,inline]{\footnotesize #1}\xspace}

\newcommand{\itd}[1]{\iftoggle{draft}{\inlinetodoobl{#1}}{}}
\newcommand{\itdobl}[1]{\iftoggle{draft}{\inlinetodoobl{#1}}{}}
\newcommand{\itdopt}[1]{\iftoggle{draft}{\inlinetodoopt{#1}}{}}

% for editing, remove later
%\usepackage{xcolor}
\newcommand{\iaddpages}{\iftoggle{draft}{\yel[add pages]{pages}\xspace}}

\newcommand{\addpages}{\iftoggle{draft}{\todostefan{add pages}}\xspace}
\newcommand{\addsource}{\iftoggle{draft}{\todostefan{add source}}\xspace}
\newcommand{\addglosses}{\iftoggle{draft}{\todostefan{add glosses}}\xspace}
%\usepackage{fontspec}

%\newtoggle{draft}\togglefalse{draft}
\newtoggle{draft}\toggletrue{draft}



%\usepackage{hyperref}
%\hypersetup{colorlinks=false, pdfborder={0 0 0}}

%\usepackage{unified-biblatex}

%\let\citew\citealp
%\newcommand{\page}{}
%\bibliography{Bib,bib-abbr,biblio}


%\usepackage{langsci-forest-setup}

%\usepackage{german}%

%\selectlanguage{USenglish}

%\usepackage{abbrev,merkmalstruktur,article-ex,makros.2e,mycommands,eng-date,my-xspace}
%\usepackage{my-gb4e-article}

%\usepackage{soul}

%\usepackage{forest}


\let\citew\cite

\newcommand{\is}[1]{}

\begin{document}
\noindent
{\large\bf Replies to reviews of \emph{An HPSG account for verbal reduplication in Mandarin Chinese}}

\noindent
Dear editors,

Thank you very much for the positive reviews. The following text is a repetition of the points the
reviewers raised and an explanation of how we addressed open issues.

\section{Reviewer 1}\label{sec:1}

\begin{enumerate}
%%%%% YL
\item
Regarding the interaction with Aktionsart (Sec. 2.3.2), the author(s) conclude(s) that ``This shows
that although the reduplication does have a tendency to interact with volitional verbs and with
activities and semelfactives due to its dynamic meaning, this is by no means a rigid constraint, and
non-volitional verbs, states and achievements can be reduplicated in appropriate contexts as well.''

Regarding volitionality, it should be noted that, as reported in various grammars, while it is generally true that verbs describing events not controlled by an agent cannot be reduplicated, some of them, under certain circumstances (e.g. in imperative or conditional sentences, or in causative \obj{rang/jiao/shi} sentences, as in ex. (15e) provided by the author(s)), can reduplicate (it might be worthwhile to delve into why this occurs in such sentence structures). As for the possibility of reduplicating stative verbs, this is limited to specific verbs and circumstances. It has been observed that some stative verbs, e.g. bing ‘be sick’, are basically change of state predicates, which means that they are dynamic (see e.g. Tham 2013: 663; see also Sybesma 1997, Liu 2010, Peck, Lin \& Sun 2013; Basciano 2019). The same goes for those verbs expressing psychological states which can have a dynamic interpretation, as e.g. liaojié `understand' (Basciano \& Melloni 2017, fn. 3). With regard to achievements, it is worth considering whether, in certain circumstances (the cases in which they are reduplicated), they assume a reading that cancels the felos. Anyway, since these cases are not the rule, I do not think they can be taken as an argument against the constraints on the action type of the base verb. Otherwise, if reduplication does not affect lexical aspect, how can the changes in the aspect properties of the verb after reduplication be explained? (e.g. \& (1) — AN/ kan (le ) san ge xidoshi ‘read for three hours’ vs. *\& (1) \& =AW kan (le ) kan san ge xicioshi (both reduplication and for X time expressions set a boundary to the event) / IEETS zhéngzai kan / *EAEG TG zhéngzai kan kan). In order to get a clearer picture and to find a semantic explanation for these facts, one should look at what these ‘special’ cases of reduplication have in common and try to draw generalizations: why can some stative and achievement verbs reduplicate under certain circumstances, while other cannot? What do these verbs have in common? Can we single out specific subclasses of these verbs which can reduplicate?

%%%%%%%%%%%%%%%%%%%
%%%%%%%%%%%%%%%%%%%
\noindent
\textbf{Answer:}\label{verbclass}
First, the notion of ``aspect'' adopted in this article has to be explained in more detail.
We adopt the two-component aspect model proposed by \citet{XiaoMcEnery2004} based on \citet{Smith1991}.
The general term ``aspect'' is considered to encompass the following two components:
situation aspect, i.e. ``aspectual information conveyed by the inherent semantic representation of a verb or an idealized situation'' \citep[21]{XiaoMcEnery2004};
and viewpoint aspect, i.e. ``the aspectual information reflected by the temporal perspective the speaker takes in presenting a situation'' \citep[21]{XiaoMcEnery2004}.
Situation aspect can be further modeled as verb classes (dubbed ``Aktionsart'' in the previous version of this paper) at the lexical level
and situation types (the interaction of verb classes and other constituents) at the sentential level \citep[33]{XiaoMcEnery2004}.
The verb classes are determined with verbs in a neutral context (preferably in a perfective viewpoint aspect, with a simple object only when it is obligatory),
where everything that might change the aspectual value of a verb is excluded
and only the inherent features of the verb itself are considered
(see \citealt[52]{XiaoMcEnery2004} for more details).
This does not rule out the fact that the same verb may express different aspectual properties in other contexts,
but its verb class remains the same,
as the aspectual change can be attributable to other components at the sentential level.
This section is concerned with the interaction of the reduplication with verb classes,
while the next section discusses the interaction with viewpoint aspect markers.
The combination with other constituents such as adjuncts is briefly touched upon below,
but a detailed discussion has to be left for future research.

Second, there are more examples.
Regarding volitionality, although (15e) is a causative sentence, (13a--b) are not imperative, conditional or causative (more on this below).
Example (\ref{ex:sheng}) shows the same verb \obj{sheng1} `give birth to' as in (15e) reduplicated in a declarative sentence.
\ea\label{ex:sheng}%我们女人是这么神经质又没用,只能生生孩子,做做女人的活动.。(BCC)
\gll \objex{wo2men} \objex{nv3ren2} \ldots\, \objex{zhi3} \objex{neng2} \obj{sheng1-sheng1} \objex{hai2zi} \ldots \\
we women {} only can give.birth.to-give.birth.to child\\\jambox{(BCC)}
\glt `We women can only give birth to children \ldots'
\z

As for stative verbs, \citet[Sec. 3.3]{Tham2013} uses \obj{xi3huan1} `like' and \obj{xiang1xin4} `believe' as examples of stative verbs
to contrast adjectives
and what she calls basic change of state verbs such as \obj{bing4} `sick'.
\citet[680]{PeckEtAl2013} also list \obj{xi3huan1} `like' and \obj{xiang1xin4} `believe' as stative verbs.
For these two verbs, examples such as (\ref{ex:xihuan}) and (\ref{ex:xiangxin}) can be found.
\ea\label{ex:xihuan}%这车好型啊!!!。怎么了,我就喜欢喜欢 不行么~ (BCC)
\gll \objex{zhe4} \objex{che1} \objex{hao3} \objex{xing2} \objex{a}! \objex{zen3me} \objex{le}, \objex{wo3} \objex{jiu4} \obj{xi3huan1-xi3huan1} \objex{bu4xing2} \objex{me}?\\
this car so cool \textsc{ptc} what \textsc{ptc} I just like-like cannot \textsc{ptc}\\\jambox{(BCC)}
\glt `This car is so cool! What? I can't just like it for a little bit?'

\ex\label{ex:xiangxin}%谁让我来相信相信 (CCL)
\gll \objex{shei2} \objex{rang4} \objex{wo3} \objex{lai2} \objex{xiang1xin4-xiang1xin4}\\
who let I come believe-believe\\\jambox{(CCL)}
\glt `Who lets me believe him/her?'
\z
However, as the reviewer %\citealt[283]{Ding2010}, \citealt[fn. 3]{BascianoMelloni2017}
and indeed \citet[669--670]{Tham2013} herself noted,
verbs expressing psychological states such as these can have a change of state (COS) interpretation (but not necessarily),
let us look at other examples of stative verbs listed by \citet[680]{PeckEtAl2013} which do not express psychological states,
and examples in  \citet[Sec. 3.3.3]{XiaoMcEnery2004} of individual-level states (ILSs) which only have stative interpretation,
as opposed to stage-level states (SLSs) which can have both stative and dynamic interpretations.\footnote{
Note that most examples of ILSs in \citet[Sec. 3.3.3]{XiaoMcEnery2004} are adjectives.
Since in Mandarin Chinese, adjectives have a different reduplication pattern (see e.g. \citealt[Sec. 2.2]{Tsao2001}, \citealt[Sec. 4.1]{FanSongBond2015}, \citealt[Sec. 3]{Sui2018}),
and further, COS verbs can be systematically derived from adjectives \citep[Sec. 3]{Tham2013},
these predicates are not included in our examples.}
The following examples contain the reduplication of \obj{xiang4} `look like' and \obj{zai4-chang3} `be present on the scene'.
\ea\label{ex:xiang}%范丞丞雪中撑伞这是蒋丞吧王圣迪也拿着滑板!太像顾淼了王安宇你快点也像一像吧!(https://www.163.com/dy/article/FS1FKUC30534DZO1.html (2020/11/22, accessed 2024/03/26))
Context: \objex{Wang2} \objex{Sheng4di2} is the actor of the character \objex{Gu4} \objex{Miao3} in a TV series.
\objex{Wang2} \objex{An1yu3} is another actor in the series.\\
\\
\gll \objex{Wang2} \objex{Sheng4di2} \ldots\, \objex{tai4} \objex{xiang4} \objex{Gu4} \objex{Miao3} \objex{le}. \objex{Wang2} \objex{An1yu3} \objex{ni3} \objex{kuai4dian} \objex{ye3} \obj{xiang4-yi-xiang4} \objex{ba}!\footnotemark\\
Wang Shengdi {} very look.like Gu Miao \textsc{ptc} Wang Anyu you fast also look.like-one-look.like \textsc{ptc}\\
\glt `Wang Shengdi \ldots\,  really looks like Gu Miao. Wang Anyu, you too should just look like (your character) already!'

\ex\label{ex:zaichang}%她此刻正沉浸在一个前所未有的悲剧中,很需要母亲在一在场便能给予她的那点安慰 (CCL)
\objex{ta1} \ldots\, \objex{hen4} \objex{xu1yao4} \objex{mu3qin1} \obj{zai4-yi-zai4-chang3} \ldots\\
she {} very need mother be.present-one-be.present-scene\\\jambox{(CCL)}
\glt `She \ldots\, really needs her mother to be present on scene for a little while \ldots'
\z
\footnotetext{\url{https://www.163.com/dy/article/FS1FKUC30534DZO1.html} (Accessed 26th  March 2024).}

With regard to achievements, the reduplication does not seem to cancel the telos.
Applying the classic ``for/in X-time'' test, reduplication is compatible with ``in X-time'' (\ref{ex:telob}) but not with ``for X-time'' (\ref{ex:teloc}).
\ea
\ea[]{\label{ex:teloa}% 一块... 到附近市场买买东西 (CCL)
\gll \objex{yi2kuai4} \ldots\, \objex{dao4} \objex{fu4jin4} \objex{shi4chang3} \obj{mai3-mai3} \objex{dong1xi}\\
together {} arrive nearby market buy-buy thing\\\jambox{(CCL)}
\glt `(We) together \ldots\, went to the market nearby to buy some things.'}

\ex[]{\label{ex:telob}% 两天后到附近市场买买东西
\gll \objex{liang3} \objex{tian1} \objex{hou4} \objex{dao4} \objex{fu4jin4} \objex{shi4chang3} \obj{mai3-mai3} \objex{dong1xi}\\
two day later arrive nearby market buy-buy thing\\
\glt `Two days later, (we) went to the market nearby to buy some things.}

\ex[*]{\label{ex:teloc}%到附近市场买买东西两天
    \gll \objex{dao4} \objex{fu4jin4} \objex{shi4chang3} \obj{mai3-mai3} \objex{dong1xi} \objex{liang3} \objex{tian1}\\
    arrive nearby market buy-buy thing two day\\
    \glt Intended: `(We) went to the market nearby to buy some things for two days.'}
\z\z
The following continuation that cancels the telo is also infelicitous (\ref{ex:telod}).
\ea\label{ex:telod} %到附近市场买买东西 \# 但什么也没买到
\gll \objex{dao4} \objex{fu4jin4} \objex{shi4chang3} \obj{mai3-mai3} \objex{dong1xi}, \# \objex{dan4} \objex{shen2me} \objex{ye3} \objex{mei2} \objex{mai3-dao4}\\
arrive nearby market buy-buy thing {} but anything also not buy-arrive\\
\glt Intended: `(We) went to the market nearby to buy some things, but we didn't buy anything.'
\z

All the examples above show that reduplication is applicable to all verb classes, contrary to common beliefs in the literature,
and thus, a theoretical account of reduplication should not restrict its use to only a subset of verbs.
With our account, we aim to explain as much language data as possible, regardless whether the reviewer considers some examples to be ``special'' or not.

Finally, as we discussed in Sec. 2.3.1 in the previous version of the paper, the semantics of the reduplication has the properties of tansitoriness, holisticity and dynamicity.
This means that the use of reduplication affects the situation aspect at the sentential level.
One might argue that examples such as (\ref{ex:xiang}) and (\ref{ex:zaichang}) above convey dynamic meaning, even though stative verbs are used.
%i.e. they may be interpreted as `become similar to' and `become present' respectively.
However, we argue that the dynamic interpretation does not come from the verb but from reduplication.
These verbs are stative in a neutral context and thus, we consider the intrinsic feature of these verbs to be stative.
The dynamic interpretation only arises when they are used in specific contexts, in this case, when they are reduplicated.
The point of our examples is to show that reduplication can be applied to verbs of all verb classes.

As noted briefly in Sec. 2.2 in the previous version of the paper,
the reduplication is incompatible with an expression that quantifies the duration or the extent of the event (\ref{ex:syn-quan}),
 because the reduplication already contains a (abstract) quantity meaning, namely a short duration or a small extent (transitoriness),
 which cannot be measured on a concrete scale (\citealt[155]{XiaoMcEnery2004}, \citealt[333]{SuiHu2016}).
 \ea %With an expression of quantity: 
 \label{ex:syn-quan}
 \ea[]{\label{ex:syn-quana}\gll \objex{ta1} \objex{yi4} \objex{tian1} \objex{pao3} \objex{shi2} \objex{li3}.\\
     he one day run ten mile\\ \jambox{\citep[83]{Li1998}}
     \glt `He runs ten miles a day.'}
 \ex[*]{\gll \objex{ta1} \objex{yi4} \objex{tian1} \obj{pao3}-\obj{pao3} \objex{shi2} \objex{li3}.\\
     he one day run-run ten mile\\}
 \z
 \z
% The same can be argued for the incompatibility of reduplication and a quantized object () , as both serve to delimit the event expressed by the base verb \citep[322]{SuiHu2016}.
 %() *我看看三本书。
 This results from the properties of reduplication and does not change the verb class of the predicate.

Due to its unique features compared to the perfective aspect marker \obj{le} and the experiential aspect marker \obj{guo4},
the reduplication is the only perfective viewpoint aspect that can freely occur in future situations \citep[156--157]{XiaoMcEnery2004}.
\textit{Le} expresses a situation being realized,
while \textit{guo4} conveys a situation already experienced,
and thus, both rarely occur in future situations.
The use of reduplication does not have these constraints.
This explains the general tendency for the reduplication to appear in conditional clauses and imperative or causative contexts,
which are frequently used to describe future situations.

%%%%%%%%%%%%%%%%%%%
%%%%%%%%%%%%%%%%%%%
%%%%% YL
\item
The author(s) argue(s) that verbal reduplication is a morphological phenomenon based on some tests (Sect. 2.4). However, the tests used have some shortcomings.

1) Compositionality: this test is not reliable. There are phrases whose meaning is not

compositional (see e.g. kick the bucket, shoot the breeze). Also, take verb-object constructions,
which are ambiguous between words and phrases: e.g. 1:0> dan-xin ‘take on-heart, worry” has a
non-compositional meaning, suggesting that it is a word, but is separable, suggesting that it is a
phrase (see Huang 1984). On p. 14, the author(s) state(s): “The meaning of the reduplication is not
compositional, as it does not mean that the event denoted by the verb happens twice or multiple
times, but rather that the event happens for a short duration and/or for low frequency. This non-
compositionality suggests that a reduplication is more word-like.”

If it were compositional, perhaps the meaning should be something like ‘look look” (do the action twice?) but not multiple times. However, I think it is a matter of iconicity rather than compositionality here. In addition, would this imply that the reduplicated verb is a compound? Its meaning and function, though, seem to be closer to inflection or derivation. If it is a derivational item, i.e. a suffix-like element, it can bear a fix aspectual meaning (delimitative aspect): in this sense, the meaning would be compositional.

2) Lexical integrity: if the reduplicated verb is a word, the position of the perfective aspect
marker le is unexpected. In fact, it is never found in between the two constituents of a compound
word, as the author(s) also show(s) in (27c). Her (2006: 1282), talking about V-4 géi sequences,
states that “The simple fact that the V-gei sequence cannot be separated, as in (19¢) indicates its
lexical integrity”, i.e. they form a single lexical item (745 T fih ji-géi-le ta vs. * 77 T 431 Ji-le géi ta).
\itdopt{St. Mü.: fix brackets and references to examples. The quoted text contains Chinese
  characters. Please try to insert them. Also for the examples below.}


Even if you assume that aspect markers are suffixes (morphological elements), they should not be placed in between the constituents of compounds, because this violates lexical integrity (as shown in ex. 27b by the author(s)): the suffix /e must be placed after the whole complex verb, which forms a single lexical unit, and not within it. In fact, in (non-separable) verb-object compounds, which are lexical units, | /e is never placed after the verbal constituent (F0» T guan-xin-le | *% T > guan-le xin vs. % | % gi-le md).

It is worth noticing that, different from diminishing verb reduplication, in AABB verbal reduplication (which has a different meaning and function), the perfective aspect marker /e can never be placed inside the reduplicated verb but can be put at the end of the AABB pattern. Also, I do not understand the comparison with SVCs. Let’s say that reduplication is a syntactic phenomenon, but that does not mean that it is a SVC, i.e. a sequence of verbs or verb phrases in a relation of coordination or subordination, which are basically independent of each other. It could simply be another type of syntactic construction with its specific properties.

3) Phrasal substitution. The inclusion of this test is questionable since reduplication primarily targets verbs and not phrases. Therefore, syntactic objects should not be included in the reduplication; it is not repetition of phrases. Indeed, if the object is part of a non-separable verb-object compound (which is a lexical unit), it is included in the reduplication (F</0> 0> guan-xin guan-xin/*  K> guan-guan xin vs. WsBkE tico-tiao wii | * B FEBEE ticown tiaowii).

4) Conjunction reduction. This criterion can only be applied to set apart coordinated phrases from coordinate compound words with a head in common (e.g. ‘K huiché e ¥4 giché in (33) share the head  ché ‘vehicle’). But the examples provided by the author(s) to argue against conjunction reduction with reduplicated verbs are completely different: the verbs do not have the head in common (exx. 34). I believe that this is not a reasonable test to argue for the wordhood of reduplicated verbs.

In summary, the tests used to support the morphological status of reduplication appear
unconvincing.


%%%%%%%%%%%%%%%%%%%%
%%%%%%%%%%%%%%%%%%%%
\textbf{Answer:}\label{morph}
Based on the suggestions by the reviewers and further literature research, this section is drastically revised.
The tests of semantic compositionality and conjunction reduction are deemed unreliable.
\citet[37--38]{Dai1992}, \citeyear[120--122]{Dai1998} explains the problems of conjunction reduction (he calls it ``coordination'').
That leaves us with phrasal extension (or ``modification'' according to \citealt[32]{Dai1992}, \citeyear[117]{Dai1998})
and phrasal substitution (or ``expansion'' according to \citealt[33]{Dai1992}, \citeyear[117--120]{Dai1998}).
These tests requires further clarifications.

First, the modification test suggests that subparts of a word cannot be modified at a phrasal level.
This is possible for a VP (\ref{ex:modVP}), as the NP inside of the VP can be modified by e.g. an AP.
\ea\label{ex:modVP} %开红色的门
\gll \objex{kai1} \objex{hong2se4de} \objex{men2}\\
open  red door\\
\glt `open the red door'
\z
In contrast, the individual verbs in reduplication cannot be modified by an e.g. AdvP.
(\ref{ex:modVV}) is ungrammatical whether the AdvP is interpreted to modify the first or the second verb.
\ea[*]{\label{ex:modVV} %看偷偷地看
\gll \objex{kan4} \objex{tou1tou1de} \objex{kan4}\\
look secretly look\\
}
\z

\itdopt{Stmü: Fix examples and referecnes to them.}

Second, the expansion test suggests that a phrasal dependent (either a modifier or an argument) cannot be inserted into a word.
This is possible for a verbal classifier phrase (\ref{ex:inVCP}), as the object can occur after or in-between.
\ea\label{ex:inVCP}%开两次门
%开门两次
\ea \gll \objex{kai1} \objex{men2} \objex{san1} \objex{ci4}\\
open door three time\\
\glt `open the door three times'

\ex  \gll \objex{kai1}  \objex{san1} \objex{ci4} \objex{men2}\\
open three time door\\
\glt `open the door three times'
\z\z
For reduplication, this is also not possible (\ref{ex:inVV}), as the object cannot be inserted between the two verbs.
\ea\label{ex:inVV}
\ea[]{\gll \obj{kai1}-(\obj{le})-\obj{kai1} \objex{men2}\\
open-\textsc{pfv}-open door\\
\glt `open the door for a little while'}

\ex[*]{%开(了)门(了)开
\gll \obj{kai1}-(\obj{le})-\obj{men2}-\obj{kai1}\\
open-\textsc{pfv}-door-open\\
}
\z\z
% If we analyze reduplication as a phrasal operation, we would expect the structure of
% [VP [VP [V-le V]] O]
% Since both phrases are VP, there should be nothing against the following strucutre:
% [VP [VP [V-le O]] V]
% However, this is not possible.

Regarding the position of \obj{le}, there are counter-examples that show V-\obj{le-gei3} (\ref{ex:Vgei}) as well as \obj{guan1-le-xin1} `close-\textsc{pfv}-heart, care for' (\ref{ex:guanxin}) are possible.
\ea\label{ex:Vgei}%还有许多女子,将自己的相片,亲笔签字在上面,寄了给他。(CCL)
\gll \ldots\, \objex{xu3duo1} \objex{nv2zi3}, \objex{jiang1} \objex{zi4ji3de} \objex{xiang4pian1} \ldots\, \objex{ji4-le-gei3} \objex{ta1}.\\
{} many women take own photo {} send-\textsc{pfv}-give he\\\jambox{(CCL)}
\glt `Many women \ldots\, sent him photos of themselves.'
\z

\ea\label{ex:guanxin}%许多同志看戏以后,自动的对病员关了心。大家情绪很高... (CCL)
\gll \objex{xu3duo1} \objex{tong2zhi4} \ldots\, \objex{zi4dong4de} \objex{dui4} \objex{bing4yuan2} \objex{guan1-le-xin1}.\\
many comrade {} voluntarily to patient close-\textsc{pfv}-heart\\\jambox{(CCL)}
\glt `Many comrades \ldots\, voluntarily cared for the patients.'
\z

In any case, since reduplication is not compounding (\citealt[149--150]{Sui2018}; \citealt{GaoEtAl2021} provides empirical evidence.),
and the patterns discussed here constitute a different process than the AABB pattern
(see the discussion on \citealt{Xie2020} in our paper, also \citealt[Sec. 4.3]{Deng2013}, \citealt[Sec. 2]{SuiHu2016}, \citealt{Sui2018} and \citealt{Wang2023}),
it is not surprising that \textit{le} occurs at a different position.

The comparison with Parallel Verb Compounds and Serial Verb Constructions only serves to facilitate understanding
by contrasting the behaviors of different multi-verb constructions in Mandarin Chinese,
but this does not contribute to the arguments in any substantial way.
Due to the critics from the reviewers, this comparison is removed.

Cross-linguistically, verbal reduplication in Mandarin Chinese patterns more with morphological reduplication (below as reduplication) in other languages
than syntactic reduplication (below as repetition \citealt[31]{Gil2005}, \citealt[1--2]{Forza2016}).
\citet[35--36]{Gil2005} considers non-iconicity and having only two copies as sufficient but not necessary conditions for reduplication.
He further proposes building one intonational group as sufficient and necessary condition for reduplication (p. 36).
All three conditions are true for verbal reduplication in Mandarin Chinese.
% one intonational group \citep[154]{Sui2018}
\citet[9]{Forza2016} argues that the substantial difference between reduplication and repetition lies in the fact that only the former affects grammatical features such as aspect.
This is also true for verbal reduplication in Mandarin Chinese.

In sum, we maintain that verbal reduplication in Mandarin Chinese is better off analyzed as a morphological phenomenon.

%%%%%%%%%%%%%%%%%%%%
%%%%%%%%%%%%%%%%%%%%%
%%%%YL
\item
Sect. 3.3: I cannot understand the parallel with Mandarin verbal reduplication. The
author(s) say(s): “The following section discusses previous literature which proposed a special
construction for the reduplication”. However, it becomes evident that the phenomenon under
examination is entirely different from Mandarin verbal reduplication. Ghomeshi et al. (2004) analyze
a phenomenon which is well attested in world languages, even in those without productive
reduplication, i.e. contrastive focus reduplication/repetition, which involves the repetitions of
words and sometimes phrases (Oh, we re not LIVING-TOGETHER-living-together, Gomeshi et al. 2004: 308). Their function is “to focus the denotation of the reduplicated element on a more sharply delimited, more specialized, range” (Gomeshi etal. 2004: 308); for example, in ex. (47) cited by the author(s) SALAD-salad means green salad vs. salads in general. This has nothing to do with diminishing verbal reduplication in Mandarin. The authors’ endeavor to apply the analysis proposed for contrastive focus reduplication to Mandarin verbal reduplication raises questions about its rationale (why? These two phenomena are fundamentally distinct and serve different linguistic functions). Consequently, attempting to employ the same analytical framework for both is problematic, and, indeed, the author(s) conclude(s) that: “this approach fails to account for the similarities of the reduplication and other aspect markers in Mandarin Chinese”. This is precisely because contrastive focus repetition/reduplication is a different phenomenon with a different function, and Mandarin verbal reduplication cannot be analyzed in the same way. Furthermore, the author(s) suggest(s) that this approach “[...] also provides a formal account for the phonology of the reduplication”. However, contrastive focus reduplication/repetition generally does not involve morpho-phonological readjustment phenomena, which are typically found in reduplication. As a result, it cannot adequately account for phenomena such as the neutral tone of the reduplicant in the case of reduplication of monosyllabic verbs (kan > kan kan).


%%%%%%%%%%%%%%%%%%
\textbf{Answer:}
Since the reviewer deemed it irrelevant, the section concerning \citet{Ghomeshietal2004} is removed.

%%%%%%%%%%%%%%%%%%%%%
%%%%%%%%%%%%%%%%%%%%%%
%% YL
\item Pp. 26, lines 24--25: ``The output reduplicates the phonology (phon) of the input verb with the possibility to have further phonological material in between. [] indicates an underspecified list which could be empty or not''. However, it is not entirely clear what prevents phonological material other than yi and le to appear in between the reduplicated elements. This aspect seems to be somewhat arbitrary, and it is not evident why certain elements are permissible while others are not.

%%%%%%%%%%%%%%%%%%%%
\textbf{Answer:}
\obj{le} `\textsc{pfv}' marks the perfective aspect.
The reason for \obj{le} `\textsc{pfv}' being the only aspect marker compatible with reduplication is explained in the Semantics Section.

As for \obj{yi} `one', its use stems from the historical development of delimitative verbal reduplication but is synchronically opac.
\citet[13--15]{Zhang2000} shows that delimitative verbal reduplication originates from the ``V + numeral + verbal classifier'' phrase,
where the verbal classifier is borrowed from a verb of the same form, as shown in (\ref{ex:hist1}).
\ea\label{ex:hist1}
%明又喝,岐亦喝,明连喝两喝,岐便礼拜。(《宋代卷·虚堂和尚语录》3 87 页) as cited in \citet[12]{Zhang2000}
\gll \objex{ming2} \objex{you4} \objex{he4}, \objex{qi2} \objex{yi4} \objex{he4}, \objex{ming2} \objex{liang2} \objex{he4} \objex{liang3} \objex{he4}, \objex{qi2} \objex{bian4} \objex{li3bai4}\footnotemark\\
Ming agian call Qi also call Ming in.succession call two call Qi thus bow\\
\glt `Ming called again. Qi called too. Ming called twice. Qi thus bowed.'
\z
\footnotetext{\obj{Song4dai4} \obj{juan4}: \obj{Xu1tang2} \obj{He2shang} \obj{yu3lu4} \textit{[The Song dynasty volume: Quotations from Abbot Xutang]} p. 387 as cited in \citet[12]{Zhang2000}}
Here the numeral refers to the actual number of the action taking place
and can be any number.
This evolves into the structure of V-\obj{yi}-V (or A-\obj{yi}-A), which does not express the actual number of the action anymore,
but simply conveys that the action happens in a short period of time or for only few times (\ref{ex:hist2}).
\ea\label{ex:hist2}
%且歇一歇了,去坐地。(《宋代卷·张协状元》5 19 页) as cited in \citet[13]{Zhang2000}
\gll \objex{qie3} \obj{xie1-yi-xie1} \objex{le}, \objex{qu4} \objex{zuo4} \objex{di4}.\footnotemark\\
just rest-one-rest \textsc{ptc} go sit ground\\
\glt `(Let me) just take a rest and go sit on the ground.'
\z
\footnotetext{\obj{Song4dai3} \obj{juan4}: \obj{Zhang1} \obj{Xie2} \obj{zhuang4yuan2} \textit{[The Song dynasty volume: Top graduate Zhang Xie]} p. 519 as cited in \citet[13]{Zhang2000}}
In this case, \obj{yi1} `one' cannot be replaced by other numerals.
Since V-\obj{yi}-V does not express the actual number of the action anymore,
it became possible to delete the \obj{yi1} `one',
hence the use of the VV (or AA) form, too.

Synchronically, A-\obj{yi}-A has the same meaning as AA and both do not refer to the actual number of the event denoted by the verb.
This means that \obj{yi1} `one' in A-\obj{yi}-A does not contribute any specific meaning to the structure
and is only there as a historical remnant.
In this sense,  \obj{yi1} `one' appearing in between the reduplication is synchronically arbitrary. %similar to linking elements in compounds

Besides the obvious conceptual reason that  \obj{yi1} `one' means `little, few',
there might be a phonological explanation for  \obj{yi1} `one' rather than other numerals is used in the A-\obj{yi}-A structure.
In Chinese, one form of intensifying reduplication is A-\obj{li}-AB e.g. \obj{hu2-li-hu2tu2} `confused-li-confused',
where the \obj{li} is fixed and also does not bear any meaning.
\citet[137]{Sui2018} assumes that this syllable is filled with \obj{li} because the open syllable \obj{li} is a relatively unmarked phonological constituent \citep{Yip1992}
and the second syllable in A-\obj{li}-AB occupies an unstressed position.
We can see the similarities between \obj{li} in A-\obj{li}-AB and \obj{yi} in A-\obj{yi}-A: they are both relatively unmarked and occupies an unstressed position.
This can also make it easier for  \obj{yi1} `one' rather than other numerals to become part of a fixed structure.
But a phonological account is out of the scope of this paper and has to be left for further research.

The subtypes of \type{verbal-reduplication-lr} prevent phonological materials other than \phonliste{le} and \phonliste{yi} from appearing in between the reduplication by specifying what can appear in the list and what not.

%%%%%%%%%%%%%%%%%%%%
%%%%%%%%%%%%%%%%%%%%
%% YL
\item
The proposed analysis appears to have limitations in explaining the constraints on verbs and the observed modifications at the Aktionsart level. How can one determine which verbs are exempt from the proposed lexical rule? The proposed rule does not seem to provide a clear prediction of which verbs are allowed and which are excluded from reduplication. Why only some states and achievements are allowed? The restrictions on the usage of the progressive with reduplicated verbs also remain unexplained.

%\item
The proposed analysis appears to have a descriptive rather than an explanatory power.
%%%%%%%%%%%%%%%%%%%%
%%%%%%%%%%%%%%%%%%%%

%\textbf{Answer:} \itdopt{Stefan, say something. StMü: I would suggest to ignore this comment. }

%%%%%%%%%%%%%%%%%%%%%
\textbf{Answer:}
As explained in point \ref{verbclass} on page \pageref{verbclass}, we propose that all verbs regardless of their basic Aktionsart can be reduplicated given suitable context.
The incompatibility with the progressive aspect marker is explained in the Semantics section.
%%%%%%%%%%%%%%%%%%%%%
%%%%%%%%%%%%%%%%%%%%%


%%%%%%%%%%%%%%%%%%%%
%%%%%%%%%%%%%%%%%%%%
%% YL
\item
I would add tones to the pinyin transcription.

%%%%
\textbf{Answer:} done

%%%%%
%Other comments:
\item
p. 4: ex. (8b) is interesting, since it appears to involve the reduplication of V1 of a resultative compound. To the best of my knowledge, exiting literature on the subject consistently maintains that resultative compounds are not amenable to reduplication.

%%%%%%%%

\textbf{Answer:}A footnote is added to the example to point this out: 
``As a reviewer comments, this example also involves the reduplication of V1 of a resultative compound. 
This runs counter to exiting literature on this subject which consistently maintains that resultative compounds are not amenable to reduplication''.

%%%%%%%%
\item
p. 5, exx. (11): I believe it would be useful to provide also an example illustrating that reduplicated verbs are incompatible with (post-verbal) for X-time adverbials.
%%%%

\textbf{Answer:} See the answer to point \ref{verbclass} on page \pageref{verbclass}.

%%%%
\item
p. 18, exx. (39b): to the best of my knowledge, the object can be placed before —F yixia only in limited cases (pronouns and proper nouns).
%%%%%%%%%%%%%

\noindent
\textbf{Answer:}
There are some examples from the corpus showing that not only pronouns and proper nouns can be placed before \obj{yi2} \obj{xia4} `once':
\ea%她看了他的脸一下。(CCL)
\gll \objex{ta1} \objex{kan4-le} \objex{ta1de} \objex{lian3} \objex{yi2} \objex{xia4}.\\
she look-\textsc{pfv} his face one time\\\jambox{(CCL)}
\glt `She took a look at his face.'

\ex%他没碰窗户,就看了玻璃一下,玻璃就烂了。(CCL)
\gll \objex{ta1} \ldots\, \objex{jiu4} \objex{kan4-le} \objex{bo1li} \objex{yi2} \objex{xia4}, \objex{bo1li} \objex{jiu4} \objex{lan4-le}.\\
he {} just look-\textsc{pfv} glass one time glass just break-\textsc{pfv}\\\jambox{(CCL)}
\glt `He just took a look at the glass and the glass just broke.'
\z
We think it has to do with the weight/length of the object, i.e. shorter objects can be placed before  \obj{yi2} \obj{xia4} `once' while longer ones cannot.
%This is not necessarily a syntactic constraint

Even if we accept that the object can be only be placed before  \obj{yi2} \obj{xia4} `once' in limited cases,
the fact that the object can be placed before  \obj{yi2} \obj{xia4} `once' in some cases but in no cases in between reduplication suggests there is a fundamental difference between the structure of the two phenomena.

%%%%%%%%%%%%%%
\item
p. 20, lines 28--31: “Arcodia et al. (2014) and Basciano \& Melloni (2017) assumed that the first
element in the reduplication is the actual verb, which resides under init and proc, and that the
second element is an aspect marker, which resides in the complement position of proc, as it delimits
the process of the event”. Based on my understanding, these authors do not claim that the second
element serves as an aspect marker. They propose that the verb and the reduplicant are the spell-out
of two copies of the same lexical item within the vP domain: the reduplicant is the lower copy, in
complement position, and serves as an event delimiter. In this analysis, reduplication affects the Aktionsart of the verb.

%%%%%%%%%%%%%
\noindent
\textbf{Answer:} We reformulated as ``The second element is the copy of the verb, which resides in the complement position of proc and serves as an event delimiter.''

%%%%%%%%%%%%%
\item
p. 32, lines 28--34: “Although these forms are generally considered unacceptable (Li \& Thompson 1981: 30; Hong 1999: 275-276; Basciano \& Melloni 2017: 160; Yang \& Wei 2017: 239), Fan (1964: 269) and Sui (2018: 143) considered AB-yi-AB and AB-le -yi-AB to be possible, even though they both recognized that these two forms are rare.” Given that these patterns were possible in previous stages of the language (see a.0. Basciano \& Melloni 2017), these rare occurrences can be seen as relics of this usage.

%%%%%%%%%%%%%%%%%%%%%%
\textbf{Answer:} A footnote is added to this sentence to point this out: 
``One reviewer suggests that given that AB-\obj{yi}-AB and AB-\obj{le}-\obj{yi}-AB were possible in previous stages of the language (see \citealt[15]{Zhang2000}, \citealt[160--161]{BascianoMelloni2017}), these rare occurrences can be seen as relics of this usage''.

%\itdopt{St.Mü.: Did you add a footnote or is this something to be done?}
%%%%%%%%%%%%%%%%%%%%%%
\end{enumerate}


\section{Reviewer 2}\label{sec:2}
\begin{enumerate}
\item
This paper presents an innovative analysis of verb reduplication in Mandarin Chinese, demonstrating
clarity and well-organized sections. To enhance clarity, I recommend providing a brief illustration for
technical terms like LTOP, catering to readers less familiar with them.
The authors employ various tests to assert that verb reduplication exhibits more word-like
characteristics than phrase-like, hence justifying its morphological classification. The subsequent
adoption of a Head-Driven Phrase Structure Grammar (HPSG) framework, integrating Minimal
Recursion Semantics (Copestake et al., 2005), presents advantages over prior analyses.
Despite these strengths, certain questions warrant attention. The primary concern revolves around
treating verb reduplication as morphological. The authors compare it with Parallel Verb Compound
(PVC) and Serial Verb Construction (SVC), asserting that verb reduplication aligns more closely with
SVC than PVC in tests like phrase extension, substitution, and conjunction reduction. However, the
choice of SVC over regular verb phrases as a point of comparison with PVC and verb reduplication
lacks clarity. While SVC passes these tests, it is crucial to note that regular verb phrases may not pass
some, e.g., separate modification of subparts within a constituent.
In light of this, concluding that verb reduplication is more word-like and less phrase-like is
questionable, as regular verb phrases encounter similar challenges. Therefore, it is crucial to
explicitly justify why SVC was chosen over regular verb phrases for this comparison. Providing such
clarification will strengthen the robustness of the paper's argumentation and render the conclusion
that verb reduplication should be morphological more convincing.
For additional comments, please refer to the attached PDF file.

\itdopt{St.Mü.: This is the original review. We should say something about it. For example point to
  explanations above or say that the issues will be taken up below when we discuss the reviewers
  detailed comments.}


%%%%%%%%%%%%%%%%
%%%%%%%%%%%%%%%%%
%% YL
\item
R2 comments on the following passage: A-\obj{kan4-kan4}

I am curious about the acceptability of this form 'A-kan-kan.' For instance, phrases like 'shuo-kan-kan' ('say-look-look') or 'ting-kan-kan' ('listen-look-look') do not seem acceptable to me. 

\textbf{Answer:}
The acceptability of the A-\obj{kan4-kan4} form seems to vary among Mandarin Chinese speakers.
\citet[73]{Cheng2012} suggests that unlike AA-\obj{kan4}, A-\obj{kan4-kan4} is an emerging construction that is still undergoing grammaticalization.
He found in the Academia Sinica Balanced Corpus of Modern Chinese (Sinica Corpus, \citealt{sinica}) 23 tokens of A-\obj{kan4-kan4} but 141 tokens of AA-\obj{kan4} (p. 64).
This shows that the former is less common than the latter.
(\ref{ex:Akankan}) is an example of A-\obj{kan4-kan4} from the Sinica Corpus.
\ea\label{ex:Akankan} %就順著北風的吹勢,試看看能否將海沙攔下來吧!
\gll \objex{jiu4} \objex{shun4-zhe} \objex{bei3-feng1-de} \objex{chui1-shi4}, \obj{shi4-kan4-kan4} \objex{neng2} \objex{fou3} \objex{jiang1} \objex{hai3-sha1} \objex{lan2-xia4lai2} \objex{ba}!\\
just follow-\textsc{prog} north-wind-\textsc{de} blow-direction try-look-look can not make sea-sand block-down \textsc{ptc}\\\jambox{(Sinica Corpus)}
\glt `Let's just follow the north wind, try and see if we can block the sand!'
\z 

\citegen{Cheng2012} study is based on the Mandarin spoken in Taiwan.
We might assume that A-\obj{kan4-kan4} is more widely used in Taiwan than in Mainland China.
But we also found the following examples of A-\obj{kan4-kan4} in the BCC corpus \citep{BCC}, which includes mostly data from Mainland China.
This suggests that A-\obj{kan4-kan4} is acceptable for and used by at least some Mainland Chinese Mandarin speakers as well.
\ea %那么,你倒说看看 ,你会有什么样的改变\ldots \jambox{(BCC)}
\gll \objex{na4me}, \objex{ni3} \objex{dao4} \obj{shuo1-kan4-kan4}, \objex{ni3} \objex{hui4} \objex{you3} \objex{shen2meyang4de} \objex{gai3bian4}\ldots\\
then you just say-look-look you will have what.kind.of change\\\jambox{(BCC)}
\glt `Then you just try and say what kind of changes you will have.'
\z

\ea %我得听看看你的条件再说\ldots \jambox{(BCC)}
\gll \objex{wo3} \objex{dei3} \obj{ting1-kan4-kan4} \objex{ni3de} \objex{tiao2jian4} \objex{zai4} \objex{shuo1}\ldots\\
I have.to listen-look-look your condition then say\\\jambox{(BCC)}
\glt `I'll have to listen to your conditions first before talking about it/ deciding.'
\z

A statistical comparison of the acceptability and the productivity of A-\obj{kan4-kan4} among speakers of different varieties of Mandarin Chinese has to be left for future studies.

\item
R2 comments on the following passage: The current study will only focus on the AA, A-yi-A, A-le-A, A-le-yi-A, ABAB and
AB-le-AB forms of verbal reduplication in Mandarin Chinese. AA-kan, A-kan-kan,
AAB, A-yi-AB, A-le-AB will also be mentioned.

One of the key questions addressed in this paper pertains to whether verb reduplication is best understood as a morphological or a syntactic phenomenon. However, it's noteworthy that the forms emphasized in this paper are all instances of syntactic verb reduplication (Xie, 2020).

To further enrich the ongoing debate, I wonder whether the authors will consider the inclusion of examples that previous studies have identified as morphological verb reduplications. This addition would help demonstrate the broader scope and applicability of your analysis. Alternatively, it would be beneficial to clarify the reasons for selecting these specific forms in this article.

%%%%%%%%%%%%%%%%
\noindent
\textbf{Answer:}
\citet{Xie2020} identifies the AABB form as morphological verb reduplication.
Even though we do not agree with \citegen{Xie2020} claim that the distinction between ABAB and AABB lies in the distinction between morphology and syntax,
we do agree that they are different grammatical processes and thus requires separate analyses (See also \citealt[Sec. 4.3]{Deng2013}, \citealt{Wang2023}).
Also note that there are numerous scholars that do consider the forms analyzed here as morphological processes (\citealt[Ch. 4]{Chao1968}, \citealt[Ch. 3]{LiThompson1981}, \citealt[4--5]{Liao2014} and \citealt[Sec. 4]{Sui2018}).
\citet[14--15]{Zhang2000} argues that the AABB form of reduplication develops from a completely separate historical origin from the AA and ABAB forms,
as the former has a intensifying rather than delimitative meaning.
\citet[83--84]{Xie2020} further points out that AABB is much less productive than ABAB.
This suggests that at least some AABB might not be formulated by rules but is stored in the lexicon as a whole.
This is probably the case for those AABBs that have an idiomatic meaning, such as \obj{zhi2-zhi3-dian2-dian3} `point-point-gesture-gesture, gossip'.
%Productively formed AABBs can be accounted for by the current analysis as the reduplication happens before the compounding process,
%i.e. monosyllabic verbs are reduplicated separately (as AA and BB respectively) and then compounded together to form AABB.
%This is in line with the proposal in \citet[Sec. 4.2]{Wang2023},
%and also explains why some AABBs do not have a base AB form, such as \textit{zou-zou-ting-ting} `walk-walk-stop-stop, walk for a while and rest for a while', *\textit{zou-ting} `walk-stop'
The current analysis has a broad scope in that it explains all forms of delimitative verbal reduplication, which previous analyses were not able to do.
An analysis for the AABB form of reduplication has to be left for further research.

\itdopt{St. Mü: Können wir das nicht auch AABB? Das ist gar nicht schwer. Ich habe das für
  adverbiale Adjektive bereits implementiert. Man muss nur PHON in zwei Teile \ibox{1} und \ibox{2}
  spalten und dann eben \ibox{1} $\oplus$ \ibox{1} $\oplus$ \ibox{2} $\oplus$ \ibox{2} als PHON-Wert
nehmen. Wenn das wichtig ist, können wir das noch machen. Oder wir machen eine Fußnote hin.}

%%%%%%%%%%%%%%%%%
%%%%%%%%%%%%%%%%%
%% YL
\item
R2 comments on the following passage: The meaning of the reduplication is not compositional, as it does
not mean that the event denoted by the verb happens twice or multiple times, but
rather that the event happens for a short duration and/or for low frequency. This
non-compositionality suggests that a reduplication is more word-like.



Example (26) was employed to demonstrate that verb reduplication can be modified by an adverb as a whole but cannot have a part of it modified. The authors utilize this observation to bolster the argument that verb reduplication is a morphological, not a syntactic phenomenon.

However, the issue at hand is that even a standard verb phrase cannot be modified in this manner. Consider the following examples, in which the phrase 'dakai men' is a phrase rather than a word.

(1) ta qingsheng de dakai men.
       he  quietly    DE open door
      'He opened the door quietly.'

(2) * ta dakai qingsheng de men.
         he open quietly       DE door
         'He opened the door quietly.'

The key consideration here is that the unacceptability of (26b) and examples such as (2) may not hinge solely on whether part of it can be modified. Instead, it could be associated with the post-verbal placement of the adverb.

Consequently, it becomes challenging to use these examples as support for the argument that verb reduplication should be classified as a morphological rather than a syntactic phenomenon. This is because even regular verb phrases cannot be partially modified in the same manner.

%\textbf{Answer:} %Yes, this is true.

Examples (27) and (28) serve as illustrations to demonstrate the application of the ``Lexical Integrity Hypothesis'' in distinguishing PVC from SVC.

In reference to (25), it is explained that the Lexical Integrity Hypothesis posits that 'No
phrase-level rule may affect a proper subpart of a word.' However, it is noteworthy that the authors
employ le-insertion in (27) and (28), indicating that le- cannot be inserted into PVC but can be
inserted into SVC. This might suggest that the use of le-insertion in these examples could be
construed as a form of phrase-level operation, potentially conflicting with the author's perspective
of regarding 'le' as a morphological process.

Regarding example (34), the conjunctive structures do not exhibit conjunction reduction, a phenomenon that typically requires a degree of overlap between two elements, as shown in (32). Example (34) presents a scenario where there is no overlap between the verb structures.

Moreover, in Mandarin, conjunctions can only be employed with nouns and not with verbs phrases or
clauses (Schafer, 2009). If this holds true, it is less likely to observe conjunction reduction for
verb phrases. Consequently, it would be inappropriate to assert that verb reduplication is more
word-like and less phrase-like, as even regular verb phrases may encounter problems undergoing
conjunction reduction.

In Section 2.4, the authors asserted that verb reduplication is more phrase-like than word-like based on its similarities with Parallel Verb Compounds as apposed to Serial Verb Construction (SVC).

However, it is crucial to acknowledge that SVC has distinctive features, some of which diverge from those of a standard verb phrase. This raises a concern that a verb phrase might not successfully pass certain tests applicable to SVC, such as the ability for a constituent to be partially modified (A regular verb phrase like 'Verb + Object' cannot be partially modified by an adverb, while each verb in an SVC structure can be modified separately (28)). Therefore, when verb reduplication cannot be partially modified, asserting that it is less phrase-like solely on the basis of its differences from SVC might be inappropriate. This is because even a regular verb phrase faces limitations.

In light of this, it becomes essential to clarify why SVC, rather than regular verb phrases, was
chosen as a comparison to Parallel Verb Compounds. Such clarification would enhance the
persuasiveness of the tests presented in this section.

%%%%%%%%%%%%%%%%%%%
\textbf{Answer:} Based on the suggestions by the reviewers and further literature research, this section is drastically revised.
The different syntactic behaviors of the reduplication and a VP are compared
and the behavior of the reduplication is compatible with a lexical analysis.
The comparison with Parallel Verb Compounds and Serial Verb Constructions are not substantial to our arguments and is thus removed.
See the answer to Reviewer~\ref{sec:1}, point~\ref{morph} on page~\pageref{morph} for more details. 

%\itdopt{St. Mü.: We have numbers now due to enumerate. Please specify the point you are referring to exactly.}

%%%%%%%%%%%%%%%%%%%
%%%%%%%%%%%%%%%%%%%


%%%%%%%%%%%%%%%%%%%
%%%%%%%%%%%%%%%%%%%
%% YL
\item
The reason for the loss of idiomatic meaning in (43b) but not in (43c) could be attributed to the
presence of the number 'san' in (43b). If 'san xia' is replaced with 'yi xia,' which remains a
verbal classifier, the idiomatic meaning is perserved. In this case, there appears to be no
distinction between verbal classifier and reduplication.

%%%%%%%%%%%%%%%%%%%%
\noindent
\textbf{Answer}:
It is not \obj{san1} \obj{xia4} `three times' that is special, but it is \obj{yi2} \obj{xia4} `once' that is special.
We can replace \obj{san1} `three' with any number above two,
and the distinction still exist.
But \obj{yi2} \obj{xia4} `once' has acquired a duration reading `for a little while' that is not available to the other event quantifiers formed by \obj{xia4}, which only have the `for X times' interpretation (\citealt[77]{Deng2013}, \citealt[16]{Zhang2000}).
We think \obj{liang3} \obj{xia4} `twice' is following this tendency, too.
In this case, it is easy to interpret \obj{yi1} \obj{xia4} and \obj{liang3} \obj{xia4} not as referring to the actual number of action taking place,
but as duration adverbials as a whole,
thus differing them from ``actual'' verbal classifier phrases with other numerals.
%%%%%%%%%%%%%%%%%%%%
%%%%%%%%%%%%%%%%%%%%

%%%%%%%%%%%%%%%%%%%%
%%%%%%%%%%%%%%%%%%%%
%% Stefan
\item
It would enhance reader comprehension if the authors could provide a brief illustration of symbols
like HOOK, LTOP, IND, ARG, etc. This will aid readers in better grasping the authors' subsequent
proposals in Section 4.
\itdopt{Stefan}

\item
Perhaps PHON is missed in LEX-DTR of (57), even though PHON is empty in (57).

\noindent
\textbf{Answer:} PHON is not missing. The constraints regarding PHON on the mother and the daughter are coming from
another constraint. A constraint that is a subtype in the hierarchy. The type
perfective-reduplication-lr in (51), which says that the PHON value of the daughter \ibox{1} appears
twice in the PHON value of the mother with some other material (\etag) between the two occurrences of \ibox{1}.
%%%%%%%%%%%%%%%%%%%%
%%%%%%%%%%%%%%%%%%%%

\end{enumerate}

\section{Reviewer 3}

\begin{enumerate}
%%%%%%%%%%%%%%%%%%%%%
%%%%%%%%%%%%%%%%%%%%%
%% YL
\item
In Page 2, the draft says that CCL corpus was explored in addition to the author(s)’ intuition. I think this corpus-based method to bolster the acceptability judgments of the verbal reduplication expressions would be one of the merits in the current draft. The author(s) might be better to say more about why the corpus data was additionally explored and what was the findings from the corpus data (in comparison to the introspective judgments).

\textbf{Answer:}
 Corpus data contains a variation of linguistic properties, which help us obtain a better picture of the phenomenon under study,
 and discover relevant constraints that can otherwise go unnoticed through introspection.
 Corpus data is generally natural and contextualized, which can be helpful when evaluating the examples through introspection \citep[921]{MM2009a}.
 In the case of this study, the corpus data shows that certain syntactic and semantic constraints that previous studies claim for reduplication in fact do not apply.
Example sentences retrieved from the corpora are marked with their sources (``CCL'' or ``BCC'') behind them,
as opposed to handcraft examples, which are not marked.

\item
From Pages 4 to 5, the draft provides thorough overview of the subtypes of verbal reduplication in Mandarin Chinese. Yet, I think the examples would be too heavy for now. I am not sure whether all the examples are connected to the analyses in the later sections.

\textbf{Answer:} The examples are shortened to improve readability.

\item 
In Page 7 (section 2.3.2), the draft analyzes verbal reduplication with respect to Aktionsarten. Nonetheless, I am not well-aware of the concept. The author(s) might be better to account for the concept first in more detail.

\textbf{Answer:} 
the notion of ``aspect'' adopted in this article has to be explained in more detail.
We adopt the two-component aspect model proposed by \citet{XiaoMcEnery2004} based on \citet{Smith1991}.
The general term ``aspect'' is considered to encompass the following two components:
situation aspect, i.e. ``aspectual information conveyed by the inherent semantic representation of a verb or an idealized situation'' \citep[21]{XiaoMcEnery2004};
and viewpoint aspect, i.e. ``the aspectual information reflected by the temporal perspective the speaker takes in presenting a situation'' \citep[21]{XiaoMcEnery2004}.
Situation aspect can be further modeled as verb classes (dubbed ``Aktionsart'' in the previous version of this paper) at the lexical level
and situation types (the interaction of verb classes and other constituents) at the sentential level \citep[33]{XiaoMcEnery2004}.
The verb classes are determined with verbs in a neutral context (preferably in a perfective viewpoint aspect, with a simple object only when it is obligatory),
where everything that might change the aspectual value of a verb is excluded
and only the inherent features of the verb itself are considered
(see \citealt[52]{XiaoMcEnery2004} for more details).
This does not rule out the fact that the same verb may express different aspectual properties in other contexts,
but its verb class remains the same,
as the aspectual change can be attributable to other components at the sentential level.
This section is concerned with the interaction of the reduplication with verb classes,
while the next section discusses the interaction with viewpoint aspect markers.
The combination with other constituents such as adjuncts is briefly touched upon below,
but a detailed discussion has to be left for future research.

\itdopt{St. Mü.: We have numbers now due to enumerate. Please specify the point you are referring to exactly.}

\item
In Page 13,
there is a typo: ... (but see Section 2.3.2). in comparison, ...

\textbf{Answer:} We fixed the typo.

\item
In the middle, the draft mentions the -bar derivation and be-prefixation in German. I understand the phenomenon has a similarity to verbal reduplication in Chinese. Yet, at a glance the German examples looked out-of-the-blue to me.

\textbf{Answer:} The German examples are used to illustrate the arguments that morphological processes can be productive and does not necessarily change the category or valency of the input.
To improve readability, English examples of ``-able'' derivation and ``-ed'' inflection are used instead now.

\item
The following paragraph looks terse to me. The author(s) want to provide more explanation about the four tests. In addition, the footnote 6 might be better to move into the body.
Duanmu (1998) and Schäfer (2009) proposed the following four tests to distinguish words from phrases
in Mandarin Chinese: semantic compositionality, phrasal extension, phrasal substitution and
conjunction reduction.

\textbf{Answer:} 
Based on the suggestions by the reviewers and further literature research, this section is drastically revised.
The tests of semantic compositionality and conjunction reduction are deemed unreliable,
and only phrasal extension (or ``modification'' according to \citealt[32]{Dai1992}, \citeyear[117]{Dai1998})
and phrasal substitution (or ``expansion'' according to \citealt[33]{Dai1992}, \citeyear[117--120]{Dai1998}) remain.
The explanations to the tests are simplified and illustrated with examples.
See the answer to Reviewer \ref{sec:1}, point \ref{morph} on page \pageref{morph} for more detail.

%\itdopt{St. Mü.: We have numbers now due to enumerate. Please specify the point you are referring to
%  exactly. For example: See the answers to Reviewer X, point Y on page Z. If you did something in
%  the text to react, say so. This pleases reviewers. They are the kings/queens and want to be take seriously.}

\item
In Page 14, the draft makes use of a new terminology such as ``word-like''. Of course, I can understand the meaning within the context. However, I think the author(s) might be better to use a more academic and formal terminology.

%%%%%%%%%%%%%%%%%%%%%%
\textbf{Answer:} We reformulate ``word-like'' as expressions such as ``behaves more similar to a word'', ``is compatible with a lexical analysis''.

%\itdopt{St. Mü.: Was hast Du denn hier gemacht? Das scheint im Text jetzt ganz auskommentiert zu sein.}

%%%%%%%%%%%%%%%%%%%%%%
\item
Section 3 (Previous Analysis) appears in Page 17. I think it would be a little bit late. The author(s) may have a plan to restructure Sections 2 and 3 again.

%%%%%%%%%%%%%%%%%%%%%%
%%%%%%%%%%%%%%%%%%%%%%
%% YL
\noindent
\textbf{Answer: }Since the discussion on the pros and cons of previous analyses relies heavily on understanding the syntactic and semantic constraints of reduplication, we believe that it is necessary to discuss the phenomenon thoroughly before discussing previous analyses.

%%%%%%%%%%%%%%%%%%%%%%%
%%%%%%%%%%%%%%%%%%%%%%%
%% Stefan
\item
In Page 26, the draft introduces a new notion (i.e., an empty rectangle) into the AVMs. Is it what
the author(s) invent only for modeling the verbal reduplication in this drat? If so, the author(s)
want to account more for how it works in AVMs. Otherwise, the author(s) might be better to refer to
the previous analysis about the notion.

\noindent
\textbf{Answer: } We added a footnote on first appearance explaining that the \etag is a more
explicit version of the usual ``\ldots''.


\item
In Section 4, I am expecting the author(s) provide a sample derivation (maybe a tree diagram). The draft provides the subsets about how verbal reduplication can be constrained within the AVMs. On top of them, a full derivation can be made for better understanding.
%%%%%%%%%%%%%%%%%%%%%%%
%%%%%%%%%%%%%%%%%%%%%%%

%\itdopt{Stefan. Question to Yanru: Which example should we analyze?}
\noindent
\textbf{Answer: } We added a figure with the analysis of \obj{chang2-le-chang2} and some
accompanying text.

\end{enumerate}

{\sloppy
\printbibliography[heading=subbibliography,notkeyword=this]
}
\end{document}



%      <!-- Local IspellDict: en_US-w_accents -->

\section{Introduction}\label{sec:intro}

In Mandarin Chinese, verbs can be reduplicated to express a delimitative aspectual meaning (e.g. \citealt[204--205]{Chao1968}; \citealt[232]{LiThompson1981};  \citealt[14]{Li1996}; \citealt[70]{Dai1997};  \citealt[382--383]{Zhu1998}; \citealt[420--421]{Xing2000}; \citealt[48]{Chen2001};  \citealt[288]{Tsao2001}; \citealt[11--12]{Yang2003}; \citealt[Sec. 4.3]{XiaoMcEnery2004}). 
This means that the event or state denoted by the verb happens in a short duration and/or a low frequency \citep[155]{XiaoMcEnery2004}, such as illustrated in (\ref{ex:redup-ex}).\footnote{Reduplications
    in the example sentences will be set in italics.}
Thus, verbal reduplication in Mandarin Chinese is often translated as doing something ``a little bit/for a little while''.

\ea\label{ex:redup-ex} 
	\ea
	\gll qing ni chang zhe dao cai.\\
	please you taste this \textsc{clf} dish\\
	\glt `Please taste this dish.'
	
	\ex
	\gll qing ni \textit{chang}-\textit{chang} zhe dao cai.\\
	please you taste-taste this \textsc{clf} dish\\
	\glt `Please taste this dish a little bit.' 
	\z
\z

The current study tries to determine a suitable formal and unified analysis for the structure of verbal reduplication in Mandarin Chinese.
It contributes more empirical evidence and 
offers a novel analysis in the theoretical framework of Head-driven Phrase Structure Grammar (HPSG; \citealt{PollardSag1994, Sag1997, HPSGHandbook}) to this phenomenon 
using Minimal Recursion Semantics (MRS; \citealt{Copestakeetal2005}) as the semantic representation formalism.
This new account avoids the problems of previous approaches and
explains more forms of verbal reduplication in Mandarin Chinese.

This paper is organized as follows: 
after this introduction, we will present in \sectref{sec:phen} the forms and syntactic distribution as well as the semantics of verbal reduplication in Mandarin Chinese. 
Importantly, we restrict the object of this study to the AA, A-\textit{yi}-A, A-\textit{le}-A, A-\textit{le}-\textit{yi}-A, ABAB and AB-\textit{le}-AB forms of verbal reduplication in Mandarin Chinese.
We will also discuss in this section, with the help of corpus data, the question of whether the reduplication is a morphological or a syntactic process.
In \sectref{sec:prev}, we will discuss the advantages and drawbacks of previous approaches. 
\sectref{sec:analysis} will present a new {HPSG} account for verbal reduplication in Mandarin Chinese.
Finally,  \sectref{sec:conclu} will conclude the paper.

The data in this paper was drawn from several sources.
In addition to introspection, the Modern Chinese subcorpus of the corpus of the \textit{Center for Chinese Linguistics of Peking University} ({CCL}) \citep{Zhanetal2003, Zhanetal2019} was also consulted. 
Further, examples from novels and plays written by native speakers were considered.
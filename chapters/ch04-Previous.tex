\section{Previous analyses}\label{sec:prev}

Previous analyses on the reduplication in Mandarin Chinese and in other languages can be classified
into three groups:  the reduplication as a verbal classifier phrase (\sectref{sec:clf}), as an aspect modifier (\sectref{sec:asp}), and
as a special reduplication construction (\sectref{sec:construc}).\footnote{The term \textit{construction} is used here in its general sense, not in the sense of Construction Grammar.} 
This section will review these analyses and will discuss their advantages and shortcomings.



%%%%%%%%%%%
\subsection{The reduplication as a verbal classifier phrase}\label{sec:clf}

\citet{Fan1964}, \citet[205]{Chao1968} and \citet{Xiong2016} analyze the reduplication in Mandarin Chinese
as a verbal classifier phrase\footnote{Alternative terms for \textit{verbal classifier}: \textit{measure for verbs of action} in \citet[615]{Chao1968} or \textit{cognate object} in \citet[312]{Chao1968} and  \citet[263]{Hong1999}.
The \textit{verbal classifier phrase}  is also termed \textit{quantity adverbial} in \citet[352--353]{LiThompson1981} or \textit{frequency phrase} in \citet[91]{Huangetal2009}.}.
A verbal classifier is ``a measure for verbs of action expresses the number of times an action takes place” \citep[615]{Chao1968}, such as the \obj{ci4} in (\ref{ex:clf}). 

%\settowidth\jamwidth{(Chomsky, 1957: 15)}
\ea\label{ex:clf}
\gll \objex{mei2} \objex{che3}-\objex{guo} \objex{yi2} \objex{ci4} \objex{huang3}\\
not tell-\textsc{exp} one \textsc{clf} lie\\ \hfill {\citep[616]{Chao1968}}
\glt `haven't told lie once'
\z

In this analysis, the first element in the reduplication is the actual verb, 
the second element is a verbal classifier borrowed from a verb, 
and \obj{yi} `one' is an optional pseudo-numeral that only has an abstract `a little bit' meaning.
The analysis is syntactic.

The parallel between the reduplication and a verbal classifier phrase is obvious. 
Both the reduplication and the verbal classifier phrase serve to quantify the duration or the extent of a situation. 
A reduplication structure can often be paraphrased into a verbal classifier phrase such as \obj{yi2} \obj{xia4} `once, a while', \obj{yi2} \obj{hui4} `a while', as illustrated in (\ref{ex:redup-clf}).

\ea\label{ex:redup-clf}
	\ea \gll \obj{deng3}-\obj{deng3} \objex{wo3}.\\
		wait-wait I\\
		\glt `Wait for me a little bit.'
	\ex \gll \objex{deng3} \objex{wo3} \objex{yi2} \objex{xia4}.\\
	wait I one \textsc{clf}\\
	\glt `Wait for me a little while.'
	\z
\z

However, there are several arguments suggesting that the reduplication cannot be analyzed the same way as a verbal classifier phrase. First, the verb and the verbal classifier can be separated (\ref{ex:clf-sep}), while the reduplication cannot (\ref{ex:redup-sep}) \citep[269]{Paris2013}.\footnote{
One reviewer notes that the objects that can be placed in between the verbal classifier phrase is usually limited to pronouns and proper nouns.
We found some examples from the corpus showing that not only pronouns and proper nouns can be placed before \obj{yi2} \obj{xia4} `once':
\ea%她看了他的脸一下。(CCL)
\gll \objex{ta1} \objex{kan4-le} \objex{ta1de} \objex{lian3} \objex{yi2} \objex{xia4}.\\
she look-\textsc{pfv} his face one \textsc{clf}\\\jambox{(CCL)}
\glt `She took a look at his face.'

\ex%他没碰窗户,就看了玻璃一下,玻璃就烂了。(CCL)
\gll \objex{ta1} \ldots\, \objex{jiu4} \objex{kan4-le} \objex{bo1li} \objex{yi2} \objex{xia4}, \objex{bo1li} \objex{jiu4} \objex{lan4-le}.\\
he {} just look-\textsc{pfv} glass one \textsc{clf} glass just break-\textsc{pfv}\\\jambox{(CCL)}
\glt `He just took a look at the glass and the glass just broke.'
\z
We think it has to do with the length of the object, i.e. shorter objects can be placed before  \obj{yi2} \obj{xia4} `once' while longer ones cannot.
%This is not necessarily a syntactic constraint

Even if we accept that the object can be only be placed before  \obj{yi2} \obj{xia4} `once' in limited cases,
the fact that the object can be placed before  \obj{yi2} \obj{xia4} `once' in some cases but in no cases in between reduplication suggests there is a fundamental difference between the structure of the two phenomena.
}

\ea\label{ex:clf-sep}
  \ea \gll \objex{ni3} \objex{deng3} \objex{yi2} \objex{xia4} \objex{Zhang1san1}!\\
   you wait one \textsc{clf} Zhangsan\\
   \glt `Wait for Zhangsan for a while!'
  \ex \gll \objex{ni3} \objex{deng3} \objex{Zhang1san1} \objex{yi2} \objex{xia4}!\\
   you wait Zhangsan one \textsc{clf}\\
   \glt `Wait for Zhangsan for a while!'
\z
\z

\ea\label{ex:redup-sep}
  \ea[]{\gll \objex{ni3} \obj{deng3}-\obj{yi}-\obj{deng3} \objex{Zhang1san1}!\\
   you wait-one-wait Zhangsan\\
   \glt `Wait for Zhangsan a little bit!'}
  \ex[*]{\gll \objex{ni3} \obj{deng3} \objex{Zhang1san1} \obj{yi} \obj{deng3}!\\
  you wait Zhangsan one wait\\}
\z
\z

Second, unlike verbal classifiers (\ref{ex:clf-num}), the \obj{yi} ‘one’ in A-\obj{yi}-A cannot be replaced by other numerals (\ref{ex:redup-num}) \citep[299--230]{YangWei2017}.

\ea\label{ex:clf-num}
  \ea \gll \objex{ta1} \objex{pai1}-le \objex{wo3} \objex{yi2} \objex{xia4}.\\
   he pat-\textsc{pfv} I one \textsc{clf}\\
   \glt `He patted me once.'
   \ex \gll \objex{ta1} \objex{pai1}-le \objex{wo3} \objex{liang3} \objex{xia4}.\\
    he pat-\textsc{pfv} I two \textsc{clf}\\
    \glt `He patted me twice.'
\z
\z

\ea\label{ex:redup-num}
  \ea[]{\gll \objex{ta1} \obj{pai1}-\obj{le}-\obj{yi}-\obj{pai1} \objex{wo3}.\\
   he pat-\textsc{pfv}-one-pat I\\
   \glt `He patted me a little bit.'}
   \ex[*]{\gll \objex{ta1} \obj{pai1}-\obj{le}-\obj{liang3}-\obj{pai1} \objex{wo3}.\\
   he pat-\textsc{pfv}-two-pat I\\}
\z
\z

Third, idioms (\ref{ex:idiom}) lose their idiomatic meaning when used with verbal classifiers (\ref{ex:clf-idiom}), but maintain their idiomatic meaning with reduplications (\ref{ex:redup-idiom}) \citep[230--231]{YangWei2017}.\footnote{
This applies except when \obj{yi2} \obj{xia4} `once' is used. 
Because of this exception, one reviewer suggests that the loss of the idiomatic meaning in (\ref{ex:clf-idiom}) 
but not in (\ref{ex:redup-idiom}) may be attributed to the use of the numeral \obj{san1} `three' in (\ref{ex:clf-idiom}),
because if \obj{san1} \obj{xia4} `three times' is replaced with \obj{yi2} \obj{xia4} `once' in (\ref{ex:clf-idiom}), he/she can still get the idiomatic interpretation.
We suggest that it is not \obj{san1} \obj{xia4} `three times' that is special, 
but it is \obj{yi2} \obj{xia4} `once' that is special.
We can replace \obj{san1} `three' with any number above two,
and the distinction still exist.
But \obj{yi2} \obj{xia4} `once' has acquired a duration reading `for a little while' that is not available to the other event quantifiers formed by \obj{xia4}, which only have the `for X times' interpretation (\citealt[77]{Deng2013}, \citealt[16]{Zhang2000}).
We think \obj{liang3} \obj{xia4} `twice' is following this tendency, too.
In this case, it is easy to interpret \obj{yi2} \obj{xia4} and \obj{liang3} \obj{xia4} not as referring to the actual number of action taking place,
but as duration adverbials as a whole,
thus differing them from ``actual'' verbal classifier phrases with other numerals.
}

\ea
  \ea\label{ex:idiom}
  \gll \objex{bao4} \objex{fo2}-\objex{jiao3}\\
  clasp Buddha-foot\\
   \glt Literal: `clasp the Buddha's foot'\\
   Idiomatic: `make a last-minute effort'
   
   \ex\label{ex:clf-idiom}
   \gll \objex{ta1} \objex{kao3shi4} \objex{qian2} \objex{bao4}-le \objex{san3} \objex{xia4} \objex{fo2}-\objex{jiao3}.\\
   he exam before clasp-\textsc{pfv} three \textsc{clf} Buddha-foot\\
   \glt `He clasped the Buddha's foot three times before the exam.' (idiomatic reading unavailable)
   
   \ex\label{ex:redup-idiom}
   \gll \objex{ta1} \objex{kao3shi4} \objex{qian2} \obj{bao4}-\obj{le}-\obj{bao4} \objex{fo2}-\objex{jiao3}.\\
   he exam before clasp-\textsc{pfv}-clasp Buddha-foot\\
   \glt Literal: `He clasped the Buddha's foot a little bit before the exam.'\\
   Idiomatic: `He made a bit of a last-minute effort before the exam.'
\z
\z

Based on these observations, it seems inappropriate to view the reduplication as a kind of verbal classifier phrase.






%%%%%%%%%%%
\subsection{The reduplicant as an aspect modifier}\label{sec:asp}

A number of studies consider the reduplication to be an element that modifies the aspectual properties of the base verb \citep{Arcodiaetal2014, BascianoMelloni2017, YangWei2017} 
due to the delimitative aspectual meaning of reduplication. 
\citet{Travis1999, Travis2000} also analyzes the reduplication in Tagalog as an imperfective aspect marker.
 
 
 
 
 %%%%%%
%Arcodiaetal2014, BascianoMelloni2017
 
 \citet{Arcodiaetal2014} and \citet{BascianoMelloni2017} analyze the reduplication within the
framework of First Phase Syntax \citep{Ramchand2008}.
\citet{Ramchand2008} proposes that an event is comprised of the following phrases: 
the causative subevent (\textit{init}P), the process subevent (\textit{proc}P) and the result subevent (\textit{res}P), which are ordered hierarchically, as illustrated in \figref{tree:ramchand}.\footnote{
The present study does not argue for a {NP} or a {DP} analysis and simply takes over the illustration provided in the cited papers.}
Dynamic and volitional verbs have the features [init proc] and are therefore located under \textit{init} and \textit{proc}  \citetext{\citealp[24]{Arcodiaetal2014}; \citealp[147]{BascianoMelloni2017}}.  
Achievement verbs possess the feature [res] and reside under \textit{res}  \citetext{\citealp[24]{Arcodiaetal2014}; \citealp[147]{BascianoMelloni2017}}. 
Stative verbs do not contain a \textit{proc}P \citep[152]{BascianoMelloni2017}.


\begin{figure}
 \centering
\begin{forest}
[\textit{init}P (causing projection) [DP$_3$\\subject of `cause']
  [ [init]
    [\textit{proc}P (process projection)
      [DP$_2$\\subject of `process']
      [[\textit{proc}]
        [\textit{res}P (result projection)
        [DP$_1$\\subject of `result']
          [[\textit{res}] [XP]
          ]
        ]
      ] ]
    ]
  ]
\end{forest} 
\caption{Event structure according to \citet[193]{Ramchand2008}}
\label{tree:ramchand}
\end{figure}

 
\citet{Arcodiaetal2014} and \citet{BascianoMelloni2017} assume that the first element in the reduplication is the actual verb, which resides under \textit{init} and \textit{proc}, 
and that the second element is the copy of the verb, which resides in the complement position of \textit{proc} and serves as an event delimiter.
Since the second element occupies the same syntactic position as \textit{res}P, it should have complementary distribution with \textit{res}P
and should thus be incompatible with achievement verbs because of their [res] feature.
Furthermore, if \textit{proc}P does not exist in the event, as in the case of states, there should be no place for the reduplication either.

This analysis correctly predicts that the reduplication of achievement verbs and stative verbs is not as easily acceptable as that of dynamic and volitional verbs (marked by [init, proc] features).

However, as shown in \sectref{sec:Aktionsarten}, the reduplication of states and achievements is unusual but not impossible.
This suggests that the reduced acceptability of reduplicated achievement and stative verbs is \textit{semantic} rather than \textit{structural}. 
Their use is possible in specific contexts and should not be ruled out \textit{syntactically}.
Consequently, this proposal does not seem to offer an appropriate account for  reduplication.
 
 
 
 
 %%%%%%
 %YangWei2017, Tsai2008
 
 \citet[229]{YangWei2017} endorse the analysis of  reduplication as an aspect marker following the structure of Mandarin Chinese aspects proposed by \citet{Tsai2008}.
 \citet{Tsai2008} provides the syntactic analysis for aspect markers in Mandarin Chinese as shown in \figref{tree:tsai}.\footnote{Asp = Aspect}
 He observes the so-called incompleteness effect, namely that a minimal sentence, 
 which only contains a verb marked by \obj{zhe} `\textsc{dur}',
 \obj{le} `\textsc{pfv}' or \obj{wan2} `\textsc{compl}' 
 and its arguments, 
 seems incomplete without further sentential elements 
 such as the sentence final particle \obj{le} 
 or a temporal adverbial like \obj{gang1cai2} `just now' (\ref{ex:incomplete}). 
 In contrast, a minimal sentence with a verb marked by  \obj{zai4} `\textsc{prog}' or \obj{guo}
 `\textsc{exp}'  and its arguments can stand alone (\ref{ex:complete}).
 
\ea\label{ex:incomplete}
 \gll \objex{Xiao3li3} \objex{chi1}-zhe/le/\objex{wan2} \objex{fan4} *(le).\\
 Xiaoli eat-\textsc{dur}/\textsc{pfv}/\textsc{compl} meal \textsc{ptc}\\
 \glt `Xiaoli is eating/ate/finished eating the meal.'
\z

\ea\label{ex:complete}
\gll \objex{Xiao3li3} (\objex{gang1cai2}) \objex{zai4} \objex{ku1}/\objex{ku1}-\objex{guo}.\\
Xiaoli just.now \textsc{prog} cry/cry-\textsc{exp}\\
\glt `Xiaoli was crying/cried just now.'
\z
He thus proposes three aspect positions under TP. \obj{zai4} `\textsc{prog}' and \obj{guo} `\textsc{exp}' reside under Asp$_1$, 
while \obj{zhe} `\textsc{dur}' and \obj{le} `\textsc{pfv}' under Asp$_2$, as illustrated in \figref{tree:tsai}.\footnote{
    \citet{Tsai2008} differentiated the middle and the inner aspect based on the fact that 
    \obj{wan2} can only occur with certain types of predicate.
     This differentiation does not play a role for our purpose and will not be further discussed here.}

\begin{figure}
\centering
\begin{forest}
[TP [T]
  [...
    [AspP$_1$ (outer aspect) [Asp$_1$\\\obj{zai4/guo}]
      [\textit{v}P [\textit{v}]
        [AspP$_2$ (middle aspect) [Asp$_2$\\\obj{zhe/le}]
          [VP [V-Asp$_3$ (inner aspect)\\\obj{wan2}]
          ]
        ]
      ]
    ]
  ]
]
\end{forest}
\caption{Structure of the aspectual system in Mandarin Chinese according to \citet[683]{Tsai2008}}
\label{tree:tsai}
\end{figure}

Turning to reduplication, a minimal sentence with reduplication also seems incomplete (\ref{ex:redup-tsai}).
Based on this, the reduplicant should reside under Asp$_2$, as illustrated in \figref{tree:redupasp}.

\ea\label{ex:redup-tsai}
\ea[*] {\gll \objex{ta1} \obj{xiao4}-\obj{xiao}\\
he laugh-laugh\\}
\ex[] {\gll \objex{ta1} \obj{xiao4}-\obj{le}-\obj{xiao4}\\
he laugh-\textsc{pfv}-laugh\\
\glt `He laughed a little.'}
\ex[] {\gll \objex{ta1} \obj{xiao4}-\obj{xiao}, \objex{bu4} \objex{shuo1hua4}\\
he laugh-laugh not speak\\
\glt `He laughed a little, and didn't speak.'}\label{ex:redup-tsai-co}
\z
\z

\begin{figure}
    \centering
    \begin{forest}
        [TP [T]
        [...
        [AspP$_1$ (outer aspect) [Asp$_1$\\\obj{zai4/guo}]
        [\textit{v}P [\textit{v}]
        [AspP$_2$ (middle aspect) [Asp$_2$\\\obj{zhe}/\obj{le}/reduplication]
        [VP [V-Asp$_3$ (inner aspect)\\\obj{wan2}]
        ]
        ]
        ]
        ]
        ]
        ]
    \end{forest}
    \caption{Position of reduplication according to the aspectual system in \citet{Tsai2008}}
    \label{tree:redupasp}
\end{figure}

This analysis would result in a mismatch between syntax and semantics, 
in the sense that the aspect markers that belong to the same semantic group do not occur in the same syntactic position. 
Even though \obj{le} `\textsc{pfv}', \obj{guo} `\textsc{exp}' and reduplication all mark perfective aspects (\sectref{sec:aspM}, \citealt{Dai1997, XiaoMcEnery2004}),
\obj{guo} `\textsc{exp}' is situated under Asp$_1$, while \obj{le} `\textsc{pfv}' and reduplication are under Asp$_2$.
Similarly, \obj{zai4} `\textsc{prog}' and \obj{zhe} `\textsc{dur}' are both imperfective aspects but also occur in different syntactic positions.

In sum, both analyses of the reduplicant as an aspect modifier do not seem to be convincing.




 
 
%%%%%%%%%%%%
\subsection{Reduplication construction}\label{sec:construc}
 
%The following section discusses previous literature which proposed a special construction for the reduplication.
 

 %%%%%%%
%Ghomeshietal2004

%\citet{Ghomeshietal2004} provided an analysis within the theoretical framework of Parallel Architecture \citep{Jackendoff1997, Jackendoff2002} for Contrastive Reduplication (CRs) in English like (\ref{ex:salad}),\footnote{Capitalization indicates contrastive intonation.} 
%as shown in Figure~\ref{ghomeshi-cr}.\footnote{{P} = phonological unit, {P/E/S CTR} = prototypical/extreme/salient contrast, XP$^{min}$ = XP without its specifier} 
%They postulated a new syntactic category {CR}$^{syn}$ to incorporate the reduplicant.
%It has the semantics of {P/E/S CTR} and an empty phonological content that copies from the phonology of the base it applies to.
%According to their observations, the base can be an X (e.g. \textit{SALAD-salad}) or an XP$^{min}$ (e.g. \textit{OVER-THE-HILL-over-the-hill}).
%In Figure~\ref{ghomeshi-cr}, {CR}$^{syn}$ is coindexed with the first {P}, 
%which is in turn coindexed with the phonology of X/XP$^{min}$ (P$_k$). 
%{CR}$^{syn}$ is coindexed with the semantics of {P/E/S CTR}, 
%which modifies the semantics of  X/XP$^{min}$ (Z$_k$).
%\ea\label{ex:salad}
%I make the tuna salad, and you make the \textit{SALAD-salad}.
%\z
%\begin{figure}[h!]
%\centering
%\begin{multicols}{3}
%\begin{minipage}[t]{.3\linewidth}
%\begin{center}
%Phonology\\
%\vspace{15pt}
%P$_{j, k}$ P$_k$
%\end{center}
%\end{minipage}
%\columnbreak
%\begin{minipage}[t]{.3\linewidth}
%\begin{center}
%Syntax\\
%\begin{forest}
%for tree = {inner sep = 0pt,
%	s sep = 1pt,
%	anchor=children last,
%    	align=center}
%[X/XP$^{min}$
% [CR$^{syn}$$_j$]
% [X/XP$^{min}$$_k$]
%]
%\end{forest}
%\end{center}
%\end{minipage}
%\columnbreak
%\begin{minipage}[t]{.3\linewidth}
%\begin{center}
%Semantics
%\[
%\begin{bmatrix}
%Z_{k}\\
%\begin{bmatrix}
%P/E/S\\
%CTR
%\end{bmatrix}_{\!j}
%\end{bmatrix}
%\]
%\end{center}
%\end{minipage}
%\end{multicols}
%\caption{Analysis for {CR}s in English according to \citet[344]{Ghomeshietal2004}}
%\label{ghomeshi-cr}
%\end{figure}
 
%Applying this to the reduplication in Mandarin Chinese, the structure should be something like \figref{ghomeshi-cn}.\footnote{{DELIM} = delimitative}\textsuperscript{,}\footnote{
%Although the reduplication in Mandarin Chinese does not have a contrastive meaning, 
%we preserve the notation of CR$^{syn}$ in \citet{Ghomeshietal2004} here to simply refer to the reduplicant.}
%The base verb is coindexed with its phonology and its semantics `look'. 
%The phonology of the reduplicant is coindexed with the phonology of the verb.
%The reduplicant is coindexed with the delimiative semantics
%and modifies the semantics of the base verb.
%The base verb and the reduplicant combine to a V.\footnote{In English, it makes sense to assume CR$^{syn}$ to be a syntactic unit, because the base can be XP$^{min}$ (e.g. \textit{OVER-THE-HILL-over-the-hill}). 
%But for Mandarin Chinese, the base can only be V.
%As \citet[353]{Ghomeshietal2004} wrote: ``when applying to its smallest scope, X inside of a word, it has the feel of other things that attach there, i.e., morphological affixes''.
%It seems that it suffices to assume the reduplication in Mandarin Chinese to be a morphological phenomenon (cf. \sectref{sec:word}).
%We continue to call the second column ``syntax'' to preserve the consistency of the notations.
%}
%We draw the head on the left because all analyses mentioned above considered the first element in the reduplication to be the actual verb and the second element the reduplicant, 
%but this analysis itself does not forbid the head to be on the right side.

%\begin{figure}[h!]
%\centering
%\begin{minipage}[t]{.3\linewidth}
%\begin{center}
%Phonology\\
%\begin{forest}
%[P$_{j}$ [kan]] 
%\end{forest}
%\begin{forest}
%[P$_{j, k}$ [kan]]
%\end{forest}
%\end{center}
%\end{minipage}
%\columnbreak
%\begin{minipage}[t]{.3\linewidth}
%\begin{center}
%Syntax\\
%\begin{forest}
%[V
% [V$_j$]
% [CR$^{syn}$$_k$]
%]
%\end{forest}
%\end{center}
%\end{minipage}
%\columnbreak
%\begin{minipage}[t]{.3\linewidth}
%\begin{center}
%Semantics
%\[
%\begin{bmatrix}
%\begin{bmatrix}
%DELIM\\([LOOK]_{j})
%\end{bmatrix}_{\!k}\\
%\end{bmatrix}
%\]
%\end{center}
%\end{minipage}
%\caption{Analysis for AA following \citet{Ghomeshietal2004}}
%\label{ghomeshi-cn}
%\end{figure}


%As for the other forms of reduplication, A-\textit{le}-A can be viewed as two compositional processes \mbox{[[[A] -\textit{le}] -A]}.
%Furthermore, the \textit{yi} in A-\textit{yi}-A and A-\textit{le}-\textit{yi}-A can simply be viewed as a dangling phonological unit. 
%In this case, \phonliste{ yi } is neither coindexed with a syntactic unit nor with a semantic one.

%This analysis correctly captures the fact that the addition of \textit{yi} does not change the syntactic and semantic behavior of the reduplication.
%It also provides a formal account for the phonology of the reduplication.
%Nevertheless, by assuming a construction specially for the reduplication, this approach fails to account for the similarities of the reduplication and other aspect markers in Mandarin Chinese, unlike the affixation analyses described in \sectref{sec:asp}.

 
 
 
 
 
 %%%%%%%%%%%%%%
 %FanSongBond2015
 
\citet*{FanSongBond2015} provide a unified HPSG analysis for the reduplication of both verbs and adjectives in Mandarin Chinese.
They consider reduplication to be a morphological process and model it via lexical rules.
They provide the lexical rule (\ref{avm:fsb-redup}) for reduplication in general,
and further propose \type{redup-a-lr} and \type{redup-v-lr} as subtypes of \type{redup-type}, 
as illustrated in (\ref{avm:fsb-redup-a}) and (\ref{avm:fsb-redup-v}) respectively.\footnote{%
  Note that the format in which the lexical rule is given is not the input--output format usually
  used for lexical rules in HPSG. Instead it is depicted as a unary branching phrase structure
  rule: the input is the daughter on the right-hand side of the rule. The output is the mother on
  the left-hand side. The view of lexical rules as unary branching rules is adopted in most current work on
  lexical rules \citep{BC99a,Meurers2001a} despite the notation that is commonly assumed.  

  \textsc{hook} is a technical feature for sharing information. \textsc{ltop} and \textsc{lbl} will
  be explained in Section~\ref{sec:analysis} below.
}
For them, the reduplication functions as an intensifier predicate, as represented in the \textsc{predicate (pred)} in the \textsc{constructional-content (c-cont)}.
The \type{intensifier\_x\_rel} has two subtypes: \type{redup\_up\_x\_rel} for the amplifiying meaning of adjectival reduplication and \type{redup\_down\_x\_rel} for the delimitative meaning of verbal reduplication.
The orthography is handled separately.
The AABB form for adjectives and the ABAB form for verbs, as well as the AAB form for V-O compounds, are handled as irregular derivation forms.


\ea\label{avm:fsb-redup}
\avm{
[\type*{redup-type}\\
cat\textbar head & \1\\
val & \2\\
cont & \3 \normalfont \textsc{hook} [ltop & \4\\
			ind & \5]\\
c-cont & <[\type*{event-rel}\\
		pred & intensifier\_x\_rel\\
		lbl & \4\\
		arg1 & \5]>
]
}
$\to$
\avm{
[cat\textbar head & \1\\
val & \2\\
cont & \3
]
}
\z


\ea\label{avm:fsb-redup-a}
\avm{
[\type*{redup-a-lr $\subset$ redup-type}\\
cat\textbar head & adjective\\
val & [spr <>]\\
c-cont & <[pred & redup\_up\_x\_rel]>
]
}\\
\textsc{orthography}: A $\to$ AA; (irregular AB $\to$ AABB)
\z

\ea\label{avm:fsb-redup-v}
\avm{
[\type*{redup-v-lr $\subset$ redup-type}\\
cat\textbar head & verb\\
cont\textbar hook & [aspect & non-aspect]\\
c-cont & <[pred & redup\_down\_x\_rel]>
]
}\\
\textsc{orthography}: A $\to$ AA; A $\to$ A-\textit{yi}-A; (irregular AB $\to$ ABAB)
\z

This approach provides a unified account for adjectival and verbal reduplication.
Their commonalities are captured by inheritance hierarchies of the intensifier predicates and the lexical rules.
In the case of verbal reduplication, A-\textit{yi}-A is analyzed as an alternative orthographical form of AA.
This correctly captured the intuition that AA and A-\textit{yi}-A express the same meaning and only differ from each other phonologically/orthographically (see \sectref{sec:core-sem}).

Nevertheless, this analysis has some shortcomings.
To begin with, since the combination with aspect markers is completely forbidden, it is impossible for this approach to account for A\hyp{}\textit{le}\hyp{}A.
Moreover, as verbal reduplication is  considered to express a delimitative aspectual meaning,
it seems unconvincing to assume that there is no aspect information in its semantics.
We consider a semantic explanation as described in  \sectref{sec:aspM} to be more reasonable for ruling out aspect markers other than \textit{le}.
Furthermore, this account can only deal with monosyllabic reduplication and handles ABAB and AAB as irregular forms, for the reason that ABAB and AAB reduplication of AB verbs ``are not very productive in Chinese'' \citep[102]{FanSongBond2015}.
This is not true. 
\citet[33]{Xing2000stat}, \citet[161]{BascianoMelloni2017}, \citet[329]{MelloniBasciano2018} and  \citet[Sec.\,3.1]{Xie2020}  all consider both AA and ABAB to be productive,
and \citet[36]{Xing2000stat} concludes that AAB is productive as well.
Thus, these forms should not be handled as  irregular forms, but should be derivable by lexical rules.


 
 


The shortcomings of previous analyses lead us to propose a new analysis on verbal reduplication with {HPSG}, 
that formalizes the phonology of the reduplication, resolves the problem of \textit{yi} and preserves the generalization on aspect marking, 
as we will elaborate in \sectref{sec:analysis}. 


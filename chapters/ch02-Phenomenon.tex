\section{The phenomenon}\label{sec:phen}

This section introduces the fundamental grammatical behaviors of verbal reduplication in Mandarin Chinese.
 After illustrating its forms, syntactic distribution and semantics, 
 we discuss the questions of whether it is better analyzed as a morphological or a syntactic phenomenon.


%%%%%%%%%%%%%
\subsection{Forms}\label{sec:forms}

There is no general agreement on the forms of verbal reduplication in Mandarin Chinese.
We adopt a broad definition in terms of the forms of verbal reduplication in Mandarin Chinese 
and list in (\ref{ex:forms-mono})--(\ref{ex:forms-vo}) all the forms commonly discussed in the literature.

\settowidth\jamwidth{AB-\textit{le}-AB}

\ea\label{ex:forms-mono} for monosyllabic verbs: \obj{shuo1} `say'
	\ea \gll \objex{shuo1-shuo}\\ 
		say-say\\ \jambox{AA}
		\ex \gll \objex{shuo1-yi-shuo1}\\
		say-one-say\\ \jambox{A-\textit{yi}-A}
		\ex \gll \objex{shuo1-le-shuo1}\\
		say-\textsc{pfv}-say\\ \jambox{A-\textit{le}-A}
		\ex \gll \objex{shuo1-le-yi-shuo1}\\
		say-\textsc{pfv}-one-say\\ \jambox{A-\textit{le}-\textit{yi}-A}
		\ex \gll \objex{shuo1-shuo1-kan4}\\
		say-say-look\\ \jambox{AA-\obj{kan4}}
		\ex \gll \objex{shuo1-kan4-kan4}\\
		 say-look-look\\ \jambox{A-\obj{kan4}-\obj{kan4}\footnotemark}
		\z
\footnotetext{One reviewer points out that this form is not acceptable for him/her.
The acceptability of the A-\obj{kan4-kan4} form seems to vary among Mandarin Chinese speakers.
\citet[73]{Cheng2012} suggests that unlike AA-\obj{kan4}, A-\obj{kan4-kan4} is an emerging construction that is still undergoing grammaticalization.
He found in the Academia Sinica Balanced Corpus of Modern Chinese (Sinica Corpus; \citealt{sinica}) 23 tokens of A-\obj{kan4-kan4} but 141 tokens of AA-\obj{kan4} (p. 64).
This shows that the former is less common than the latter.
(\ref{ex:Akankan}) is an example of A-\obj{kan4-kan4} from the Sinica Corpus.
\ea\label{ex:Akankan} %就順著北風的吹勢,試看看能否將海沙攔下來吧!
\gll \objex{jiu4} \objex{shun4-zhe} \objex{bei3-feng1-de} \objex{chui1-shi4}, \obj{shi4-kan4-kan4} \objex{neng2} \objex{fou3} \objex{jiang1} \objex{hai3-sha1} \objex{lan2-xia4lai2} \objex{ba}!\\
just follow-\textsc{prog} north-wind-\textsc{de} blow-direction try-look-look can not make sea-sand block-down \textsc{ptc}\\\jambox{(Sinica Corpus)}
\glt `Let's just follow the north wind, try and see if we can block the sand!'
\z 

\citeauthor{Cheng2012}'s (\citeyear{Cheng2012}) study is based on the Mandarin spoken in Taiwan.
We might assume that A-\obj{kan4-kan4} is more widely used in Taiwan than in Mainland China.
But we also found the following examples of A-\obj{kan4-kan4} in the BCC corpus, which includes mostly data from Mainland China.
This suggests that A-\obj{kan4-kan4} is acceptable for and used by at least some Mainland Chinese Mandarin speakers as well.
\ea %那么,你倒说看看 ,你会有什么样的改变\ldots \jambox{(BCC)}
\gll \objex{na4me}, \objex{ni3} \objex{dao4} \obj{shuo1-kan4-kan4}, \objex{ni3} \objex{hui4} \objex{you3} \objex{shen2meyang4de} \objex{gai3bian4}\ldots\\
then you just say-look-look you will have what.kind.of change\\\jambox{(BCC)}
\glt `Then you just try and say what kind of changes you will have.'
%\z

\ex %我得听看看你的条件再说\ldots \jambox{(BCC)}
\gll \objex{wo3} \objex{dei3} \obj{ting1-kan4-kan4} \objex{ni3de} \objex{tiao2jian4} \objex{zai4} \objex{shuo1}\ldots\\
I have.to listen-look-look your condition then say\\\jambox{(BCC)}
\glt `I'll have to listen to your conditions first before talking about it/ deciding.'
\z

A statistical comparison of the acceptability and the productivity of A-\obj{kan4-kan4} among speakers of different varieties of Mandarin Chinese has to be left for future studies.
}

\ex\label{ex:forms-di} for disyllabic verbs: \obj{lai2-wang3} come-go `come and go/communicate'
		\ea \gll \objex{lai2-wang3-lai2-wang3}\\
		come-go-come-go\\ \jambox{ABAB}
		\ex \gll \objex{lai2-wang3-le-lai2-wang3}\\
		come-go-\textsc{pfv}-come-go\\ \jambox{AB-\textit{le}-AB}
		\ex \gll \objex{lai2-lai2-wang3-wang3}\\
		come-come-go-go\\ \jambox{AABB}
		\z

\ex\label{ex:forms-vo} for V-O compounds: \obj{shuo1-huang3} tell-lie `lie'
		\ea \gll \objex{shuo1-shuo1-huang3}\\
		tell-tell-lie\\ \jambox{AAB}
		\ex \gll \objex{shuo1-yi-shuo1-huang3}\\
		tell-one-tell-lie\\ \jambox{A-\textit{yi}-AB}
		\ex \gll \objex{shuo1-le-shuo1-huang3}\\
		tell-\textsc{pfv}-tell-lie\\ \jambox{A-\textit{le}-AB}
		\z
\z




 \citet{Fan1964}, \citet{Arcodiaetal2014}  and \citet{Xie2020} compare the AA, ABAB and AABB forms of reduplication 
and found a number of differences between the AA, ABAB forms compared to the AABB form in terms of their semantics, productivity, syntactic distribution and origin. 
Specifically, \citet[17--18]{Arcodiaetal2014}, \citet[144]{MelloniBasciano2018} and \citet[90]{Xie2020} identify that AA and ABAB have a diminishing meaning, 
namely that the event happens for a short duration or to a small extent. 
By contrast, AABB expresses an increasing meaning, which indicates a repetition or an action in progress. 
\citet[Sec.\,3.1]{Xie2020} also finds that AA and ABAB have relatively high productivity, 
whereas the productivity of AABB is low. 
She further shows that AABB is generally correlated with the lack of a postverbal object, but the direct object remains present when a transitive verb undergoes AA or ABAB patterns of reduplication. 
\citet[277]{Fan1964} proposes that AA, ABAB originated from the verb-measure word combination from Middle Chinese, 
while AABB developed from the reiterative rhetoric from Old Chinese. 
These differences seem to suggest that there is a fundmental difference between these two groups of verbal reduplication. 
The current study will only focus on the AA, A-\textit{yi}-A, A-\textit{le}-A, A-\textit{le}-\textit{yi}-A, ABAB and AB-\textit{le}-AB forms of verbal reduplication in Mandarin Chinese.
AA-\obj{kan4}, A-\obj{kan4}-\obj{kan4}, AAB, A-\textit{yi}-AB, A-\textit{le}-AB will also be mentioned occasionally to provide further arguments.
In what follows, the term \textit{reduplication} will be used  to refer specifically to the AA, A\hyp{}\textit{yi}\hyp{}A, A-\textit{le}-A, A-\textit{le}-\textit{yi}-A, ABAB and AB-\textit{le}-AB forms, if not specified otherwise.




%%%%%%%%%%%%
\subsection{Syntactic distribution}\label{sec:syn-dis}

The reduplication has a similar syntactic distribution as an unreduplicated verb (\ref{ex:syn-vi})--(\ref{ex:syn-svc}). 
%This suggests that a reduplicated verb can be treated as its unreduplicated counterpart in syntax (see \sectref{sec:word}).


\settowidth\jamwidth{\citep[83]{Li1998}}
\begin{sloppypar}
\ea\label{ex:syn-vi} Intransitive verb:
		\ea \gll \objex{ta1} \objex{xiao4-le}.\\
		he laugh-\textsc{pfv}\\
		\glt `He laughed.'
		\ex\label{ex:xiaole} \gll \objex{ta1} \obj{xiao4}-\obj{le}-\obj{xiao4}.\\
		he laugh-\textsc{pfv}-laugh\\
		\glt `He laughed a little bit.'
		\z
        
\ex Transitive verb:
		\ea \gll \objex{ni3} \objex{wen4} \objex{ta1}.\\
		you ask him\\
		\glt `Ask him.'
		\ex \gll \objex{ni3} \obj{wen4}-\obj{(yi)}-\obj{wen4} \objex{ta1}.\\
		you ask-(one)-ask he\\
		\glt `Try to ask him.'
		\z



\ex In a \textit{ba}-construction:
		\ea \gll \objex{gou4-mai3} \objex{zhi1qian2} \objex{zhen1} \objex{gai1} \objex{ba3} \objex{qing2kuang4} \objex{mo1-qing1chu3}.\\
		purchase-buy before really should \textsc{ba} situation touch-clearly\\
		\glt `(I) should really check the situation clearly before I make the purchase.'
		\ex \gll \objex{gou4-mai3} \objex{zhi1qian2} \objex{zhen1} \objex{gai1} \objex{ba3} \objex{qing2kuang4} \obj{mo1-mo-qing1chu3}.\footnotemark\\
		purchase-buy before really should \textsc{ba} situation touch-touch-clearly\\ \jambox{({CCL})}
		\glt `(I) should really quickly check the situation clearly before I make the purchase.'
		\z
\footnotetext{As a reviewer comments, this example also involves the reduplication of V1 of a resultative compound.
    This runs counter to exiting literature on this subject which consistently maintains that resultative compounds are not amenable to reduplication.}

\ex With modal verb:
		\ea \gll \objex{you3} \objex{liang3} \objex{ben3} \objex{shu1} \objex{ni3-men} \objex{ke3yi3} \objex{kan4} \ldots\\
		there.be two \textsc{clf} book you-\textsc{pl} can read\\
		\glt `There are two books that you can read \ldots'
		\ex \gll \objex{you3} \objex{liang3} \objex{ben3} \objex{shu1} \objex{ni3-men} \objex{ke3yi3} \obj{kan4-kan} \ldots\\
		there.be two \textsc{clf} book you-\textsc{pl} can read-read\\ \jambox{({CCL})}
		\glt `There are two books that you can read a little bit \ldots'
		\z

\ex\label{ex:syn-svc} In a Serial Verb Construction (SVC):
		\ea \gll \objex{ta1} \ldots\, \objex{qing3} \objex{shi1fu} \objex{bang1mang2} \objex{kan4} \objex{na3li3} \objex{chu1-le} \objex{wen4ti1}.\\
		she {}  ask master help look where come.out-\textsc{pfv} problem\\
		\glt `She \ldots\, asked the master to help have a look at where went wrong.'
		\ex \gll \objex{ta1} \ldots\, \objex{qing3} \objex{shi1fu} \objex{bang1mang2} \obj{kan4-kan} \objex{na3li3} \objex{chu1-le} \objex{wen4ti1}.\\
		she {} ask master help look-look where come.out-\textsc{pfv} problem\\ \jambox{({CCL})}
		\glt `She \ldots\, asked the master to help have a quick look at where went wrong.'
		\z
\z

\end{sloppypar}

\citet{SuiHu2016} claim that the syntactic distribution of the deliminative reduplication is subject to the following constraints.
First, it cannot appear in a relative clause without a modal verb, providing the contrast in (\ref{ex:rel-suihu}) as an example \citep[319]{SuiHu2016}.
\ea\label{ex:rel-suihu}
\ea[]{ \gll {[[\objex{tong2xue2-men}} \objex{tao3lun4} \objex{de]} \objex{zhe4} \objex{ge4} \objex{wen4ti2]} \objex{fei1chang2} \objex{zhong4yao4}.\\
student-\textsc{pl} discuss \textsc{de} this \textsc{clf} question very important\\
\glt `The question that the students discussed is very important.'}

\ex[*]{ \gll {[[\objex{tong2xue2-men}} \obj{tao3lun4-tao3lun4} \objex{de]} \objex{zhe4} \objex{ge4} \objex{wen4ti2]} \objex{fei1chang2} \objex{zhong4yao4}.\\
student-\textsc{pl} discuss-discuss \textsc{de} this \textsc{clf} question very important\\}
%\glt `The question that the students discussed is very important.'}

\ex[]{ \gll {[[\objex{tong2xue2-men}} \objex{xu1yao4} (\objex{zai4}) \obj{tao3lun4-tao3lun4} \objex{de]} \objex{zhe4} \objex{ge4} \objex{wen4ti2]} \objex{fei1chang2} \objex{zhong4yao4}.\\
    student-\textsc{pl} need again discuss-discuss \textsc{de} this \textsc{clf} question very important\\
    \glt `The question that the students need to discuss (again) is very important.'}
\z\z
This claim can be falsified by (\ref{ex:rel-ccl}) found in CCL.
We thus consider the contrast in (\ref{ex:rel-suihu}) a matter of providing a proper context.
\ea\label{ex:rel-ccl}
\gll \objex{er2qie3} \objex{yan3-qian2} \objex{zhe4} \objex{[[ku3-liu2} \objex{ta1} \objex{zai4} \obj{tan2-tan} \objex{de]} \objex{ren2]} \ldots\\
moreover eye-in.front.of this hard-make.stay she again discuss-discuss \textsc{de} person\\\jambox{(CCL)}
\glt `Moreover, this person in front of her who is trying hard to make her stay to discuss a bit more \ldots'
\z

Secondly, \citet[319]{SuiHu2016} show that the reduplication cannot co-occur with a duration or a frequency phrase (\ref{ex:durfreq}).
\ea\label{ex:durfreq}
\ea[]{\gll \objex{zhe4} \objex{ge4} \objex{wen4ti2} \objex{da4jia1} \objex{hai2} \objex{yao4} \objex{tao3lun4} \objex{yi2} \objex{hui4er}/\objex{ji3} \objex{ci4}.\\
this \textsc{clf} question everybody still need.to discuss one while/several time\\
\glt `As for this question, everybody still needs to discuss it for a while/several times.'}

\ex[*]{\gll \objex{zhe4} \objex{ge4} \objex{wen4ti2} \objex{da4jia1} \objex{hai2} \objex{yao4} \objex{tao3lun4-tao3lun4} \objex{yi2} \objex{hui4er}/\objex{ji3} \objex{ci4}.\\
    this \textsc{clf} question everybody still need.to discuss-discuss one while/several time\\}
%    \glt `As for this question, everybody still needs to discuss it for a while/several times.'}
\z\z
As we will show in \sectref{sec:adjuncts}, this can be explained by the redundant semantics of the reduplication and duration and frequency phrases.
%with an expression that quantifies the duration or the extent of the event described by the sentence due to its semantic properties (see \sectref{sec:adjuncts} for a detailed discussion).

Thirdly, \citet[319]{SuiHu2016} note that reduplicated verbs cannot be aspect marked (\ref{ex:redup-asp}).
\ea\label{ex:redup-asp}
\ea[]{ \gll \objex{wo3} \objex{kan4-le/guo4/zhe} \objex{zhe4} \objex{ben3} \objex{shu1}\\
I read-\textsc{pfv/exp/dur} this \textsc{clf} book\\
\glt `I read/have read/am reading this book.'
}
\ex[*]{ \gll \objex{wo3} \objex{kan4-kan4-le/guo4/zhe} \objex{zhe4} \objex{ben3} \objex{shu1}\\
    I read-read-\textsc{pfv/exp/dur} this \textsc{clf} book\\
%    \glt `I read/have read/am reading this book.'
}
\z\z
However, as shown in (\ref{ex:xiaole}), \obj{le} can occur in between the reduplicated verb
 and expresses the perfective aspectual meaning, that the event denoted by the sentence is realized.
Thus, we argue that the reduplication can in fact be aspect marked,
but only by \obj{le} `\textsc{pfv}' but not the other aspect markers.
As we will argue in \sectref{sec:aspM}, this is also not a syntactic but a semantic constraint.
%, though, except with the perfective aspect marker \textit{le} (see \sectref{sec:aspM}).% for further discussions on interactions with aspect markers). 

\citet[319]{SuiHu2016} further claim that the reduplication cannot be embedded under negation. 
But (\ref{ex:redup-neg}) shows that this is not the case.
\ea \label{ex:redup-neg}
\ea \gll \objex{wei4shen2me} \objex{wo3} \objex{bu4} \objex{kan4} \objex{ta1} \objex{qi2ta1} \objex{fang1mian4} de \objex{jin4bu4} \ldots\\
why I not look he other aspect \textsc{de} progress\\
\glt `Why don't I look at his progress in other aspects?'
\ex \gll  \objex{wei4shen2me} \objex{wo3} \objex{bu4} \obj{kan4-kan} \objex{ta1} \objex{qi2ta1} \objex{fang1mian4} de \objex{jin4bu4}\ldots\\
why I not look-look he other aspect \textsc{de} progress \\ \jambox{({CCL})}
\glt `Why don't I look a little bit at his progress in other aspects?'
\z\z

Finally, \citet[322]{SuiHu2016} claim that a reduplicated verb cannot combine with a quantized object.
However, examples in (\ref{ex:quantobj}) prove otherwise.
\ea\label{ex:quantobj}
\ea\gll \objex{ta1} \ldots\, \obj{kan4-kan} \objex{san1} \objex{ge4} \objex{ren2} \objex{he2} \objex{na4} \objex{liang4} \objex{che1zi} \dots\\
he {} look-look three \textsc{clf} person and that \textsc{clf} car\\\jambox{(CCL)}
\glt `He took a look at the three people and that car \ldots'

\ex \gll \objex{wo3} \objex{xiang3} \objex{zhuo2zhong4} \obj{shuo1-shuo} \objex{liang2} \objex{ge4} \objex{hu4xiang1} \objex{lian2xi4-zhe} \objex{de} \objex{zhong4yao4} \objex{cheng2guo3}: \ldots\\
I want emphasize say-say two \textsc{clf} each.other connect-\textsc{dur} \textsc{de} important outcome\\
\glt `I want to highlight two important interconnected outcomes.'
\z\z

In sum, we consider that the reduplication has a similar syntactic distribution as an unreduplicated verb,
and the incompatibility of the reduplication with duration and frequency phrases 
as well as aspect markers other than \obj{le} can be explained semantically, 
as we will show in the next section.

%%%%%%%%%%%%%
\subsection{Semantics}\label{sec:sem}

As shown in Section~\ref{sec:intro}, the reduplication seems to be connected to certain aspectual properties.
The current study adopts the two-component aspect model proposed by \citet{XiaoMcEnery2004} based on \citet{Smith1991}.
The general term ``aspect'' is considered to encompass the following two components:
situation aspect, i.e. ``aspectual information conveyed by the inherent semantic representation of a verb or an idealized situation'' \citep[21]{XiaoMcEnery2004};
and viewpoint aspect, i.e. ``the aspectual information reflected by the temporal perspective the speaker takes in presenting a situation'' \citep[21]{XiaoMcEnery2004}.
Situation aspect can be further modeled as verb classes  at the lexical level
and situation types (the interaction of verb classes and other constituents, such as adjuncts) at the sentential level \citep[33]{XiaoMcEnery2004}.
The verb classes are determined with verbs in a neutral context (preferably in a perfective viewpoint aspect, with a simple object only when it is obligatory),
where everything that might change the aspectual value of a verb is excluded
and only the inherent features of the verb itself are considered
(see \citealt[52]{XiaoMcEnery2004} for more details).
This does not rule out the fact that the same verb may express different aspectual properties in other contexts,
but its verb class remains the same,
as the aspectual change can be attributable to other components at the sentential level.
%This section is concerned with the interaction of the reduplication with verb classes,
%while Section~\ref{sec:asp} discusses the interaction with viewpoint aspect markers.
%The combination with other constituents such as adjuncts is briefly touched upon below,
%but a detailed discussion has to be left for future research.

In this section, we will first discuss the core meaning of the reduplication as well as the meaning of its different forms (\sectref{sec:core-sem}).
We will then investigate the interaction of the reduplication with verb classes (\sectref{sec:Aktionsarten}), aspect markers (\sectref{sec:aspM}) and other sentential components (\sectref{sec:adjuncts}) .

\subsubsection{Core meaning}\label{sec:core-sem}

The reduplication has a \textit{delimitativeness} meaning (e.g.\ \citealt[204--205]{Chao1968}; \citealt[232]{LiThompson1981};  \citealt[14]{Li1996}; \citealt[70]{Dai1997};  \citealt[382--383]{Zhu1998}; \citealt[420--421]{Xing2000}; \citealt[48]{Chen2001};  \citealt[288]{Tsao2001}; \citealt[11--12]{Yang2003}; \citealt[Sec.\,4.3]{XiaoMcEnery2004}). 
To be specific, the reduplication of [$+$durative] verbs reduces the duration of the events,
and the reduplication of [$-$durative] verbs reduces the iteration frequency of the events \citetext{\citealp[14]{Li1996}; \citealp[149--150]{XiaoMcEnery2004}}.
Besides delimitativeness, \citet[204]{Chao1968}, \citet[276]{Fan1964}, \citeauthor{Smith1991} (\citeyear[356]{Smith1991}; \citeyear[199--120]{Smith1994}), \citet[14]{Li1996} and \citet[290--291]{Tsao2001} suggest that the reduplication signifies \textit{tentativeness}, which can be used
``to refer modestly to one's own activities, or for mild imperatives'' \citep[356]{Smith1991}, or ``trying to'' do something \citep[234]{LiThompson1981}.
\textit{Frequentativeness} or \textit{habitualness}, that the event denoted by the verb happens frequently or habitually, is mentioned by \citet[276]{Fan1964}, \citet[15]{Li1996} and \citet[1]{Qian2000} as the meaning of reduplication as well.
 \citet[276]{Fan1964} further proposes a meaning of \textit{slightness} or \textit{casualness} for reduplication, which implies that the event is unimportant or conveys a casual attitude of the speaker.
 \citet[Sec.\,3.1.3]{Zhu1998} suggests that the main function of reduplication is to \textit{increase the agency} of the action or the change denoted by the verb.


In general, all of the above cited research agree that the reduplication expresses a short duration and/or a low frequency, which fits the definition of delimitativeness.
\citet[152--154]{XiaoMcEnery2004} and \citet{Yang2003} argue that the core meaning of reduplication is delimitativeness, 
while all other meanings are merely pragmatic extensions  in specific contexts.
\citet[152--154]{XiaoMcEnery2004} points out that tentativeness and casualness are constrained by a number of contextual elements 
such as the reduplicated verb must be volitional and the subject of the sentence must be animate.
But these constraints are only necessary but not sufficient conditions for a tentative or casual meaning of reduplication.
Among all instances of verbal reduplication they found in a corpus, all of them have a delimitative reading, while only some of them convey tentativeness or casualness.
\citet{Yang2003} compares the sentence pairs with reduplicated verbs and their unreduplicated counterparts,
and shows that the reduplication itself does not add a tentative, frequentative, casualness or increased agency meaning to the sentence.
Rather, these additional meanings arise from the sentences or the contexts as a whole.
She concludes that these additional meanings are results of meaning extensions of delimitativeness in specific contexts.
We follow \citet{XiaoMcEnery2004} and \citet{Yang2003} and treat delimitativeness as the central meaning of reduplication,
and the other meanings as pragmatic extensions.

The semantics of the reduplication has the properties of transitoriness,  holisticity  and dynamicity \citetext{\citealp[70--79]{Dai1997}; \citealp[155--159]{XiaoMcEnery2004}}.
It presents the situation as a  transitory and non\hyp{}decomposable whole.
A situation expressed by a sentence with the reduplication involves changes not only in the initiation and termination of an event, but also in the transitory process itself.
Compared to (\ref{ex:le-dyn-look}), which could mean that the protagonist kept staring at the the footprint,
(\ref{ex:redup-dyn-look}) indicates that the protagonist took a brief look or several brief looks at the footprint and looked away in the end, which is a process full of changes.

\settowidth\jamwidth{\citep[158]{XiaoMcEnery2004}}

\ea
  \ea\label{ex:le-dyn-look}
    \gll \objex{Wu2} \objex{Xu4mang2} \objex{kan4-le} \objex{zuo4-an4} \objex{shi2} \objex{liu2xia4} de \objex{jiao3yin1} \ldots\\ 
    Wu Xumang look-\textsc{pfv} commit-crime when leave \textsc{de} footprint\\ \jambox{\citep[158]{XiaoMcEnery2004}}
    \glt `Wu Xumang looked at the footprint left when the crime was committed.'
  \ex\label{ex:redup-dyn-look}
    \gll  \objex{Wu2} \objex{Xu4mang2} \objex{kan4-le-kan4} \objex{zuo4-an4} \objex{shi2} \objex{liu2xia4} de \objex{jiao3yin1} \ldots\\
    Wu Xumang look-\textsc{pfv}-look commit-crime when leave \textsc{de} footprint\\ \jambox{\citep[158]{XiaoMcEnery2004}}
    \glt `Wu Xumang looked a little bit at the footprint left when the crime was committed.'
  \z
\z

The semantics of A-\textit{le}-A can be deduced compositionally from its structure. 
It is a hierarchical combination of the perfective aspect and delimitativeness, ``conveying a transitory event which has been actualized'' \citep[151]{XiaoMcEnery2004}.


As for A-\textit{yi}-A, \citet[273]{Fan1964} compares examples found in novels and plays and concludes that A-\textit{yi}-A has exactly the same meaning as its AA counterpart.
She thus assumed that AA is merely a form of A-\textit{yi}-A, where the \textit{yi} is omitted phonologically.
\citet[Sec.\,5]{Xing2000} considers that the major  difference  between AA and A-\textit{yi}-A lies in the speaker's attitude.
The former conveys a casual attitude whereas the latter sounds more serious. %\footnote{This can be a register effect. Register project sponsorship xxxxxxx}
However, he stresses that there is no difference in the delimitative semantics of both forms,
and that the variance in meaning is a pragmatic one.
The difference is also not absolute and often only shows a tendency.
\citet{Xu2002} finds out that compared to A-\textit{yi}-A, one tends to use AA in contexts with strong emotional attitudes, urgent, casual, timid or uncertain contexts.
But he also states that these differences are pragmatic rather than semantic, 
as he argues that AA and A-\textit{yi}-A can be used interchangeably in most cases,
and the differences in meaning only arise from specific contexts as a whole.
\citet[15]{Yang2003} suggests that AA and A-\textit{yi}-A have the same core meaning, while A-\textit{yi}-A implies a slightly more serious attitude than AA due to its length.
We assume A-\textit{yi}-A to be a form of reduplication and that it has the same core semantics as AA.

AA-\obj{kan4} and A-\obj{kan4}-\obj{kan4} are described to express a ``try \ldots\, and find out'' meaning \citep[63]{Cheng2012}.
\citet[290]{Tsao2001} also observes that the tentative meaning is particularly prominent when the reduplication is followed by \obj{kan4} `look'.
We still consider the tentativeness implied by these two forms to be a pragmatic extension of delimitativeness.
The tentative meaning is made prominent by the verb \obj{kan4} `look',
and the whole structure can be understood as ``do A a little bit and see''.




%%%%%%%%
\subsubsection{Interaction with verb classes}\label{sec:Aktionsarten}

Previous research often claims that the reduplication can only be used for certain verb classes, while it is infelicitous for other ones.
 \citet[234--235]{LiThompson1981} and \citet[277--278]{Hong1999} suggest that reduplication is only possible for volitional activity verbs.
\citet[70--71]{Dai1997} and \citet[290]{Tsao2001} both consider that reduplication can only be used in dynamic situations.
The former further claims that achievement verbs cannot be reduplicated.
 \citet[155]{XiaoMcEnery2004}, \citet[20]{Arcodiaetal2014} and \citet[145]{BascianoMelloni2017} propose that only [$+$dynamic] and [$-$result] verbs can be reduplicated.
This means that the reduplication can only interact with dynamic situations which encode no results and is consequently only compatible with activities and semelfactives, but not with states and achievements.

\citet[53]{Chen2001} and \citet[10--11]{Yang2003} acknowledge that the reduplication of non\hyp{}vo\-li\-tion\-al verbs is more restricted than that of volitional ones.
But \citet[381--382]{Zhu1998} lists a number of non\hyp{}volitional predicates that can be reduplicated.
We found the examples shown in (\ref{ex:nonvol}) in {CCL} where non\nobreakdash-vo\-li\-tion\-al verbs \obj{wei3qu1} `feel wronged', \obj{ren4-xing4} `be willful' and \obj{diao4} `drop' are reduplicated.

\settowidth\jamwidth{(CCL)}

\begin{sloppypar}
\ea\label{ex:nonvol}
\ea
\gll \objex{ke3shi4} \objex{xian4mu4jin1}, \objex{da4jia1} \objex{ye3} \objex{zhi3hao3} \obj{wei3qu1-wei3qu1} le.\\
but now  everybody also can.only feel.wronged-feel.wronged \textsc{ptc}\\ \jambox{({CCL})}
\glt `But now, everybody can only feel wronged a little bit.'

\ex
\gll \objex{ta1-men} \objex{neng2} \objex{zuo4} de \objex{bu2guo4} \objex{shi4} \obj{ren4-ren4-xing4}, \objex{shua3} \objex{dian3'er} \objex{xiao3} \objex{pi2qi4}, \obj{diao4-diao} \objex{yan3lei4} \objex{shen2mede}.\\
she-\textsc{pl} can do \textsc{de} just be be.willful-be.willful-temperament play a.little small temper drop-drop tear what \\ \jambox{({CCL})}
\glt `What they can do is just to be a little bit willful, to lose their temper a little bit and to drop a little bit of tears or something.'
\z
\z
\end{sloppypar}

It is true that  the reduplication of stative and achievement verbs is not as easily acceptable as that of activities and semelfactives.
\citet[155]{XiaoMcEnery2004} classify \obj{bing4} `be sick' as a stative verb.
Indeed, compared  to the questionable reduplication of \obj{bing4} `be sick' in (\ref{ex:redup-stat}),  
 the reduplication of the activity verb \obj{kan4} `watch' in (\ref{ex:redup-actV})
and that of the semelfactive verb \obj{ke2sou4} `cough' in (\ref{ex:redup-semel}) are readily acceptable.

\settowidth\jamwidth{\citep[155]{XiaoMcEnery2004}}

\ea
\ea[?]
{\gll \objex{ta1} \obj{bing4-bing} \objex{jiu4} \objex{hao3} le.\\
he be.sick-be.sick then well \textsc{ptc}\\ \jambox{\citep[155]{XiaoMcEnery2004}}
Intended: `He was sick for a little while and then got well.'}\label{ex:redup-stat}

\ex[]
{\gll \objex{ta1} \obj{kan4}-\obj{le}-\obj{kan4} \objex{na4} \objex{chang3} \objex{bi3sai4}.\\
he watch-\textsc{pfv}-watch that \textsc{clf} competition\\
\glt `He watched that competition for a little while.' \\}\label{ex:redup-actV}

\ex[]
{\gll \objex{ta1} \obj{ke2sou4-ke2sou4} \objex{jiu4} \objex{hao3} le.\\
he cough-cough then well \textsc{ptc}\\
\glt `He coughed a little bit and then got well.'}\label{ex:redup-semel}
\z
\z

However, examples such as (\ref{ex:redup-achi-stat4}) can be found where \obj{bing4} `be sick' is reduplicated.

\ea\label{ex:redup-achi-stat4}
\gll \objex{wo3} \objex{zhen1} \objex{xiang3} \obj{bing4-yi-bing4}, \objex{xie1} \objex{ta1} \objex{ge} \objex{shi2} \objex{tian1} \objex{ban4} \objex{yue4}.\\
I really want be.sick-one-be.sick rest it \textsc{clf} ten day half month\\ \jambox{\citep[54]{Chen2001}}
\glt `I really want to be sick for a little while and rest for ten days or half a month.'
\z

As a reviewer points out, \citet[Sec.\,3.3]{Tham2013} considers  \obj{bing4} `be sick' to be a basic change of state (COS) verb, i.e. `become sick'.
In contrast, she uses \obj{xi3huan1} `like' and \obj{xiang1xin4} `believe' as examples of stative verbs.
\citet[680]{PeckEtAl2013} also list \obj{xi3huan1} `like' and \obj{xiang1xin4} `believe' as stative verbs.
For these two verbs, examples such as (\ref{ex:xihuan}) and (\ref{ex:xiangxin}) can be found.
\ea\label{ex:xihuan}%这车好型啊!!!。怎么了,我就喜欢喜欢 不行么~ (BCC)
\gll \objex{zhe4} \objex{che1} \objex{hao3} \objex{xing2} \objex{a}! \objex{zen3me} \objex{le}, \objex{wo3} \objex{jiu4} \obj{xi3huan1-xi3huan4} \objex{bu4xing2} \objex{me}?\\
this car so cool \textsc{ptc} what \textsc{ptc} I just like-like cannot \textsc{ptc}\\\jambox{(BCC)}
\glt `This car is so cool! What? I can't just like it for a little bit?'

\ex\label{ex:xiangxin}%谁让我来相信相信 (CCL)
\gll \objex{shei2} \objex{rang4} \objex{wo3} \objex{lai2} \objex{xiang1xin4-xiang1xin4}\\
who let I come believe-believe\\\jambox{(CCL)}
\glt `Who lets me believe him/her?'
\z

On the other hand, as the reviewer %\citealt[283]{Ding2010}, \citealt[fn. 3]{BascianoMelloni2017}
and indeed \citet[669--670]{Tham2013} herself note,
verbs expressing psychological states such as these can have a COS interpretation (but not necessarily),
let us look at other examples of stative verbs listed by \citet[680]{PeckEtAl2013} which do not express psychological states,
and examples in  \citet[Sec.\,3.3.3]{XiaoMcEnery2004} of individual-level states (ILSs) which only have stative interpretation,
as opposed to stage-level states (SLSs) which can have both stative and dynamic interpretations.\footnote{
    Note that most examples of ILSs in \citet[Sec.\,3.3.3]{XiaoMcEnery2004} are adjectives.
    Since in Mandarin Chinese, adjectives have a different reduplication pattern (see e.g.\ \citealt[Sec.\,2.2]{Tsao2001}, \citealt[Sec.\,4.1]{FanSongBond2015}, \citealt[Sec.\,3]{Sui2018}),
    and further, COS verbs can be systematically derived from adjectives \citep[Sec.\,3]{Tham2013},
    these predicates are not included in our examples.}
The following examples contain the reduplication of \obj{xiang4} `look like' and \obj{zai4-chang3} `be present on the scene'.
\ea\label{ex:xiang}%范丞丞雪中撑伞这是蒋丞吧王圣迪也拿着滑板!太像顾淼了王安宇你快点也像一像吧!(https://www.163.com/dy/article/FS1FKUC30534DZO1.html (2020/11/22, accessed 2024/03/26))
Context: \objex{Wang2} \objex{Sheng4di2} is the actor of the character \objex{Gu4} \objex{Miao3} in a TV series.
\objex{Wang2} Ān\objex{yu3} is another actor in the series.\\
\\
\gll \objex{Wang2} \objex{Sheng4di2} \ldots\, \objex{tai4} \objex{xiang4} \objex{Gu4} \objex{Miao3} \objex{le}. \objex{Wang2} Ān\objex{yu3} \objex{ni3} \objex{kuai4dian3} \objex{ye3} \obj{xiang4-yi-xiang4} \objex{ba}!\footnotemark\\
Wang Shengdi {} very look.like Gu Miao \textsc{ptc} Wang Anyu you fast also look.like-one-look.like \textsc{ptc}\\
\glt `Wang Shengdi \ldots\,  really looks like Gu Miao. Wang Anyu, you too should just look like (your character) already!'

\ex\label{ex:zaichang}%她此刻正沉浸在一个前所未有的悲剧中,很需要母亲在一在场便能给予她的那点安慰 (CCL)
\gll \objex{ta1} \ldots\, \objex{hen4} \objex{xu1yao4} \objex{mu3qin1} \obj{zai4-yi-zai4-chang3}~\ldots\\
she {} very need mother be.present-one-be.present-scene\\\jambox{(CCL)}
\glt `She \ldots\, really needs her mother to be present on scene for a little while \ldots'
\z
\footnotetext{\url{https://www.163.com/dy/article/FS1FKUC30534DZO1.html} (Accessed 26th  March 2024).}

One might argue that even in the examples above, dynamic rather than stative meaning is conveyed.
We argue that the dynamic interpretation does not come from the verb but from reduplication.
The use of reduplication affects the situation aspect at the sentential level.
And as we describe in \sectref{sec:core-sem}, the semantics of reduplication has the property dynamicity.
Verbs such as \obj{xiang4} `look like' and \obj{zai4-chang3} `be present on the scene' are stative in a neutral context and thus, we consider the intrinsic feature of these verbs to be stative
and they should be classified as stative verbs.
The dynamic interpretation only arises when they are used in specific contexts, in this case, when they are reduplicated.

Similar to stative verbs, the reduplication of achievement verbs is also not readily acceptable,
as shown in (\ref{ex:redup-achiV}) with reduplication of \obj{ying2} `win'.

\ea[?]
{\gll \objex{ta1} \obj{ying2}-\obj{le}-\obj{ying2} \objex{na2} \objex{chang3} \objex{bi3sai4}.\\
    he win-\textsc{pfv}-win that \textsc{clf} competition\\ \jambox{\citep[155]{XiaoMcEnery2004}}
    \glt Intended: `He won that competition a little bit.'}\label{ex:redup-achiV}
\z

However, examples such as those in (\ref{ex:redup-achi-stat1}--\subref{ex:redup-achi-stat5}) can be found.
Here, achievement verbs like \obj{wang4} `forget' and \obj{sheng1} `give birth to' are reduplicated.

\settowidth\jamwidth{\citep[112]{Chen2005}}


%\footnotetext[3]{Tian, Han. 1959. \textit{Tianhan xuanji [Selected works of Tianhan]}, 122. Beijing: People's Literature Publishing House.}
%\footnotetext[4]{Zhou, Libo. 1958. \textit{Shang xiang ju bian [Big changes of mountains and the countryside]}, 204. Beijing: The Writers Publishing House.}
%\begin{sloppypar}
\ea
\ea\label{ex:redup-achi-stat1}
\gll \objex{deng3} \objex{ren2-men} ba \objex{zhe4} \objex{jian4} \objex{shi4} \obj{wang4}-\obj{wang} \objex{zai4} \objex{shuo1} ba.\footnotemark\\
wait people-\textsc{pl} \textsc{ba} this \textsc{clf} incident forget-forget then talk \textsc{ptc}\\
\glt `Let's wait until people forget this incident a little bit and then talk about it.'\\
\footnotetext{\objex{Liu2}, \objex{Zhen1}. 1963. \obj{Chang2} \obj{chang2} \obj{de} \obj{liu2shui3} \textit{[Long long water]}, 72. Beijing: The Writers Publishing House.}
%\ex\label{ex:redup-achi-stat2}
%\gll huitou mo ge zao \textit{shutan}-\textit{shutan} ba.\footnotemark\\
%later wipe \textsc{clf} bath be.comfortable-be.comfortable \textsc{ptc}\\
%\glt `Let's take a bath later and be comfortable for a little while.'\\

%\ex\label{ex:redup-achi-stat3}
%\gll \textit{lian}-\textit{lian}-\textit{ai} shi keyi de, ban xishi dinghao chi yidian.\footnotemark\\
%like-like-love \textsc{cop} ok \textsc{de} host wedding best late a.bit\\
%\glt `It's ok to be in love a little bit, but it's better to get married a bit later.'\\

\ex\label{ex:sheng}%我们女人是这么神经质又没用,只能生生孩子,做做女人的活动.。(BCC)
\gll \objex{wo2men} \objex{nv3ren2} \ldots\, \objex{zhi3} \objex{neng2} \obj{sheng1-sheng1} \objex{hai2zi}\ldots \\
we women {} only can give.birth.to-give.birth.to child\\\jambox{(BCC)}
\glt `We women can only give birth to children \ldots'

\ex\label{ex:redup-achi-stat5}
\gll \objex{jiao4} \objex{ta1} \obj{sheng1-sheng1} \objex{xiao3hai2}, \objex{jiu4} \objex{zhi1dao4} \objex{zuo4} \objex{mu3qin1} de \objex{gan1-ku3} le.\\
let she give.birth.to-give.birth.to child then know \textsc{cop} mother \textsc{de} sweet-bitter \textsc{ptc}\\ \jambox{\citep[112]{Chen2005}}
\glt `Let her try to give birth to a child and then she will know the bittersweetness of being a mother.'
\z
\z
%\end{sloppypar}

The reduplication does not seem to cancel the \textit{telos} of achievements.
Applying the classic ``for/in X-time'' test, reduplication is compatible with ``in X-time'' (\ref{ex:telob}) but not with ``for X-time'' (\ref{ex:teloc}).
\ea
\ea[]{\label{ex:teloa}% 一块... 到附近市场买买东西 (CCL)
    \gll \objex{yi2kuai4} \ldots\, \objex{dao4} \objex{fu4jin4} \objex{shi4chang3} \obj{mai3-mai3} \objex{dong1xi}\ldots\\
    together {} arrive nearby market buy-buy thing\\\jambox{(CCL)}
    \glt `(We) together \ldots\, went to the market nearby to buy some things.'}

\ex[]{\label{ex:telob}% 两天后到附近市场买买东西
    \gll \objex{liang3} \objex{tian1} \objex{hou4} \objex{dao4} \objex{fu4jin4} \objex{shi4chang3} \obj{mai3-mai3} \objex{dong1xi}.\\
    two day later arrive nearby market buy-buy thing\\
    \glt `Two days later, (we) went to the market nearby to buy some things.}

\ex[*]{\label{ex:teloc}%到附近市场买买东西两天
    \gll \objex{dao4} \objex{fu4jin4} \objex{shi4chang3} \obj{mai3-mai3} \objex{dong1xi} \objex{liang3} \objex{tian1}.\\
    arrive nearby market buy-buy thing two day\\
    \glt Intended: `(We) went to the market nearby to buy some things for two days.'}
\z\z
The following continuation that cancels the \textit{telos} is also infelicitous (\ref{ex:telod}).
\ea\label{ex:telod} %到附近市场买买东西 \# 但什么也没买到
\gll \objex{dao4} \objex{fu4jin4} \objex{shi4chang3} \obj{mai3-mai3} \objex{dong1xi}, \# \objex{dan4} \objex{shen2me} \objex{ye3} \objex{mei2} \objex{mai3-dao4}.\\
arrive nearby market buy-buy thing {} but anything also not buy-arrive\\
\glt Intended: `(We) went to the market nearby to buy some things, but we didn't buy anything.'
\z
These tests suggest that the reduplicated verbs are indeed achievements,
rather than being used as activities.

The examples presented in this section show that although the reduplication does have a tendency to interact with volitional verbs and with activities and semelfactives due to its dynamic meaning, 
this is by no means a rigid constraint, 
and non\hyp{}volitional verbs, states and achievements can be reduplicated in appropriate contexts as well,
 contrary to common beliefs in the literature.
Thus, a theoretical account of reduplication should not restrict its use to only certain verb classes.




%%%%%%%%
\subsubsection{Interaction with aspect markers}\label{sec:aspM}

As mentioned in \sectref{sec:syn-dis}, the reduplication can only be marked by the perfective aspect marker \textit{le} but not other aspect markers.\footnote{
There is no consensus on which elements exactly are considered aspect markers in Mandarin Chinese. 
We only discuss the most commonly recognized ones here.
}
We believe this incompatibility to be for semantic reasons.

\citet[Ch. 4]{XiaoMcEnery2004} consider \obj{le}, \obj{guo4} and reduplication to indicate perfective aspects, as they all view the situation as an inseparable whole.
The perfective aspect marker \obj{le} is compatible with reduplication while the experiential aspect marker \obj{guo4} is not.
\citet[128--131]{XiaoMcEnery2004} state that \obj{le} has the semantic feature of dynamicity, since it ``can focus on both heterogeneous internal structures and changing points'' \citep[129]{XiaoMcEnery2004}.
It can be combined with a situation with a dynamic internal structure, such as crying in (\ref{ex:cry}).
It can also co\hyp{}occur with a situation with a change at a certain time point, such as getting to know in (\ref{ex:know}).

\settowidth\jamwidth{\citep[129]{XiaoMcEnery2004}}

\ea
  \ea\label{ex:cry}
    \gll \objex{wei4ci3}, \objex{Deng4} \objex{Li4jun1} \objex{shang1xin1} \objex{de} \objex{ku1-le} \objex{san1} \objex{tian1}.\\
    for.this Deng Lijun sadly \textsc{de} cry-\textsc{pfv} three day\\ \jambox{\citep[129]{XiaoMcEnery2004}}
    \glt `For this reason, Deng Lijun cried sadly for three days.'
  \ex\label{ex:know}
    \gll \objex{ta1} \objex{zhi1dao4-le} \objex{zhe4} \objex{shi4} de \objex{nei4-qing2}.\\
    he know-\textsc{pfv} this matter \textsc{de} inside-information\\ \jambox{\citep[130]{XiaoMcEnery2004}}
    \glt `He got to know the inside information on this matter.'
  \z
\z
\obj{Le} is compatible with the reduplication, because its dynamicity can relate to not only the termination or instantiation of an event (a point of change), but also the process of the situation, just like that of the reduplication (see \sectref{sec:core-sem}).

In comparison, the experiential aspect marker \obj{guo4} cannot co\hyp{}occur with a reduplicated verb, 
because its dynamicity attributes to an ``experiential change'' \citep[148]{XiaoMcEnery2004}, 
namely that a situation has been experienced historically and that ``the final state of the situation no longer obtains'' at the reference time \citep[144]{XiaoMcEnery2004}. 
Compare (\ref{ex:guo-dyn}) and (\ref{ex:le-dyn}), \obj{guo4} in (\ref{ex:guo-dyn}) suggests a change out of the state of being a soldier, 
whereas \obj{le} in (\ref{ex:le-dyn}) conveys a change into the state of being a soldier \citep[149]{XiaoMcEnery2004}.

\ea
  \ea\label{ex:guo-dyn}
    \gll \objex{ta1} \objex{dang1-guo4} \objex{bing1}.\\
    he serve.as-\textsc{exp} soldier\\ \jambox{\citep[149]{XiaoMcEnery2004}}
    \glt `He once served as a soldier.'
  \ex\label{ex:le-dyn}
    \gll \objex{ta1} \objex{dang1-le} \objex{bing1}.\\
    he serve.as-\textsc{pfv} soldier\\ \jambox{\citep[149]{XiaoMcEnery2004}}
    \glt `He became a soldier.'
  \z
\z
It is clear that \obj{guo4} only indicates a change at the termination of a situation and cannot express the dynamicity within a situation.
Hence, it is incompatible with the semantics of the reduplication.

Due to the holistic semantics of the reduplication, it is incompatible with imperfective aspect markers: the durative aspect marker \obj{zhe} and the progressive aspect marker \obj{zai4}, as both only focus on  a part of the situation and do not view the situation as a whole \citep[Ch. 5]{XiaoMcEnery2004}.

From the illustration above, it seems that due to its semantics, reduplication can only be marked by \obj{le} but not the other aspect markers.




%%%%%%%%%
\subsubsection{Interaction with other sentential components}\label{sec:adjuncts}

The reduplication is incompatible with an expression that quantifies the duration or the extent of the event expressed in the sentence (\ref{ex:syn-quan}) \citetext{\citealp[83--84]{Li1998}; \citealp[114--115]{Chen2005}}.
This is  because the reduplication already contains a quantity meaning \citetext{\citealp[84]{Li1998}; \citealp[114--115]{Chen2005}}, namely a short duration or a small extent,
which cannot be measured on a concrete scale (\citealt[155]{XiaoMcEnery2004}; \citealt[333]{SuiHu2016}).
This results from the properties of reduplication rather than the verb itself,
as the verb itself can be combined with such an expression (\ref{ex:syn-quana}).

\ea %With an expression of quantity: 
\label{ex:syn-quan}
\ea[]{\label{ex:syn-quana}\gll \objex{ta1} \objex{yi4} \objex{tian1} \objex{pao3} \objex{shi2} \objex{li3}.\\
    he one day run ten mile\\ \jambox{\citep[83]{Li1998}}
    \glt `He runs ten miles a day.'}
\ex[*]{\gll \objex{ta1} \objex{yi4} \objex{tian1} \obj{pao3}-\obj{pao3} \objex{shi2} \objex{li3}.\\
    he one day run-run ten mile\\}
\z
\z

A reviewer also notes that the reduplication appears frequently in imperative (example \ref{ex:redup-achi-stat1}) and conditional sentences (\ref{ex:cond}) as well as causative sentences with \obj{rang4}\slash\obj{jiao4}\slash\obj{shi3} `let/let/make' (examples \ref{ex:xiangxin}, \ref{ex:xiang}, \ref{ex:redup-achi-stat5}).
\ea\label{ex:cond}
\gll \objex{ni3} \objex{yao4shi4} \objex{gei3} \objex{wo3} \obj{zhi3-zhi} \objex{Li3-jia1} \objex{de} \objex{men2}, \objex{wo3} \objex{yuan4} \objex{gei3} \objex{ni3} \objex{gui4xia4}\\
you if give I point-point Li-family \textsc{de} door I be.willing.to give you kneel.down\\\jambox{\citep[157]{XiaoMcEnery2004}}
\glt `If you point at the door of Li's for me, I am willing to kneel down for you.'
\z
This can  be attributed to the holistic semantics of reduplication (see \sectref{sec:core-sem}).
%Due to its unique features compared to the perfective aspect marker \obj{le} and the experiential aspect marker \obj{guo4},
As illustrated in \sectref{sec:aspM}, \obj{le} expresses a situation being realized,
while \obj{guo4} conveys a situation already experienced,
and thus, both rarely occur in future situations.
The use of reduplication does not have these constraints.
This makes the reduplication the only perfective viewpoint aspect that can freely occur in future situations \citep[156--157]{XiaoMcEnery2004}.
This explains the general tendency for the reduplication to appear in conditional clauses and imperative or causative contexts,
which are frequently used to describe future situations.




%%%%%%%%%%%%
\subsection{Word vs.\ phrase}\label{sec:word}

The literature on reduplication makes different assumptions on whether it is a morphological or syntactic phenomenon.
\citet[Ch. 4]{Chao1968}, \citet[Ch. 3]{LiThompson1981}, \citet[Sec.\,4.1]{Dai1992} and \citet[4--5]{Liao2014} list reduplication under morphological processes. 
By contrast, \citet[23]{Arcodiaetal2014}, \citet{Xiong2016}, \citet[146]{BascianoMelloni2017}, \citet[229--231]{YangWei2017}, \citet[330]{MelloniBasciano2018} and \citet{Xie2020}  claim it to be syntactic.
%And e.g. \citet{Chen2001}, \citet{Chen2005}  and \citet{Yang2003} did not make any claim about its grammatical status.
This section reviews the arguments in \citet{Xie2020},
and applies tests from \citeauthor{Dai1992} (\citeyear[Sec.\,7]{Dai1992}, \citeyear[Sec.\,2.3--2.4]{Dai1998}) to distinguish words from phrases in Mandarin Chinese.
 %applies the tests proposed by \citet{Duanmu1998} and \citet{Schaefer2009} to distinguish words from phrases in Mandarin Chinese 
%and compares the behaviors of the reduplication with those of Parallel Verb Compounds 
%(compounds which consist of two verbs that ``either are synonymous or signal the same type of predicative notions'' \citep[68]{LiThompson1981}, e.g. \textit{gou-mai} `purchase-buy, buy') and {SVC}s. 
The results are compatible with a lexical analysis.



\citet{Xie2020} compares the AA and the ABAB forms of reduplication with the AABB form and claims that AA and ABAB are syntactic processes while AABB is morphological.
She points out that AA and ABAB behave differently from AABB in their productivity, possibility of \obj{le} insertion, categorial stability, transitivity, and input/output constraints.
While AA and ABAB are highly productive, AABB shows low productivity. 
\obj{Le} can be inserted freely into AA (\ref{ex:le-insertionAA}) and ABAB (\ref{ex:le-insertionABAB}) but not into AABB (\ref{ex:le-insertionAABB}).

\settowidth\jamwidth{\citep[85]{Xie2020}}

\ea\label{ex:le-insertionAA}
  \ea \gll \objex{Yao2} \objex{Ming2} \obj{kan4}-\obj{kan4} \objex{ta1} de \objex{fan1yi4} \objex{Ke1} \objex{Lin2} \ldots\\
  Yao Ming look-look he \textsc{de} translator Ke Lin\\
  \glt `Yao Ming looked at his translator Ke Lin a little bit \ldots'
  
  \ex \gll \objex{Yao2} \objex{Ming2} \obj{kan4}-\obj{le}-\obj{kan4} \objex{ta1} de \objex{fan1yi4} \objex{Ke1} \objex{Lin2} \ldots\\
  Yao Ming look-\textsc{pfv}-look he \textsc{de} translator Ke Lin\\ \jambox{({CCL})}
  \glt `Yao Ming looked at his translator Ke Lin a little bit \ldots'
  \z
\z

\ea\label{ex:le-insertionABAB}
  \ea \gll \objex{ta1} \obj{he2ji}-\obj{he2ji}, \objex{dui4} \objex{Jiang1} \objex{Qing1} \objex{shuo1} \ldots\\
  he consider-consider to Jiang Qing say\\
 \glt `He considered a little bit, and told Jiang Qing \ldots'
 
  \ex \gll \objex{ta1} \obj{he2ji}-\obj{le}-\obj{he2ji}, \objex{dui4} \objex{Jiang1} \objex{Qing1} \objex{shuo1} \ldots\\
  he consider-\textsc{pfv}-consider to Jiang Qing say\\ \jambox{({CCL})}
  \glt `He considered a little bit, and told Jiang Qing \ldots'
  \z
\z

\ea\label{ex:le-insertionAABB}
  \ea[]{\gll \obj{yao2}-\obj{yao2}-\obj{huang4}-\obj{huang4} \objex{jiu4} \objex{ba2} \objex{chu1lai2} le.\\
  shake-shake-sway-sway then pull out \textsc{ptc}\\  \jambox{\citep[85]{Xie2020}}
  \glt `Shake it a little bit and then it will be pulled out.'}
  
  \ex[*]{\gll \obj{yao2}-\obj{yao2}-\obj{le}-\obj{huang4}-\obj{huang4} \objex{jiu4} \objex{ba2} \objex{chu1lai2} le.\\
  shake-shake-\textsc{pfv}-sway-sway then pull out \textsc{ptc}\\ \jambox{\citep[85]{Xie2020}}}
  \z
\z

The output of AA and ABAB does not change the grammatical category of the input (verb), but the output of AABB could have other categories such as adverb (\ref{ex:AABBadv}) or adjective (\ref{ex:AABBadj}).

\ea\label{ex:AABBadv}
  \ea[*]{\gll \objex{dian4-che1} \obj{yao2-huang4}-\obj{yao2-huang4} \objex{kai1} \objex{zou3} \ldots\\
  electric-car shake-sway-shake-sway drive away\\ \jambox{\citep[86]{Xie2020}}}
  
  \ex[]{\gll \objex{dian4-che1} \obj{yao2}-\obj{yao2}-\obj{huang4}-\obj{huang4} \objex{kai1} \objex{zou3} \ldots\\
  electric-car shake-shake-sway-sway drive away\\ \jambox{\citep[86]{Xie2020}}
  \glt `The tram drove away jiggly \ldots'}
  \z
\z

\ea\label{ex:AABBadj}
  \ea[*]{\gll \ldots\,  \objex{zuo4} \objex{zai4}  \obj{yao2-huang4}-\obj{yao2-huang4} de \objex{che1} \objex{shang4}\\
  {} sit on shake-sway-shake-sway \textsc{de} car on\\ \jambox{\citep[86]{Xie2020}}}
  
  \ex[]{\gll \ldots\, \objex{zuo4} \objex{zai4}   \obj{yao2}-\obj{yao2}-\obj{huang4}-\obj{huang} de \objex{che1} \objex{shang4}\\
  {} sit on shake-shake-sway-sway \textsc{de} car on\\ \jambox{\citep[86]{Xie2020}}
  \glt `\ldots\, sit on the jiggling car'}
  \z
\z
 
AA and ABAB do not change the valency of the input verb, but AABB makes a transitive verb intransitive (\ref{ex:trans}). 

\ea\label{ex:trans}
  \ea[]{\gll \objex{qiao1-da3} \objex{gan1-jing1} \objex{shi4} \objex{huan3jie3} \objex{gan1-qi4} de \objex{hao3} \objex{ban4fa3}.\\
  knock-beat liver-channel \textsc{cop} relieve liver-\textit{qi} \textsc{de} good method\\ \jambox{\citep[88]{Xie2020}}
  \glt `Beating the liver channel is a good method to relieve the stagnation of liver \textit{qi}.'}
  
  \ex[]{\gll \obj{qiao1-da3}-\obj{qiao1-da3} \objex{gan1-jing1} \objex{shi4} \objex{huan3jie3} \objex{gan1-qi4} de \objex{hao3} \objex{ban4fa3}.\\
  knock-beat-knock-beat liver-channel \textsc{cop} relieve liver-\textit{qi} \textsc{de} good method\\ \jambox{\citep[88]{Xie2020}}
   \glt `Beating the liver channel a little bit is a good method to relieve the stagnation of liver \textit{qi}.'}
   
   \ex[*]{\gll \obj{qiao1}-\obj{qiao1}-\obj{da3}-\obj{da2} \objex{gan1-jing1} \objex{shi4} \objex{huan3jie3} \objex{gan1-qi4} de \objex{hao3} \objex{ban4fa3}\\
  knock-knock-beat-beat liver-channel \textsc{cop} relieve liver-\textit{qi} \textsc{de} good method\\ \jambox{\citep[88]{Xie2020}}}
  
  \ex[]{\gll \obj{qiao1}-\obj{qiao1}-\obj{da3}-\obj{da3} \objex{shi4} \objex{huan3jie3} \objex{gan1-qi4} de \objex{hao3} \objex{ban4fa3}\\
   knock-knock-beat-beat \textsc{cop} relieve liver-\textit{qi} \textsc{de} good method\\
  \glt `Knocking around is a good method to relieve the stagnation of liver \textit{qi}.'}
  \z
\z

The two groups also have different input and output constraints. \citet{Xie2020} claims that only dynamic and volitional verbs can undergo AA or ABAB reduplication (but see \sectref{sec:Aktionsarten}).
In comparison, AABB requires its input to be a complex verb whose constituents are either synonymous, antonymous or logically coordinated (\ref{ex:AABB-input}). 
Moreover, as can be seen in the translation in (\ref{ex:AABB-input}), the output of AABB has an increasing meaning, i.e. an event happens repeatedly or continuously, as opposed to the delimitative meaning of AA and ABAB.

\ea\label{ex:AABB-input}
  \ea \gll \objex{duo3-shan3} $\rightarrow$ \obj{duo3}-\obj{duo3}-\obj{shan3}-\obj{shan2}\\
  hide-dodge {} hide-hide-dodge-dodge\\ \jambox{\citep[88]{Xie2020}}
  \glt `hide and dodge' `hide and dodge repeatedly'
  
  \ex \gll \objex{jin4-chu1} $\rightarrow$ \obj{jin4}-\obj{jin4}-\obj{chu1}-\obj{chu1}\\
  enter-exit {} enter-enter-exit-exit\\ \jambox{\citep[88]{Xie2020}}
  \glt `enter and exit' `enter and exit repeatedly'
  
  \ex \gll \objex{shuo1-xiao4} $\rightarrow$ \obj{shuo1}-\obj{shuo1}-\obj{xiao4}-\obj{xiao4}\\
  talk-laugh {} talk-talk-laugh-laugh\\ \jambox{\citep[88]{Xie2020}}
  \glt `talk and laugh' `talk and laugh continuously'
  \z
\z

However, a morphological process can be productive, and it does not necessarily change the category or valency of the input.
For instance, the \obj{-able} derivation in English is a productive morphological process.
Tense inflections in English such as \obj{-ed} change neither the category nor the valency of the input verb.
%the \textit{-bar} derivation in German is a productive morphological process \citep[330]{Mueller2002}. 
%The \textit{be-} prefixation in German does not change the category of the input: \textit{arbeiten} `work' $\rightarrow$ \textit{bearbeiten} `handle'. 
%And inflections of tense in German do not change the valency of the input verb.
Further, if \obj{le} is considered to be a morphological element (e.g.\ \citealt[101--102]{Huangetal2009}; \citealt[246]{MuellerLipenkova2013}), the insertion of \obj{le} does not have to be viewed as a syntactic
process either.
It seems that \citet{Xie2020} only shows that AA and ABAB are different processes than AABB, but not necessarily that the former is syntactic while the latter morphological.

A reviewer claims that \obj{le} insertion can be seen as a violation of lexical integrity,
because it is never found in between the two constituents of a compound word,
but must be placed after the whole unit.
For instance, \citet[1282]{Her2006} claims that the V-\obj{gei3} sequence cannot be separated 
and uses this as evidence for analyzing it as a single lexical item 
(\obj{ji4-gei3-le} \obj{ta1} `send-give-\textsc{pfv} he, sent him' vs. ?\ \obj{ji4-le-gei3} \obj{ta1} `send-\textsc{pfv}-give he').
In non-separable VO compounds, \obj{le} insertion also does not seem to be possible 
(\obj{guan1-xin1-le} `close-heart-\textsc{pfv}, cared for' vs. ?\ \obj{guan1-le-}\obj{xin1} `close-\textsc{pfv}-heart').
The AABB form of reduplication also only accepts \obj{le} to its right but not in between (\ref{ex:yaohuang}).
\ea\label{ex:yaohuang} %我趁机从大货车以及左边的轿车车门上贴肤而过,	摇摇晃晃了	几下
\ea[]{\gll \objex{wo3} \ldots\, \obj{yao2-yao2-huang4-huang3-le} \objex{ji3} \objex{xia4} \ldots\\
I {} shake-shake-sway-sway-\textsc{pfv} several time\\\jambox{(CCL)}
\glt `I shook and swayed several times \ldots'}

\ex[*]{ \gll \objex{wo3} \ldots\, \obj{yao2-yao2-le-huang4-huang3} \objex{ji3} \objex{xia4} \ldots\\
I {} shake-shake-\textsc{pfv}-sway-sway several time\\}
\z\z

In respond to this, we found counter-examples that show \obj{le} insertion in between V-\obj{gei3} (\ref{ex:Vgei}) as well as \obj{guan1-xin1} `close-heart, care for' (\ref{ex:guanxin}) is possible.
\ea\label{ex:Vgei}%还有许多女子,将自己的相片,亲笔签字在上面,寄了给他。(CCL)
\gll \ldots\, \objex{xu3duo1} \objex{nv2zi3}, \objex{jiang1} \objex{zi4ji3de} \objex{xiang4pian1} \ldots\, \objex{ji4-le-gei3} \objex{ta1}.\\
{} many women take own photo {} send-\textsc{pfv}-give he\\\jambox{(CCL)}
\glt `Many women \ldots\, sent him photos of themselves.'
\z

\ea\label{ex:guanxin}%许多同志看戏以后,自动的对病员关了心。大家情绪很高... (CCL)
\gll \objex{xu3duo1} \objex{tong2zhi4} \ldots\, \objex{zi4dong4de} \objex{dui4} \objex{bing4yuan2} \objex{guan1-le-}\obj{xin1}.\\
many comrade {} voluntarily to patient close-\textsc{pfv}-heart\\\jambox{(CCL)}
\glt `Many comrades \ldots\, voluntarily cared for the patients.'
\z

In any case, since reduplication is not compounding (\citealt[149--150]{Sui2018}; \citealt{GaoEtAl2021} provide psycholinguistic evidence),
and the patterns discussed here constitute a different process than the AABB pattern
(see the discussions above, also \citealt[Sec.\,4.3]{Deng2013}, \citealt[Sec.\,2]{SuiHu2016}, \citealt{Sui2018} and \citealt{Wang2023}),
it is not surprising that \obj{le} occurs at a different position.

It is, therefore, necessary to resort to other tests that are intended to distinguish words from phrases in Mandarin Chinese. 
%Furthermore, we applied the same tests to Parallel Verb Compounds, which are inarguably words (because at least one element in the compound cannot appear in a sentence independently), 
%and {SVC}s, which are by definition phrases. 
%We then compared the test results of these three kinds of expressions to see whether the reduplication behaves in a more similar way to Parallel Verb Compounds or to {SVC}s.
 For this purpose, \citeauthor{Dai1992} (\citeyear[32--33]{Dai1992}, \citeyear[117--120]{Dai1998}) proposes the modification and the expansion tests.
%These tests requires further clarifications.

First, the modification test suggests that subparts of a word cannot be modified at a phrasal level.
This is possible for a VP (\ref{ex:modVP}), as the NP inside of the VP can be modified by e.g.\ an AP.
\ea\label{ex:modVP} %开红色的门
\gll \objex{kai1} \objex{hong2se4de} \objex{men2}\\
open  red door\\
\glt `open the red door'
\z
In contrast, the individual verbs in reduplication cannot be modified by an e.g.\ AdvP.
(\ref{ex:modVV}) is ungrammatical whether the AdvP is interpreted to modify the first or the second verb.
This shows that it has nothing to do with the relative position of the verb and the AdvP.
\ea[*]{\label{ex:modVV} %看偷偷地看
    \gll \objex{kan4} \objex{tou1tou1de} \objex{kan4}\\
    look secretly look\\
}
\z


Second, the expansion test suggests that a phrasal dependent (either a modifier or an argument) cannot be inserted into a word.
This is possible for a verbal classifier phrase (\ref{ex:inVCP}), as the object can occur after or in between.
\ea\label{ex:inVCP}%开两次门
%开门两次
\ea \gll \objex{kai1} \objex{men2} \objex{san1} \objex{ci4}\\
open door three time\\
\glt `open the door three times'

\ex  \gll \objex{kai1}  \objex{san1} \objex{ci4} \objex{men2}\\
open three time door\\
\glt `open the door three times'
\z\z
For reduplication, this is also not possible (\ref{ex:inVV}), as the object cannot be inserted between the two verbs.
\ea\label{ex:inVV}
\ea[]{\gll \obj{kai1}-(\obj{le})-\obj{kai1} \objex{men2}\\
    open-\textsc{pfv}-open door\\
    \glt `open the door for a little while'}

\ex[*]{%开(了)门(了)开
    \gll \obj{kai1}-(\obj{le})-\obj{men2}-\obj{kai1}\\
    open-\textsc{pfv}-door-open\\
}
\z\z
The above two tests seem to indicate a lexical analysis for reduplication.

Cross-linguistically, verbal reduplication in Mandarin Chinese patterns more with morphological reduplication (below as \textit{reduplication}) in other languages
than syntactic reduplication (below as \textit{repetition}; \citealt[31]{Gil2005}, \citealt[1--2]{Forza2016}).
\citet[35--36]{Gil2005} considers non-iconicity and having only two (but not more) copies as sufficient but not necessary conditions for reduplication.
He further proposes building one intonational group as sufficient and necessary condition for reduplication (p. 36).
All three conditions are true for verbal reduplication in Mandarin Chinese (see \citealt[154]{Sui2018} on the intonation of verbal reduplication in Mandarin Chinese).
\citet[9]{Forza2016} argues that the substantial difference between reduplication and repetition lies in the fact that only the former affects grammatical features such as aspect.
This is also the case for verbal reduplication in Mandarin Chinese.

In sum, we maintain that verbal reduplication in Mandarin Chinese is better off analyzed as a morphological phenomenon.


%\citet{Duanmu1998} and \citet{Schaefer2009} proposed the following four tests to
%distinguish words from phrases in Mandarin Chinese: semantic compositionality, phrasal extension,
%phrasal substitution and conjunction reduction.\footnote{It is important to note that none of these criteria are sufficient or necessary to determine the word or phrase status of an expression. Nevertheless, they together might suggest which of the two statuses is more likely.}

%The semantic criterion is that the meaning of a phrase is usually built up in a compositional way while that of a is word usually not \citetext{\citealp[140]{Duanmu1998}; \citealp[275]{Schaefer2009}}. 
%The meaning of the reduplication is not compositional, as it does not mean that the event denoted by the verb happens twice or multiple times, but rather that the event happens for a short duration and/or for low frequency.
%This non-compositionality suggests that a reduplication is more word-like.


%The syntactic criteria are all based on the Lexical Integrity Hypothesis according to \citet[60]{Huang1984}, as shown in (\ref{lih}):

%\ea\label{lih}The Lexical Integrity Hypothesis\\
%No phrase-level rule may affect a proper subpart of a word.
%\z

%The first syntactic test is phrasal extension, namely the addition of optional elements \citetext{\citealp[150]{Duanmu1998}; \citealp[280]{Schaefer2009}}. 
%Optional elements that can possibly appear in a phrase should be able to be added into it,
%and subparts of a phrase should be able to be modified separately. 
%If the unit is a word, however, then neither of these should  be possible.
%As illustrated in (\ref{ex:forms-mono})--(\ref{ex:forms-vo}) in \sectref{sec:forms}, the reduplication can only be separated by \textit{le} and \textit{yi}. 
%As mentioned above, whether aspect markers are considered to be morphological or syntactic elements depends on the theoretical framework (and possibly the target language).\footnote{For example, \citet[Sec.\,3.3.1]{Huangetal2009} and \citet[246]{MuellerLipenkova2013} considered the postverbal aspect markers  in Mandarin Chinese (\textit{le}, \textit{zhe} and \textit{guo}) to be morphological elements. E.g. \citet{Travis2000}, \citet{Ramchand2008},  \citet[23]{Arcodiaetal2014}, \citet[146]{BascianoMelloni2017} considered all aspect information to be syntactically encoded.} 
%And the status of \textit{yi} is unclear, since it does not carry any additional meaning or grammatical function in the structure. % (more on this topic see Section \ref{sec:jackendoff}).
%Turning to separate modification,  the element in the reduplication cannot be modified individually. 
%Compared to (\ref{ex:ext-redup1}), where the adverbial \textit{qingsheng de} `quietly' modifies the whole reduplication, (\ref{ex:ext-redup2}) is ungrammatical, 
%as the adverbial cannot modify the second element in the reduplication alone.
%It is worth mentioning that in (\ref{ex:ext-redup1}), it is not possible to analyze \textit{qingsheng de} `quietly' to modify only the first element in the reduplication, 
%because the adverbial does not necessarily occur adjacent to the verb, as in (\ref{ex:ext-redup3}), 
%and it still modifies the whole predicate.
%(\ref{ex:ext-redup1}) and  (\ref{ex:ext-redup3}) have the same meaning.

%\ea
%  \ea[]{\label{ex:ext-redup1}
%  \gll ta dui ziji qingsheng de \textit{xiao-le-xiao}.\\
%  he to \textsc{refl} quietly \textsc{de} laugh-\textsc{le}-laugh\\
%  \glt `He quietly laughed a little bit to himself.'}
  
%  \ex[*]{\label{ex:ext-redup2}
%  \gll ta dui ziji \textit{xiao}-\textit{le} qingsheng de \textit{xiao}.\\
%  he to \textsc{refl} laugh-\textsc{pfv} quietly \textsc{de} laugh\\}
  
%  \ex[]{\label{ex:ext-redup3}
% \gll ta qingsheng de dui ziji \textit{xiao-le-xiao}.\\
%  he quietly \textsc{de} to \textsc{refl} laugh-\textsc{pfv}-laugh\\
%  \glt `He quietly laughed a little bit to himself.'}
%  \z
%\z

%Nothing can intervene between a Parallel Verb Compound and its subparts cannot be modified separately either (\ref{ex:ext-pvc}).

%\ea\label{ex:ext-pvc}
%  \ea[]{
%  \gll ta gou-mai-le yi jian dayi.\\
%  he purchase-buy-\textsc{pfv} one \textsc{clf} coat\\
%  \glt `He bought a coat.'}
  
%  \ex[*]{
%  \gll ta gou-le-mai yi jian dayi.\\
%  he purchase-\textsc{pfv}-buy one \textsc{clf} coat\\}
  
%  \ex[*]{
%  \gll ta gou-le-yi-mai yi jian dayi.\\
%  he purchase-\textsc{pfv}-one-buy one \textsc{clf} coat\\}
  
%  \ex[]{
%  \gll ta kaixin de gou-mai-le yi jian dayi.\\
%  he happily \textsc{de} purchase-buy-\textsc{pfv} one \textsc{clf} coat\\
%  \glt `He bought a coat happily.'}
  
%  \ex[*]{
%  \gll ta gou kaixin de mai-le yi jian dayi.\\
%  he purchase happily \textsc{de} buy-\textsc{pfv} one \textsc{clf} coat\\}
%  \z
%\z

%But an {SVC} readily allows other elements to appear between the two verbs (in (\ref{ex:ext-svc1}), the direct object of the first verb) and each verb can be modified separately (\ref{ex:ext-svc2}), as well.

%\ea
%  \ea\label{ex:ext-svc1}
%  \gll ta zhong cai mai.\\
%  he plant vegetables sell\\
%  \glt `He plants vegetables and sells them.
  
%  \ex\label{ex:ext-svc2}
%  \gll ta xinku de zhong cai qinfen de mai.\\
%  he hardworkingly \textsc{de} plant vegetables diligently \textsc{de} sell\\
%  \glt `He plants vegetables hard-workingly and sells them diligently.'
%  \z
%\z
%All in all, by the test of phrasal extension, reduplications behave more like words than like phrases.



%The second syntactic test is phrasal substitution, namely the substitution of smaller exemplars of a specific category with a full blown XP \citetext{\citealp[152]{Duanmu1998}; \citealp[280]{Schaefer2009}}. 
%If a part of an expression is actually an XP that only contains one element, a full realization of this XP should be possible as well.
%Otherwise, this expression is considered to be a word.
%In a reduplication structure, it is ungrammatical to substitute each element with a full {VP} (\ref{ex:sub-redup}).

%\ea\label{ex:sub-redup}
%  \ea[]{\label{ex-he-tasted-the-soup-a-little-bit}
%  \gll ta \textit{chang}-\textit{le}-\textit{chang} tang.\\
 %   he taste-\textsc{pfv}-taste soup\\
 %   \glt `He tasted the soup a little bit.'}
    
 %   \ex[*]{
 %   \gll ta \textit{chang} \textit{tang} \textit{le} \textit{chang} \textit{tang}.\\
 %   he taste soup \textsc{pfv} taste soup\\}
 % \z
%\z

%It is the same case with Parallel Verb Compounds (\ref{ex:sub-pvc}).

%\ea[*]{
%\gll ta gou yi jian dayi mai le yi tiao kuzi.\\
%he purchase one \textsc{clf} coat buy \textsc{pfv} one \textsc{clf} pants\\}
%\label{ex:sub-pvc}
%\z

%However, it is grammatical to replace a reduced part of a {SVC} with a full VP (\ref{ex:sub-svc}).

%\ea\label{ex:sub-svc}
%\gll ta zhong cai mai (cai).\\
% he plant vegetables sell vegetables\\
%\glt `He plants vegetables and sells vegetables.'
 %\z
%Again, reduplications look more word-like than phrase-like.


%The third syntactic criterion is conjunction reduction. It should only be possible for coordinated phrases (\ref{ex:co-phrase}) and not for coordinated words (\ref{ex:co-word}) \citetext{\citealp[137]{Duanmu1998}; \citealp[283]{Schaefer2009}}.

%\settowidth\jamwidth{\citep[137]{Duanmu1998}}

%\ea\label{ex:co-phrase}
 % \ea \gll [jiu de shu] gen [xin de shu]\\
%  old \textsc{de} book and new \textsc{de} book\\ \jambox{\citep[137]{Duanmu1998}}
%  \glt `old books and new books'
  
%  \ex \gll [jiu de gen xin de] shu\\
%  old \textsc{de} and new \textsc{de} book\\ \jambox{\citep[137]{Duanmu1998}}
%  \glt `old and new books'
%   \z
%\z

%\ea\label{ex:co-word}
%  \ea[]{
%  \gll [huo-che] gen [qi-che]\\
%  fire-car and gas-car\\ \jambox{\citep[137]{Duanmu1998}}
%  \glt `train and automobile'}
  
%  \ex[*]{
%  \gll [huo gen qi] che\\
%  fire and gas car\\ \jambox{\citep[137]{Duanmu1998}}}
%  \z
%\z

%For the reduplication, conjunction reduction does not seem to be possible. 
%In (\ref{ex:co-redup1}), the reduplication \textit{jiao-jiao} `chew a little bit' is coordinated with a simple verb \textit{mo} `apply' together with the adverbial \textit{yidian} `a little bit'.
%Without the adverbial \textit{yidian} `a little bit', \textit{mo} `apply' by itself cannot express the additional `a little bit' meaning even when it is coordinated with a reduplicated verb.
%Similarly, in (\ref{ex:co-redup2}), the reduplication \textit{kan-le-kan} `look a little bit' is coordinated with the predicate \textit{zou-le chulai} `walked out'. The verb in the latter case is not reduplicated and it cannot express the delimitative meaning either.

%\settowidth\jamwidth{(CCL)}

%\ea
%  \ea\label{ex:co-redup1}
%  \gll wujian gong-xiu mo dian bohe-gao huo \textit{jiao}-\textit{jiao} kouxiangtang.\\
%    midday work-break apply a.little mint-cream or chew-chew chewing.gum\\ \jambox{({CCL})}
%    \glt `During the working break in the midday, apply a little bit of mint cream or chew some chewing gum a little bit.'
    
%   \ex\label{ex:co-redup2}
%   \gll Song Ailing \textit{kan}-\textit{le}-\textit{kan} yupen you zou-le chulai.\\
%   Song Ailing look-\textsc{pfv}-look bath.tub again walk-\textsc{pfv} out\\  \jambox{({CCL})}
%   \glt `Song Ailing looked at the bath tub a little bit and walked out again.'
%   \z
%\z

%Conjunction reduction is not possible for Parallel Verb Compounds as well.
%We can see in (\ref{ex:co-pvc}) that for the Parallel Verb Compounds \textit{gou-mai} `purchase-buy, buy' and \textit{gou\hyp{}zhi} `purchase\hyp{}place, purchase', it is neither possible to delete the first verb nor the second.

%\ea\label{ex:co-pvc}
 %    \ea[*]{
 %    \gll ta gou-mai-le yi jian dayi hai gou-le yi tiao kuzi.\\
 %   he purchase-buy-\textsc{pfv} one \textsc{clf} coat and purchase-\textsc{pfv} one \textsc{clf} pants\\
%    \glt Intended: `He bought a coat and purchased a pair of pants.'}
    
%    \ex[*]{
%    \gll ta gou-mai-le yi jian dayi hai -zhi-le yi tiao kuzi.\\
%    he purchase-buy-\textsc{pfv} one \textsc{clf} coat and place-\textsc{pfv} one \textsc{clf} pants\\
%   \glt  Intended: `He bought a coat and purchased a pair of pants.'}
%    \z
%\z

%By contrast, it is grammatical to delete a repeated part in a coordinated {SVC}. 
%In (\ref{ex:co-svc1}), while it is possible for the object of selling to be only cows, it is equally acceptable to interpret it as the person sells both vegetables and cows. With the second interpretation, (\ref{ex:co-svc1}) can be understood as a reduced version of (\ref{ex:co-svc2}).

%\ea
%  \ea\label{ex:co-svc1}
%  \gll ta zhong cai hai yang niu mai.\\
%  he plant vegetables and farm cow sell\\
%  \glt `He plants vegetables and farms cows and sells them.'
  
%  \ex\label{ex:co-svc2}
%  \gll ta zhong cai mai hai yang niu mai.\\
%  he plant vegetables sell and farm cow sell\\
%  \glt `He plants vegetables and sells them and farms cows and sells them.'
 %  \z
%\z
%Once again, this criterion suggests that reduplications do not have this expected property of phrases.


%Following the analyses above, it is clear that the reduplication failed all of the tests for phrasal status. 
%In comparison, Parallel Verb Compounds failed all the tests as well, whereas {SVC}s passed all of them.
%This makes the behavior of the reduplication seem more similar to that of a Parallel Verb Compound than that of an {SVC}.
%Therefore, it seems more appropriate to assume reduplication to be a morphological process rather than a syntactic one. 





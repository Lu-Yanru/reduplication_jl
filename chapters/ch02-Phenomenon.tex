\section{The phenomenon}\label{sec:phen}

This section introduces the fundamental grammatical behaviors of verbal reduplication in Mandarin Chinese.
 After illustrating its forms, syntactic distribution and semantics, 
 we discuss the questions of whether it is better analyzed as a morphological or a syntactic phenomenon.


%%%%%%%%%%%%%
\subsection{Forms}\label{sec:forms}

There is no general agreement on the forms of verbal reduplication in Mandarin Chinese.
We adopt a broad definition in terms of the forms of verbal reduplication in Mandarin Chinese 
and list in (\ref{ex:forms-mono})--(\ref{ex:forms-vo}) all the forms commonly discussed in the literature.

\settowidth\jamwidth{AB-\textit{le}-AB}

\ea\label{ex:forms-mono} for monosyllabic verbs: \textit{shuo} `say'
	\ea \gll shuo-shuo\\ 
		say-say\\ \jambox{AA}
		\ex \gll shuo-yi-shuo\\
		say-one-say\\ \jambox{A-\textit{yi}-A}
		\ex \gll shuo-le-shuo\\
		say-\textsc{pfv}-say\\ \jambox{A-\textit{le}-A}
		\ex \gll shuo-le-yi-shuo\\
		say-\textsc{pfv}-one-say\\ \jambox{A-\textit{le}-\textit{yi}-A}
		\ex \gll shuo-shuo-kan\\
		say-say-look\\ \jambox{AA-\textit{kan}}
		\ex \gll shuo-kan-kan\\
		 say-look-look\\ \jambox{A-\textit{kan}-\textit{kan}}
		\z

\ex\label{ex:forms-di} for disyllabic verbs: \textit{lai-wang} come-go `come and go/communicate'
		\ea \gll lai-wang-lai-wang\\
		come-go-come-go\\ \jambox{ABAB}
		\ex \gll lai-wang-le-lai-wang\\
		come-go-\textsc{pfv}-come-go\\ \jambox{AB-\textit{le}-AB}
		\ex \gll lai-lai-wang-wang\\
		come-come-go-go\\ \jambox{AABB}
		\z

\ex\label{ex:forms-vo} for V-O compounds: \textit{shuo-huang} tell-lie `lie'
		\ea \gll shuo-shuo-huang\\
		tell-tell-lie\\ \jambox{AAB}
		\ex \gll shuo-yi-shuo-huang\\
		tell-one-tell-lie\\ \jambox{A-\textit{yi}-AB}
		\ex \gll shuo-le-shuo-huang\\
		tell-\textsc{pfv}-tell-lie\\ \jambox{A-\textit{le}-AB}
		\z
\z




 \citet{Fan1964}, \citet{Arcodiaetal2014}  and \citet{Xie2020} compared the AA, ABAB and AABB forms of reduplication 
and found a number of differences between the AA, ABAB forms compared to the AABB form in terms of their semantics, productivity, syntactic distribution and origin. 
Specifically, \citet[17--18]{Arcodiaetal2014}, \citet[144]{MelloniBasciano2018} and \citet[90]{Xie2020} identified that AA and ABAB have a diminishing meaning, 
namely that the event happens for a short duration or to a small extent. 
By contrast, AABB expresses an increasing meaning, which indicates a repetition or action in progress. 
\citet[Sec. 3.1]{Xie2020} also found that AA and ABAB have relatively high productivity, 
whereas the productivity of AABB is low. 
She further showed that AABB is generally correlated with the lack of a postverbal object, but the direct object remains present when a transitive verb undergoes AA or ABAB patterns of reduplication. 
\citet[277]{Fan1964} proposed that AA, ABAB originated from the verb-measure word combination from Middle Chinese, 
while AABB developed from the reiterative rhetoric from Old Chinese. 
These differences seem to suggest that there is a fundmental difference between these two groups of verbal reduplication. 
The current study will only focus on the AA, A-\textit{yi}-A, A-\textit{le}-A, A-\textit{le}-\textit{yi}-A, ABAB and AB-\textit{le}-AB forms of verbal reduplication in Mandarin Chinese.
AA-\textit{kan}, A-\textit{kan}-\textit{kan}, AAB, A-\textit{yi}-AB, A-\textit{le}-AB will also be mentioned occasionally to provide further arguments.
In what follows, the term \textit{reduplication} will be used  to refer specifically to the AA, A\hyp{}\textit{yi}\hyp{}A, A-\textit{le}-A, A-\textit{le}-\textit{yi}-A, ABAB and AB-\textit{le}-AB forms, if not specified otherwise.




%%%%%%%%%%%%
\subsection{Syntactic distribution}\label{sec:syn-dis}

The reduplication has a similar syntactic distribution as an unreduplicated verb (\ref{ex:syn-vi})--(\ref{ex:syn-svc}). 
The reduplication cannot be aspect marked, though, except with the perfective aspect marker \textit{le} (for further discussions see \sectref{sec:aspM}). 
The reduplication is incompatible with an expression that quantifies the duration or the extent of the event expressed in the sentence (\ref{ex:syn-quan}) \citetext{ \citealp[83--84]{Li1998}; \citealp[114--115]{Chen2005}}.
This is probably because the reduplication already contains a quantity meaning \citetext{\citealp[84]{Li1998}; \citealp[114--115]{Chen2005}}, namely a short duration or a small extent.

\settowidth\jamwidth{\citep[83]{Li1998}}
\begin{sloppypar}
\ea\label{ex:syn-vi} Intransitive verb:
		\ea \gll ta xiao-le.\\
		he laugh-\textsc{pfv}\\
		\glt `He laughed.'
		\ex \gll ta \textit{xiao}-\textit{le}-\textit{xiao}.\\
		he laugh-\textsc{pfv}-laugh\\
		\glt `He laughed a little bit.'
		\z
        
\ex Transitive verb:
		\ea \gll ni wen ta.\\
		you ask him\\
		\glt `Ask him.'
		\ex \gll ni \textit{wen}-\textit{(yi)}-\textit{wen} ta.\\
		you ask-(one)-ask he\\
		\glt `Try to ask him.'
		\z

\ex Under negation:
		\ea \gll weishenme wo bu kan ta qita fangmian de jinbu he quanfangwei de chengzhang\\
		why I not look he other aspect \textsc{de} progress and comprehensive \textsc{de} development\\
		\glt `Why don't I look at his progress in other aspects and comprehensive development?'
		\ex \gll weishenme wo bu \textit{kan}-\textit{kan} ta qita fangmian de jinbu he quanfangwei de chengzhang\\
		why I not look-look he other aspect \textsc{de} progress and comprehensive \textsc{de} development\\ \jambox{({CCL})}
		\glt `Why don't I look a little bit at his progress in other aspects and comprehensive development?'
		\z

\ex In \textit{ba}-construction:
		\ea \gll gou-mai zhiqian zhen gai ba qingkuang mo qingchu.\\
		purchase-buy before really should \textsc{ba} situation touch clearly\\
		\glt `(I) should really check the situation clearly before I make the purchase.'
		\ex \gll gou-mai zhiqian zhen gai ba qingkuang \textit{mo}-\textit{mo} qingchu.\\
		purchase-buy before really should \textsc{ba} situation touch-touch clearly\\ \jambox{({CCL})}
		\glt `(I) should really quickly check the situation clearly before I make the purchase.'
		\z

\ex With modal verb:
		\ea \gll you liang ben shu ni-men keyi kan \ldots\\
		there.be two \textsc{clf} book you-\textsc{pl} can read\\
		\glt `There are two books that you can read \ldots'
		\ex \gll you liang ben shu ni-men keyi \textit{kan}-\textit{kan} \ldots\\
		there.be two \textsc{clf} book you-\textsc{pl} can read-read\\ \jambox{({CCL})}
		\glt `There are two books that you can read a little bit \ldots'
		\z

\ex\label{ex:syn-svc} In Serial Verb Construction (SVC):
		\ea \gll ta dao jinian-tang dong-ce de yi chu chongxi-bu qing shifu bangmang kan nali chu-le wenti.\\
		she go.to memorial-hall east-side \textsc{de} one place film.developing-department ask master help look where come.out-\textsc{pfv} problem\\
		\glt `She went to a film developing department at the east side of the memorial hall to ask the master to help have a look at where went wrong.'
		\ex \gll ta dao jinian-tang dong-ce de yi chu chongxi-bu qing shifu bangmang \textit{kan}-\textit{kan} nali chu-le wenti.\\
		she go.to memorial-hall east-side \textsc{de} one place film.developing-department ask master help look-look where come.out-\textsc{pfv} problem\\ \jambox{({CCL})}
		\glt `She went to a film developing department at the east side of the memorial hall to ask the master to help have a quick look at where went wrong.'
		\z

\ex With an expression of quantity: \label{ex:syn-quan}
		\ea[]{\gll ta yi tian pao shi li.\\
		he one day run ten mile\\ \jambox{\citep[83]{Li1998}}
		\glt `He runs ten miles a day.'}
		\ex[*]{\gll ta yi tian \textit{pao}-\textit{pao} shi li.\\
		he one day run-run ten mile\\}
		\z
\z
\end{sloppypar}




%%%%%%%%%%%%%
\subsection{Semantics}\label{sec:sem}

In this section, we will first discuss the core meaning of the reduplication as well as the meaning of its different forms (\sectref{sec:core-sem}).
We will then investigate the interaction of the reduplication with \textit{Aktionsarten} (\sectref{sec:Aktionsarten}) and aspect markers (\sectref{sec:aspM}).

\subsubsection{Core meaning}\label{sec:core-sem}

The reduplication has a \textit{delimitativeness} meaning (e.g. \citealt[204--205]{Chao1968}; \citealt[232]{LiThompson1981};  \citealt[14]{Li1996}; \citealt[70]{Dai1997};  \citealt[382--383]{Zhu1998}; \citealt[420--421]{Xing2000}; \citealt[48]{Chen2001};  \citealt[288]{Tsao2001}; \citealt[11--12]{Yang2003}; \citealt[Sec. 4.3]{XiaoMcEnery2004}). 
To be specific, the reduplication of [$+$durative] verbs reduces the duration of the events,
and the reduplication of [$-$durative] verbs reduces the iteration frequency of the events \citetext{\citealp[14]{Li1996}; \citealp[149--150]{XiaoMcEnery2004}}.
Besides delimitativeness, \citet[204]{Chao1968}, \citet[276]{Fan1964}, \citeauthor{Smith1991} (\citeyear[356]{Smith1991}; \citeyear[199--120]{Smith1994}), \citet[14]{Li1996} and \citet[290--291]{Tsao2001} suggested that the reduplication signifies \textit{tentativeness}, which can be used
``to refer modestly to one's own activities, or for mild imperatives'' \citep[356]{Smith1991}, or ``trying to'' do something \citep[234]{LiThompson1981}.
\textit{Frequentativeness} or \textit{habitualness}, that the event denoted by the verb happens frequently or habitually, is mentioned by \citet[276]{Fan1964}, \citet[15]{Li1996} and \citet[1]{Qian2000} as the meaning of reduplication, as well.
 \citet[276]{Fan1964} further proposed a meaning of \textit{slightness} or \textit{casualness} for reduplication, which implies that the event is unimportant or conveys a casual attitude of the speaker.
 \citet[Sec. 3.1.3]{Zhu1998} suggested that the main function of reduplication is to \textit{increase the agency} of the action or the change denoted by the verb.


In general, all of the above cited research agreed that the reduplication expresses a short duration and/or a low frequency, which fits the definition of delimitativeness.
\citet[152--154]{XiaoMcEnery2004} and \citet{Yang2003} argued that the core meaning of reduplication is delimitativeness, 
while all other meanings are merely pragmatic extensions  in specific contexts.
\citet[152--154]{XiaoMcEnery2004} pointed out that tentativeness and casualness are constrained by a number of contextual elements 
such as the reduplicated verb must be volitional and the subject of the sentence must be animate.
But these constraints are only necessary but not sufficient conditions for a tentative or casual meaning of reduplication.
Among all instances of verbal reduplication they found in a corpus, all of them have a delimitative reading, while only some of them convey tentativeness or casualness.
\citet{Yang2003} compared the sentence pairs with reduplicated verbs and their unreduplicated counterparts,
and showed that the reduplication itself does not add a tentative, frequentative, casualness or increased agency meaning to the sentence.
Rather, these additional meanings arise from the sentences or the contexts as a whole.
She concluded that these additional meanings are results of meaning extensions of delimitativeness in specific contexts.
We follow \citet{XiaoMcEnery2004} and \citet{Yang2003} and treat delimitativeness as the central meaning of reduplication,
and the other meanings as pragmatic extensions.

The semantics of the reduplication has the properties of transitoriness,  holisticity  and dynamicity \citetext{\citealp[70--79]{Dai1997}; \citealp[155--159]{XiaoMcEnery2004}}.
It presents the situation as a  transitory and non\hyp{}decomposable whole.
A situation expressed by a sentence with the reduplication involves changes not only in the initiation and termination of an event, but also in the transitory process itself.
Compared to (\ref{ex:le-dyn-look}), which could mean that the protagonist kept staring at the the footprint,
(\ref{ex:redup-dyn-look}) indicates that the protagonist took a brief look or several brief looks at the footprint and looked away in the end, which is a process full of changes.

\settowidth\jamwidth{\citep[158]{XiaoMcEnery2004}}

\ea
  \ea\label{ex:le-dyn-look}
    \gll Wu Xumang kan-le zuo-an shi liuxia de jiaoyin \ldots\\ 
    Wu Xumang look-\textsc{pfv} commit-crime when leave \textsc{de} footprint\\ \jambox{\citep[158]{XiaoMcEnery2004}}
    \glt `Wu Xumang looked at the footprint left when the crime was committed.'
  \ex\label{ex:redup-dyn-look}
    \gll Wu Xumang \textit{kan-le-kan} zuo-an shi liuxia de jiaoyin \ldots\\
    Wu Xumang look-\textsc{pfv}-look commit-crime when leave \textsc{de} footprint\\ \jambox{\citep[158]{XiaoMcEnery2004}}
    \glt `Wu Xumang looked a little bit at the footprint left when the crime was committed.'
  \z
\z

The semantics of A-\textit{le}-A can be deduced compositionally from its structure. 
It is a hierarchical combination of the perfective aspect and delimitativeness, ``conveying a transitory event which has been actualized'' \citep[151]{XiaoMcEnery2004}.


As for A-\textit{yi}-A, \citet[273]{Fan1964} compared examples found in novels and plays and concluded that A-\textit{yi}-A has exactly the same meaning as its AA counterpart.
She thus assumed that AA is merely a form of A-\textit{yi}-A, where the \textit{yi} is omitted phonologically.
\citet[Sec. 5]{Xing2000} considered that the major  difference in meaning  between AA and A-\textit{yi}-A lies in the speaker's attitude.
The former conveys a casual attitude whereas the latter sounds more serious.
However, he stressed that there is no difference in the delimitative semantics of both forms,
and that the variance in meaning is a pragmatic one.
The difference is also not absolute and often only shows a tendency.
\citet{Xu2002} found out that compared to A-\textit{yi}-A, one tends to use AA in contexts with strong emotional attitudes, urgent, casual, timid or uncertain contexts.
But he also stated that these differences are pragmatic rather than semantic, 
as he argued that AA and A-\textit{yi}-A can be used interchangeably in most cases,
and the specific differences in meaning only arise from specific contexts as a whole.
\citet[15]{Yang2003} suggested that AA and A-\textit{yi}-A have the same core meaning, while A-\textit{yi}-A implies a slightly more serious attitude than AA due to its length.
We assume A-\textit{yi}-A to be a form of reduplication and that it has the same core semantics as AA.

AA-\textit{kan} and A-\textit{kan}-\textit{kan} are described to express a ``try \ldots and find out'' meaning \citep[63]{Cheng2012}.
\citet[290]{Tsao2001} also observed that the tentative meaning is particularly prominent when the reduplication is followed by \textit{kan} `look'.
We still consider the tentativeness implied by these two forms to be a pragmatic extension of delimitativeness.
The tentative meaning is made prominent by the verb \textit{kan} `look',
and the whole structure can be understood as ``do A a little bit and see''.




%%%%%%%%
\subsubsection{Interaction with \textit{Aktionsarten}}\label{sec:Aktionsarten}

Previous research often claimed that the reduplication can only be used for verb classes of certain \textit{Aktionsarten}, while it is infelicitous for other ones.
 \citet[234--235]{LiThompson1981} and \citet[277--278]{Hong1999} suggested that reduplication is only possible for volitional activity verbs.
\citet[70--71]{Dai1997} und \citet[290]{Tsao2001} both considered that reduplication can only be used in dynamic situations.
The former further claimed that achievement verbs cannot be reduplicated.
 \citet[155]{XiaoMcEnery2004}, \citet[20]{Arcodiaetal2014} and \citet[145]{BascianoMelloni2017} proposed that only [$+$dynamic] and [$-$result] verbs can be reduplicated.
This means that the reduplication can only interact with dynamic situations which encode no results and is consequently only compatible with activities and semelfactives, but not with states and achievements.

\citet[53]{Chen2001} and \citet[10--11]{Yang2003} acknowledged that the reduplication of non\hyp{}vo\-li\-tion\-al verbs is more restricted than that of volitional ones.
But \citet[381--382]{Zhu1998} listed a number of non\hyp{}volitional predicates that can be reduplicated.
We found the examples shown in (\ref{ex:nonvol}) in {CCL} where non\nobreakdash-vo\-li\-tion\-al verbs \textit{weiqu} `feel wronged', \textit{ren-xing} `be willful' and \textit{diao} `drop' are reduplicated.

\settowidth\jamwidth{(CCL)}

\begin{sloppypar}
\ea\label{ex:nonvol}
\ea
\gll keshi xian mujin, dajia ye zhihao \textit{weiqu-weiqu} le.\\
but now current everybody also can.only feel.wronged-feel.wronged \textsc{ptc}\\ \jambox{({CCL})}
\glt `But now, everybody can only feel wronged a little bit.'

\ex
\gll ta-men neng zuo de buguo shi \textit{ren-ren-xing} shua dian'er xiao piqi \textit{diao-diao} yanlei shenme de.\\
she-\textsc{pl} can do \textsc{de} just be be.willful-be.willful-temperament play a.little small temper drop-drop tear what \textsc{de}\\ \jambox{({CCL})}
\glt `What they can do is just to be a little bit willful, to lose their temper a little bit and to drop a little bit of tears or something.'
\z
\z
\end{sloppypar}

It is true that  the reduplication of stative and achievement verbs is not as easily acceptable as that of activities and semelfactives.
Compared to the questionable reduplication of the stative verb \textit{bing} `be sick' in (\ref{ex:redup-stat})
and that of the achievement verb \textit{ying} `win' in (\ref{ex:redup-achiV}), 
the reduplication of the activity verb \textit{kan} `watch' in (\ref{ex:redup-actV})
and that of the semelfactive verb \textit{kesou} `cough' in (\ref{ex:redup-semel}) is readily acceptable.

\settowidth\jamwidth{\citep[155]{XiaoMcEnery2004}}

\ea
\ea[?]
{\gll ta \textit{bing-bing} jiu hao le.\\
he be.sick-be.sick then well \textsc{ptc}\\ \jambox{\citep[155]{XiaoMcEnery2004}}
Intended: `He was sick for a little while and then got well.'}\label{ex:redup-stat}

\ex[?]
{\gll ta \textit{ying}-\textit{le}-\textit{ying} na chang bisai.\\
he win-\textsc{pfv}-win that \textsc{clf} competition\\ \jambox{\citep[155]{XiaoMcEnery2004}}
\glt Intended: `He won that competition a little bit.'}\label{ex:redup-achiV}

\ex[]
{\gll ta \textit{kan}-\textit{le}-\textit{kan} na chang bisai.\\
he watch-\textsc{pfv}-watch that \textsc{clf} competition\\
\glt `He watched that competition for a little while.' \\}\label{ex:redup-actV}

\ex[]
{\gll ta \textit{kesou-kesou} jiu hao le.\\
he cough-cough then well \textsc{ptc}\\
\glt `He coughed a little bit and then got well.'}\label{ex:redup-semel}
\z
\z

However, examples such as those in (\ref{ex:redup-achi-stat1}--\subref{ex:redup-achi-stat3}) were found in novels and plays written by native speakers and example sentences like (\ref{ex:redup-achi-stat4}) and (\ref{ex:redup-achi-stat5}) constructed by native speaker linguists.
Here, achievement verbs like \textit{wang} `forget' and \textit{sheng} `give birth to' and stative verbs like \textit{shutan} `be comfortable', \textit{lian'ai} `be in love' and \textit{bing} `be sick' are reduplicated.

\settowidth\jamwidth{\citep[112]{Chen2005}}

\footnotetext[2]{Liu, Zhen. 1963. \textit{Chang chang de liushui [Long long water]}, 72. Beijing: The Writers Publishing House.}
\footnotetext[3]{Tian, Han. 1959. \textit{Tianhan xuanji [Selected works of Tianhan]}, 122. Beijing: People's Literature Publishing House.}
\footnotetext[4]{Zhou, Libo. 1958. \textit{Shang xiang ju bian [Big changes of mountains and the countryside]}, 204. Beijing: The Writers Publishing House.}
%\begin{sloppypar}
\ea
\ea\label{ex:redup-achi-stat1}
\gll deng ren-men ba zhe jian shi \textit{wang}-\textit{wang} zai shuo ba.\footnotemark\\
wait people-\textsc{pl} \textsc{ba} this \textsc{clf} incident forget-forget then talk \textsc{ptc}\\
\glt `Let's wait until people forget this incident a little bit and then talk about it.'\\

\ex\label{ex:redup-achi-stat2}
\gll huitou mo ge zao \textit{shutan}-\textit{shutan} ba.\footnotemark\\
later wipe \textsc{clf} bath be.comfortable-be.comfortable \textsc{ptc}\\
\glt `Let's take a bath later and be comfortable for a little while.'\\

\ex\label{ex:redup-achi-stat3}
\gll \textit{lian}-\textit{lian}-\textit{ai} shi keyi de, ban xishi dinghao chi yidian.\footnotemark\\
like-like-love \textsc{cop} ok \textsc{de} host wedding best late a.bit\\
\glt `It's ok to be in love a little bit, but it's better to get married a bit later.'\\

\ex\label{ex:redup-achi-stat4}
\gll wo zhen xiang \textit{bing-yi-bing}, xie ta ge shi tian ban yue.\\
I really want be.sick-one-be.sick rest it \textsc{clf} ten day half month\\ \jambox{\citep[54]{Chen2001}}
\glt `I really want to be sick for a little while and rest for ten days or half a month.'

\ex\label{ex:redup-achi-stat5}
\gll jiao ta \textit{sheng-sheng} xiaohai, jiu zhidao zuo muqin de gan-ku le.\\
let she give.birth.to-give.birth.to child then know \textsc{cop} mother \textsc{de} sweet-bitter \textsc{ptc}\\ \jambox{\citep[112]{Chen2005}}
\glt `Let her try to give birth to a child and then she will know the bittersweetness of being a mother.'
\z
\z
%\end{sloppypar}


This shows that although the reduplication does have a tendency to interact with volitional verbs and with activities and semelfactives due to its dynamic meaning, 
this is by no means a rigid constraint, 
and non\hyp{}volitional verbs, states and achievements can be reduplicated in appropriate contexts as well.




%%%%%%%%
\subsubsection{Interaction with aspect markers}\label{sec:aspM}

As mentioned in Section \ref{sec:forms}, the reduplication can only be marked by the perfective aspect marker \textit{le} but not other aspect markers.\footnote{
There is no consensus on which elements exactly are considered aspect markers in Mandarin Chinese. We only discuss the most commonly recognized ones here.
}
We believe this incompatibility to be for semantic reasons.

\citet[Ch. 4]{XiaoMcEnery2004} considered \textit{le}, \textit{guo} and reduplication to indicate perfective aspects, as they all view the situation as an inseparable whole.
The perfective aspect marker \textit{le} is compatible with reduplication while the experiential aspect marker \textit{guo} is not.
\citet[128--131]{XiaoMcEnery2004} stated that \textit{le} has the semantic feature of dynamicity, since it ``can focus on both heterogeneous internal structures and changing points'' \citep[129]{XiaoMcEnery2004}.
It can be combined with a situation with a dynamic internal structure, such as crying in (\ref{ex:cry}).
It can also co\hyp{}occur with a situation with a change at a certain time point, such as getting to know in (\ref{ex:know}).

\settowidth\jamwidth{\citep[129]{XiaoMcEnery2004}}

\ea
  \ea\label{ex:cry}
    \gll Weici, Deng Lijun shangxin de ku-le san tian.\\
    for.this Deng Lijun sadly \textsc{de} cry-\textsc{pfv} three day\\ \jambox{\citep[129]{XiaoMcEnery2004}}
    \glt `For this reason, Deng Lijun cried sadly for three days.'
  \ex\label{ex:know}
    \gll Ta zhidao-le zhe shi de nei-qing.\\
    he know-\textsc{pfv} this matter \textsc{de} inside-information\\ \jambox{\citep[130]{XiaoMcEnery2004}}
    \glt `He got to know the inside information on this matter.'
  \z
\z
\textit{Le} is compatible with the reduplication, because its dynamicity can relate to not only the termination or instantiation of an event (a point of change), but also the process of the situation, just like that of the reduplication.

In comparison, the experiential aspect marker \textit{guo} cannot co\hyp{}occur with a reduplicated verb, 
because its dynamicity attributes to an ``experiential change'' \citep[148]{XiaoMcEnery2004}, 
namely that a situation has been experienced historically and that ``the final state of the situation no longer obtains'' at the reference time \citep[144]{XiaoMcEnery2004}. 
Compare (\ref{ex:guo-dyn}) and (\ref{ex:le-dyn}), \textit{guo} in (\ref{ex:guo-dyn}) suggests a change out of the state of being a soldier, 
whereas \textit{le} in (\ref{ex:le-dyn}) conveys a change into the state of being a soldier \citep[149]{XiaoMcEnery2004}.

\ea
  \ea\label{ex:guo-dyn}
    \gll ta dang-guo bing.\\
    he serve.as-\textsc{exp} soldier\\ \jambox{\citep[149]{XiaoMcEnery2004}}
    \glt `He once served as a soldier.'
  \ex\label{ex:le-dyn}
    \gll ta dang-le bing.\\
    he serve.as-\textsc{pfv} soldier\\ \jambox{\citep[149]{XiaoMcEnery2004}}
    \glt `He became a soldier.'
  \z
\z
It is clear that \textit{guo} only indicates a change at the termination of a situation and cannot express the dynamicity within a situation.
Hence, it is incompatible with the semantics of the reduplication.

Due to the holistic semantics of the reduplication, it is incompatible with imperfective aspect markers: the durative aspect marker \textit{zhe} and the progressive aspect marker \textit{zai}, as both only focus on  a part of the situation and do not view the situation as a whole \citep[Ch. 5]{XiaoMcEnery2004}.

From the illustration above, it seems that due to its semantics, reduplication can only be marked by \textit{le} but not the other aspect markers.








%%%%%%%%%%%%
\subsection{Word vs. phrase}\label{sec:word}

The literature on reduplication makes different assumptions on whether it is a morphological or syntactic phenomenon.
\citet[Ch. 4]{Chao1968}, \citet[Ch. 3]{LiThompson1981} and \citet[4--5]{Liao2014} listed reduplication under morphological processes. 
By contrast, \citet[23]{Arcodiaetal2014}, \citet{Xiong2016}, \citet[146]{BascianoMelloni2017}, \citet[229--231]{YangWei2017}, \citet[330]{MelloniBasciano2018} and \citet{Xie2020}  claimed it to be syntactic.
%And e.g. \citet{Chen2001}, \citet{Chen2005}  and \citet{Yang2003} did not make any claim about its grammatical status.
This section reviews the arguments in \citet{Xie2020}, applies the tests proposed by \citet{Duanmu1998} and \citet{Schaefer2009} to distinguish words from phrases in Mandarin Chinese 
and compares the behaviors of the reduplication with those of Parallel Verb Compounds 
(compounds which consist of two verbs that ``either are synonymous or signal the same type of predicative notions'' \citep[68]{LiThompson1981}, e.g. \textit{gou-mai} `purchase-buy, buy') and {SVC}s. 
The results argue for a morphological status of reduplication.



\citet{Xie2020} compared the AA and the ABAB forms of reduplication with the AABB form and claimed that AA and ABAB are syntactic processes while AABB is morphological.
She pointed out that AA and ABAB behave differently from AABB in their productivity, possibility of \textit{le} insertion, categorial stability, transitivity, and input/output constraints.
While AA and ABAB are highly productive, AABB shows low productivity. 
\textit{Le} can be inserted freely into AA (\ref{ex:le-insertionAA}) and ABAB (\ref{ex:le-insertionABAB}) but not into AABB (\ref{ex:le-insertionAABB}).

\settowidth\jamwidth{\citep[85]{Xie2020}}

\ea\label{ex:le-insertionAA}
  \ea \gll Yao Ming \textit{kan}-\textit{kan} ta de fanyi Ke Lin...\\
  Yao Ming look-look he \textsc{de} translator Ke Lin\\
  \glt `Yao Ming looked at his translator Ke Lin a little bit...'
  
  \ex \gll Yao Ming \textit{kan}-\textit{le}-\textit{kan} ta de fanyi Ke Lin...\\
  Yao Ming look-\textsc{pfv}-look he \textsc{de} translator Ke Lin\\ \jambox{({CCL})}
  \glt `Yao Ming looked at his translator Ke Lin a little bit...'
  \z
\z

\ea\label{ex:le-insertionABAB}
  \ea \gll ta \textit{heji}-\textit{heji}, dui Jiangqing shuo...\\
  he consider-consider to Jiangqing say\\
 \glt `He considered a little bit, and told Jiangqing...'
 
  \ex \gll ta \textit{heji}-\textit{le}-\textit{heji}, dui Jiangqing shuo...\\
  he consider-\textsc{pfv}-consider to Jiangqing say\\ \jambox{({CCL})}
  \glt `He considered a little bit, and told Jiangqing...'
  \z
\z

\ea\label{ex:le-insertionAABB}
  \ea[]{\gll \textit{yao}-\textit{yao}-\textit{huang}-\textit{huang} jiu ba chulai le.\\
  shake-shake-sway-sway then pull out \textsc{ptc}\\  \jambox{\citep[85]{Xie2020}}
  \glt `Shake it a little bit and then it will be pulled out.'}
  
  \ex[*]{\gll \textit{yao}-\textit{yao}-\textit{le}-\textit{huang}-\textit{huang} jiu ba chulai le.\\
  shake-shake-\textsc{pfv}-sway-sway then pull out \textsc{ptc}\\ \jambox{\citep[85]{Xie2020}}}
  \z
\z

The output of AA and ABAB does not change the grammatical category of the input (verb), but the output of AABB could have other categories such as adverb (\ref{ex:AABBadv}) or adjective (\ref{ex:AABBadj}).

\ea\label{ex:AABBadv}
  \ea[*]{\gll dian-che \textit{yao-huang}-\textit{yao-huang} kai zou...\\
  electric-car shake-sway-shake-sway drive away...\\ \jambox{\citep[86]{Xie2020}}}
  
  \ex[]{\gll dian-che \textit{yao}-\textit{yao}-\textit{huang}-\textit{huang} kai zou...\\
  electric-car shake-shake-sway-sway drive away...\\ \jambox{\citep[86]{Xie2020}}
  \glt `The tram drove away jiggly...'}
  \z
\z

\ea\label{ex:AABBadj}
  \ea[*]{\gll ...zuo zai  \textit{yao-huang}-\textit{yao-huang} de che shang\\
  sit on shake-sway-shake-sway \textsc{de} car on\\ \jambox{\citep[86]{Xie2020}}}
  
  \ex[]{\gll ...zuo zai  \textit{yao}-\textit{yao}-\textit{huang}-\textit{huang} de che shang\\
  sit on shake-shake-sway-sway \textsc{de} car on\\ \jambox{\citep[86]{Xie2020}}
  \glt `...sit on the jiggling car'}
  \z
\z
 
AA and ABAB do not change the valency of the input verb, but AABB makes a transitive verb intransitive (\ref{ex:trans}). 

\ea\label{ex:trans}
  \ea[]{\gll qiao-da gan-jing shi huanjie gan-qi de hao banfa.\\
  knock-beat liver-channel \textsc{cop} relieve liver-\textit{qi} \textsc{de} good method\\ \jambox{\citep[88]{Xie2020}}
  \glt `Beating the liver channel is a good method to relieve the stagnation of liver \textit{qi}.'}
  
  \ex[]{\gll \textit{qiao-da}-\textit{qiao-da} gan-jing shi huanjie gan-qi de hao banfa.\\
  knock-beat-knock-beat liver-channel \textsc{cop} relieve liver-\textit{qi} \textsc{de} good method\\ \jambox{\citep[88]{Xie2020}}
   \glt `Beating the liver channel a little bit is a good method to relieve the stagnation of liver \textit{qi}.'}
   
   \ex[*]{\gll \textit{qiao}-\textit{qiao}-\textit{da}-\textit{da} gan-jing shi huanjie gan-qi de hao banfa.\\
  knock-knock-beat-beat liver-channel \textsc{cop} relieve liver-\textit{qi} \textsc{de} good method\\ \jambox{\citep[88]{Xie2020}}}
  
  \ex[]{\gll \textit{qiao}-\textit{qiao}-\textit{da}-\textit{da} shi huanjie gan-qi de hao banfa.\\
   knock-knock-beat-beat \textsc{cop} relieve liver-\textit{qi} \textsc{de} good method\\
  \glt `Knocking around is a good method to relieve the stagnation of liver \textit{qi}.'}
  \z
\z

The two groups also have different input and output constraints. \citet{Xie2020} claimed that only dynamic and volitional verbs can undergo AA or ABAB reduplication (but see \sectref{sec:Aktionsarten}).
in comparison, AABB requires its input to be a complex verb whose constituents are either synonymous, antonymous or logically coordinated (\ref{ex:AABB-input}). 
Moreover, as can be seen in the translation in (\ref{ex:AABB-input}), the output of AABB has an increasing meaning, i.e. an event happens repeatedly or continuously, as opposed to the delimitative meaning of AA and ABAB.

\ea\label{ex:AABB-input}
  \ea \gll duo-shan $\rightarrow$ \textit{duo}-\textit{duo}-\textit{shan}-\textit{shan}\\
  hide-dodge {} hide-hide-dodge-dodge\\ \jambox{\citep[88]{Xie2020}}
  \glt `hide and dodge' `hide and dodge repeatedly'
  
  \ex \gll jin-chu $\rightarrow$ \textit{jin}-\textit{jin}-\textit{chu}-\textit{chu}\\
  enter-exit {} enter-enter-exit-exit\\ \jambox{\citep[88]{Xie2020}}
  \glt `enter and exit' `enter and exit repeatedly'
  
  \ex \gll shuo-xiao $\rightarrow$ \textit{shuo}-\textit{shuo}-\textit{xiao}-\textit{xiao}\\
  talk-laugh {} talk-talk-laugh-laugh\\ \jambox{\citep[88]{Xie2020}}
  \glt `talk and laugh' `talk and laugh continuously'
  \z
\z

However, a morphological process can be productive, and it does not necessarily change the category or valency of the input.
For instance, the \textit{-bar} derivation in German is a productive morphological process \citep[330]{Mueller2002}. 
The \textit{be-} prefixation in German does not change the category of the input: \textit{arbeiten} `work' $\rightarrow$ \textit{bearbeiten} `handle'. 
And inflections of tense in German do not change the valency of the input verb.
Further, if \textit{le} is considered to be a morphological element (e.g. \citealt[101--102]{Huangetal2009}; \citealt[246]{MuellerLipenkova2013}), the insertion of \textit{le} does not have to be viewed as a syntactic
process either.
It seems that \citet{Xie2020} only showed that AA and ABAB are different processes than AABB, but not necessarily that the former is syntactic while the latter morphological.

It is, therefore, necessary to resort to other tests that are intended to distinguish words from phrases. 
Furthermore, we applied the same tests to Parallel Verb Compounds, which are inarguably words (because at least one element in the compound cannot appear in a sentence independently), 
and {SVC}s, which are by definition phrases. 
We then compared the test results of these three kinds of expressions to see whether the reduplication behaves in a more similar way to Parallel Verb Compounds or to {SVC}s.



\citet{Duanmu1998} and \citet{Schaefer2009} proposed the following four tests to
distinguish words from phrases in Mandarin Chinese: semantic compositionality, phrasal extension,
phrasal substitution and conjunction reduction.\footnote{It is important to note that none of these criteria are sufficient or necessary to determine the word or phrase status of an expression. Nevertheless, they together might suggest which of the two statuses is more likely.}

The semantic criterion is that the meaning of a phrase is usually built up in a compositional way while that of a is word usually not \citetext{\citealp[140]{Duanmu1998}; \citealp[275]{Schaefer2009}}. 
The meaning of the reduplication is not compositional, as it does not mean that the event denoted by the verb happens twice or multiple times, but rather that the event happens for a short duration and/or for low frequency.
This non-compositionality suggests that a reduplication is more word-like.


The syntactic criteria are all based on the Lexical Integrity Hypothesis according to \citet[60]{Huang1984}, as shown in (\ref{lih}):

\ea\label{lih}The Lexical Integrity Hypothesis\\
No phrase-level rule may affect a proper subpart of a word.
\z

The first syntactic test is phrasal extension, namely the addition of optional elements \citetext{\citealp[150]{Duanmu1998}; \citealp[280]{Schaefer2009}}. 
Optional elements that can possibly appear in a phrase should be able to be added into it,
and subparts of a phrase should be able to be modified separately. 
If the unit is a word, however, then neither of these should  be possible.
As illustrated in (\ref{ex:forms-mono})--(\ref{ex:forms-vo}) in \sectref{sec:forms}, the reduplication can only be separated by \textit{le} and \textit{yi}. 
As mentioned above, whether aspect markers are considered to be morphological or syntactic elements depends on the theoretical framework (and possibly the target language).\footnote{For example, \citet[Sec. 3.3.1]{Huangetal2009} and \citet[246]{MuellerLipenkova2013} considered the postverbal aspect markers  in Mandarin Chinese (\textit{le}, \textit{zhe} and \textit{guo}) to be morphological elements. E.g. \citet{Travis2000}, \citet{Ramchand2008},  \citet[23]{Arcodiaetal2014}, \citet[146]{BascianoMelloni2017} considered all aspect information to be syntactically encoded.} 
And the status of \textit{yi} is unclear, since it does not carry any additional meaning or grammatical function in the structure. % (more on this topic see Section \ref{sec:jackendoff}).
Turning to separate modification,  the element in the reduplication cannot be modified individually. 
Compared to (\ref{ex:ext-redup1}), where the adverbial \textit{qingsheng de} `quietly' modifies the whole reduplication, (\ref{ex:ext-redup2}) is ungrammatical, 
as the adverbial cannot modify the second element in the reduplication alone.
It is worth mentioning that in (\ref{ex:ext-redup1}), it is not possible to analyze \textit{qingsheng de} `quietly' to modify only the first element in the reduplication, 
because the adverbial does not necessarily occur adjacent to the verb, as in (\ref{ex:ext-redup3}), 
and it still modifies the whole predicate.
(\ref{ex:ext-redup1}) and  (\ref{ex:ext-redup3}) have the same meaning.

\ea
  \ea[]{\label{ex:ext-redup1}
  \gll ta dui ziji qingsheng de \textit{xiao-le-xiao}.\\
  he to \textsc{refl} quietly \textsc{de} laugh-\textsc{le}-laugh\\
  \glt `He quietly laughed a little bit to himself.'}
  
  \ex[*]{\label{ex:ext-redup2}
  \gll ta dui ziji \textit{xiao}-\textit{le} qingsheng de \textit{xiao}.\\
  he to \textsc{refl} laugh-\textsc{pfv} quietly \textsc{de} laugh\\}
  
  \ex[]{\label{ex:ext-redup3}
  \gll ta qingsheng de dui ziji \textit{xiao-le-xiao}.\\
  he quietly \textsc{de} to \textsc{refl} laugh-\textsc{pfv}-laugh\\
  \glt `He quietly laughed a little bit to himself.'}
  \z
\z

Nothing can intervene between a Parallel Verb Compound and its subparts cannot be modified separately either (\ref{ex:ext-pvc}).

\ea\label{ex:ext-pvc}
  \ea[]{
  \gll ta gou-mai-le yi jian dayi.\\
  he purchase-buy-\textsc{pfv} one \textsc{clf} coat\\
  \glt `He bought a coat.'}
  
  \ex[*]{
  \gll ta gou-le-mai yi jian dayi.\\
  he purchase-\textsc{pfv}-buy one \textsc{clf} coat\\}
  
  \ex[*]{
  \gll ta gou-le-yi-mai yi jian dayi.\\
  he purchase-\textsc{pfv}-one-buy one \textsc{clf} coat\\}
  
  \ex[]{
  \gll ta kaixin de gou-mai-le yi jian dayi.\\
  he happily \textsc{de} purchase-buy-\textsc{pfv} one \textsc{clf} coat\\
  \glt `He bought a coat happily.'}
  
  \ex[*]{
  \gll ta gou kaixin de mai-le yi jian dayi.\\
  he purchase happily \textsc{de} buy-\textsc{pfv} one \textsc{clf} coat\\}
  \z
\z

But an {SVC} readily allows other elements to appear between the two verbs (in (\ref{ex:ext-svc1}), the direct object of the first verb) and each verb can be modified separately (\ref{ex:ext-svc2}), as well.

\ea
  \ea\label{ex:ext-svc1}
  \gll ta zhong cai mai.\\
  he plant vegetables sell\\
  \glt `He plants vegetables and sells them.
  
  \ex\label{ex:ext-svc2}
  \gll ta xinku de zhong cai qinfen de mai.\\
  he hardworkingly \textsc{de} plant vegetables diligently \textsc{de} sell\\
  \glt `He plants vegetables hard-workingly and sells them diligently.'
  \z
\z
All in all, by the test of phrasal extension, reduplications behave more like words than like phrases.



The second syntactic test is phrasal substitution, namely the substitution of smaller exemplars of a specific category with a full blown XP \citetext{\citealp[152]{Duanmu1998}; \citealp[280]{Schaefer2009}}. 
If a part of an expression is actually an XP that only contains one element, a full realization of this XP should be possible as well.
Otherwise, this expression is considered to be a word.
In a reduplication structure, it is ungrammatical to substitute each element with a full {VP} (\ref{ex:sub-redup}).

\ea\label{ex:sub-redup}
  \ea[]{\label{ex-he-tasted-the-soup-a-little-bit}
  \gll ta \textit{chang}-\textit{le}-\textit{chang} tang.\\
    he taste-\textsc{pfv}-taste soup\\
    \glt `He tasted the soup a little bit.'}
    
    \ex[*]{
    \gll ta \textit{chang} \textit{tang} \textit{le} \textit{chang} \textit{tang}.\\
    he taste soup \textsc{pfv} taste soup\\}
  \z
\z

It is the same case with Parallel Verb Compounds (\ref{ex:sub-pvc}).

\ea[*]{
\gll ta gou yi jian dayi mai le yi tiao kuzi.\\
he purchase one \textsc{clf} coat buy \textsc{pfv} one \textsc{clf} pants\\}
\label{ex:sub-pvc}
\z

However, it is grammatical to replace a reduced part of a {SVC} with a full VP (\ref{ex:sub-svc}).

\ea\label{ex:sub-svc}
\gll ta zhong cai mai (cai).\\
 he plant vegetables sell vegetables\\
\glt `He plants vegetables and sells vegetables.'
 \z
Again, reduplications look more word-like than phrase-like.


The third syntactic criterion is conjunction reduction. It should only be possible for coordinated phrases (\ref{ex:co-phrase}) and not for coordinated words (\ref{ex:co-word}) \citetext{\citealp[137]{Duanmu1998}; \citealp[283]{Schaefer2009}}.

\settowidth\jamwidth{\citep[137]{Duanmu1998}}

\ea\label{ex:co-phrase}
  \ea \gll [jiu de shu] gen [xin de shu]\\
  old \textsc{de} book and new \textsc{de} book\\ \jambox{\citep[137]{Duanmu1998}}
  \glt `old books and new books'
  
  \ex \gll [jiu de gen xin de] shu\\
  old \textsc{de} and new \textsc{de} book\\ \jambox{\citep[137]{Duanmu1998}}
  \glt `old and new books'
   \z
\z

\ea\label{ex:co-word}
  \ea[]{
  \gll [huo-che] gen [qi-che]\\
  fire-car and gas-car\\ \jambox{\citep[137]{Duanmu1998}}
  \glt `train and automobile'}
  
  \ex[*]{
  \gll [huo gen qi] che\\
  fire and gas car\\ \jambox{\citep[137]{Duanmu1998}}}
  \z
\z

For the reduplication, conjunction reduction does not seem to be possible. 
In (\ref{ex:co-redup1}), the reduplication \textit{jiao-jiao} `chew a little bit' is coordinated with a simple verb \textit{mo} `apply' together with the adverbial \textit{yidian} `a little bit'.
Without the adverbial \textit{yidian} `a little bit', \textit{mo} `apply' by itself cannot express the additional `a little bit' meaning even when it is coordinated with a reduplicated verb.
Similarly, in (\ref{ex:co-redup2}), the reduplication \textit{kan-le-kan} `look a little bit' is coordinated with the predicate \textit{zou-le chulai} `walked out'. The verb in the latter case is not reduplicated and it cannot express the delimitative meaning either.

\settowidth\jamwidth{(CCL)}

\ea
  \ea\label{ex:co-redup1}
  \gll wujian gong-xiu mo dian bohe-gao huo \textit{jiao}-\textit{jiao} kouxiangtang.\\
    midday work-break apply a.little mint-cream or chew-chew chewing.gum\\ \jambox{({CCL})}
    \glt `During the working break in the midday, apply a little bit of mint cream or chew some chewing gum a little bit.'
    
   \ex\label{ex:co-redup2}
   \gll Song Ailing \textit{kan}-\textit{le}-\textit{kan} yupen you zou-le chulai.\\
   Song Ailing look-\textsc{pfv}-look bath.tub again walk-\textsc{pfv} out\\  \jambox{({CCL})}
   \glt `Song Ailing looked at the bath tub a little bit and walked out again.'
   \z
\z

Conjunction reduction is not possible for Parallel Verb Compounds as well.
We can see in (\ref{ex:co-pvc}) that for the Parallel Verb Compounds \textit{gou-mai} `purchase-buy, buy' and \textit{gou\hyp{}zhi} `purchase\hyp{}place, purchase', it is neither possible to delete the first verb nor the second.

\ea\label{ex:co-pvc}
     \ea[*]{
     \gll ta gou-mai-le yi jian dayi hai gou-le yi tiao kuzi.\\
    he purchase-buy-\textsc{pfv} one \textsc{clf} coat and purchase-\textsc{pfv} one \textsc{clf} pants\\
    \glt Intended: `He bought a coat and purchased a pair of pants.'}
    
    \ex[*]{
    \gll ta gou-mai-le yi jian dayi hai -zhi-le yi tiao kuzi.\\
    he purchase-buy-\textsc{pfv} one \textsc{clf} coat and place-\textsc{pfv} one \textsc{clf} pants\\
   \glt  Intended: `He bought a coat and purchased a pair of pants.'}
    \z
\z

By contrast, it is grammatical to delete a repeated part in a coordinated {SVC}. 
In (\ref{ex:co-svc1}), while it is possible for the object of selling to be only cows, it is equally acceptable to interpret it as the person sells both vegetables and cows. With the second interpretation, (\ref{ex:co-svc1}) can be understood as a reduced version of (\ref{ex:co-svc2}).

\ea
  \ea\label{ex:co-svc1}
  \gll ta zhong cai hai yang niu mai.\\
  he plant vegetables and farm cow sell\\
  \glt `He plants vegetables and farms cows and sells them.'
  
  \ex\label{ex:co-svc2}
  \gll ta zhong cai mai hai yang niu mai.\\
  he plant vegetables sell and farm cow sell\\
  \glt `He plants vegetables and sells them and farms cows and sells them.'
   \z
\z
Once again, this criterion suggests that reduplications do not have this expected property of phrases.


Following the analyses above, it is clear that the reduplication failed all of the tests for phrasal status. 
In comparison, Parallel Verb Compounds failed all the tests as well, whereas {SVC}s passed all of them.
This makes the behavior of the reduplication seem more similar to that of a Parallel Verb Compound than that of an {SVC}.
Therefore, it seems more appropriate to assume reduplication to be a morphological process rather than a syntactic one. 





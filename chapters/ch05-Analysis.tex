\section{A new HPSG analysis}\label{sec:analysis}

In this section, we suggest a new lexical-rule-based analysis of aspect marking and reduplication
using Minimal Recursion Semantics (MRS; \citealt{Copestakeetal2005}). %, as proposed in \citet{LuMueller2021}.
MRS uses lists of elementary predications that are connected via pointers. Scope constraints are
represented by statements of domination. This allows for elegant ways to underspecify scope. The
details cannot be discussed here. The interested reader is referred to \citet{Copestakeetal2005}. In
what follows, we will present the elementary predications with the features assumed in MRS, but leave
out handle constraints to keep things simple.

Like \citet{FanSongBond2015}, we assume lexical rules for reduplication. Our lexical rules are
organized in an inheritance network. \textit{verbal-reduplication-lr} is the most general type for
reduplication lexical rules in this network and the implicational constraint in (\ref{avm:redup}) shows the constraints on all
structures of type \textit{verbal-reduplication-lr}. Such structures take a verb as
\textsc{lexical-daughter (lex-dtr)}.  The output reduplicates the \textsc{phonology (phon)} of the
input verb with the possibility to have further phonological material in between.  \etag{} indicates
an underspecified list which could be empty or not.\footnote{%
The \etag is equivalent to a tag with a number with the difference that \etag is never shared. We
follow Müller (\citeyear[161]{Mueller2002b}, \citeyear[294]{Mueller2003e}) and elsewhere in using this
notation, since we think this is more
precise than simply using three dots.}
A delimitative relation is appended to the
\textsc{relations (rels)} value of the input verb, and it takes the event index of the input verb \iboxb{3} as
argument (\textsc{arg0}, \ldots, \textsc{arg3} are feature names for arguments. The values are
indices or events similar to variables in normal predicate logic.). The list of relations contains so-called elementary predications. There is no complicated
embedding of relations. Instead each elementary predication comes with a label (\textsc{lbl}). The
label can be used as an argument of another relation or in scope constraints, which are not provided
here to keep things simple. The feature \textsc{ltop} points to the local top. This is the
elementary predication that is considered the top-most one in the \relsl. Other elementary
predications may share the label or have arguments with labels of other elementary predications. We
will discuss an example below when we discuss the perfective lexical rule.
The label of the output \iboxb{2} is identified with the label of the input and with the
label of the delimitative relation, hence \type{delimitative-rel} is treated as a modifier.  Further
relations can be added at the beginning of the \textsc{rels} list to allow for the additional
perfective meaning in A-\textit{le}-A and A-\textit{le}-\textit{yi}-A.  The combination with the
perfective will be elaborated on in the following paragraphs.

\eas
\type{verbal-reduplication-lr} \impl\\
%\scalebox{.9}
\label{avm:redup}
\zs

To account for the variations in the phonology of the reduplication as well as 
the combination with the phonology and semantics of the perfective aspect marker \textit{le}, 
the type hierarchy of lexical rules in \figref{fig:typehi} is put forward. 
\begin{figure}
    \centering
    \scalebox{.9}{\begin{forest}
        type hierarchy
        [,phantom
        [,phantom
        [verbal-reduplication-lr,name=verbal-reduplicaiton-lr
        [non-perfective-reduplicaiton-lr
        [a-a-lr]
        [a-yi-a-lr]]
        [,identify=!r211]]]
        [aspect-marking-lr
        [perfective-lr
        [perfective-reduplicaiton-lr%,edge to=verbal-reduplicaiton-lr
        [a-le-yi-a-lr]
        [a-le-a-lr]]
        [v-le-lr]]
        [durative-lr]
        [\ldots]]]
    \end{forest}
}
    \caption{Type hierarchy for lexical rules of verbal reduplication and \textit{le}}
    \label{fig:typehi}
\end{figure}
Apart from the type \type{perfective-reduplication-lr}, 
which adds the inherited perfective relation, 
there is a subtype \type{non-perfective-reduplication-lr}, 
which does not add further relations. 
Hence, what is \etag in the \textsc{rels} list in (\ref{avm:redup}) is the empty list in (\ref{avm:nonperf}):
\ea\label{avm:nonperf}
\type{non-perfective-verbal-reduplication-lr} \impl\\
%\scalebox{.9}
\z
The \textsc{rels} list of the output of the lexical rule is the \textsc{rels} list of
the daughter \iboxb{1} plus a list with one element. 
Since this element is specified in the supertype, it is not specified in (\ref{avm:nonperf}) again. 

\type{non-perfective-verbal-reduplication-lr} has \type{aa-lr} and \type{a-yi-a-lr} as direct subtypes.
(\ref{avm:AA}) and (\ref{avm:AyiA}) show \type{aa-lr} and \type{a-yi-a-lr}, respectively.
As subtypes of \type{ver\-bal\hyp{}re\-dup\-li\-ca\-tion\hyp{}lr} illustrated in (\ref{avm:redup}), 
both inherit the constraints on the \textsc{lex-dtr} and on the semantics of the output, 
and because of (\ref{avm:nonperf}), no extra material is appended to the \textsc{rels} value of the input verb
and the list containing the \type{delimitative-rel}.
 In addition to the inherited constraints,
\type{aa-lr} and \type{a-yi-a-lr} specify the phonology of the output differently.
\type{aa-lr} determines that the \etag between the two phonological copies in (\ref{avm:redup}) is the empty list, 
whereas \type{a-yi-a-lr} specifies this list of phonological material as \phonliste{ yi }:

\ea\label{avm:constr-AA}
Constraints on lexical rules of type \type{aa-lr} and \type{a-yi-a-lr}:
%\begin{tabular}{@{}l@{\hspace{2cm}}l@{}}
 \ea   \type{aa-lr} \impl \\%& 
  %  \scalebox{.9} %& 
\ex	\type{a-yi-a-lr} \impl\\
 %    \scalebox{.9}
%\end{tabular}
\z
\z


\noindent
The lexical rules with all inherited constraints are given in (\ref{avm:AA}) and (\ref{avm:AyiA}):

\ea\label{avm:AA}
The AA lexical rule with all constraints inherited from the supertypes:\\
%\scalebox{.9}
\z

%\newpage
\ea\label{avm:AyiA}
The A-\textit{yi}-A lexical rule with all constraints inherited from the supertypes:\\
%\scalebox{.9}
\z

\type{v-le-lr} is a direct subtype of the \type{perfective-lr}.
\type{perfective-reduplication-lr} inherits from both \type{verbal-reduplication-lr} and \type{per\-fec\-tive\hyp{}lr}
and has two subtypes, \type{a-le-yi-a-lr} and \type{a-le-a-lr} itself.
\type{verbal-reduplication-lr} is already presented in (\ref{avm:redup}). 
We now turn to the constraints on \type{perfective-lr} and its subtypes.


\citet[246]{MuellerLipenkova2013} propose the perfective lexical rule given in (\ref{avm:pfv-old}), 
adapted to the formalization adopted in the current paper.
It takes a verb as \textsc{lex-dtr}
and appends \phonliste{ le } to its phonology.
Further, it accounts for the change in semantics by appending the \textsc{rels} value of the input verb to a \type{perfective-rel}.

\ea\label{avm:pfv-old}
Perfective lexical rule adapted from \citet[246]{MuellerLipenkova2013}:\\
%\scalebox{.9}
\z
The event variables \iboxb{3} of the input and the output verb are shared. 
The \textsc{ltop} of the output of the lexical rule \iboxb{2} is the label of the perfective relation, 
and this relation scopes over the embedded verb. 
The handle of the embedded verb \iboxb{4} is the argument of the \type{perfective-rel}. 

The lexical rule suggested in (\ref{avm:pfv-old}) only explains simple perfective aspect marking with \textit{le}, 
where \textit{le} immediately follows the verb.
But it cannot account for the perfective aspect marking of a reduplicated verb,
 as \textit{le} does not occur after the reduplication, 
nor can \textit{le} be reduplicated together with the verb.
It can only appear between the verb and the reduplicant.
In order to accommodate \textit{le} marking for both simple and reduplicated verbs, 
a general perfective lexical rule as in (\ref{avm:pfv-new}) and 
a subtype \type{v-le-lr} as in (\ref{avm:vle}) are posited here.
Besides adding a \type{perfective-rel} in the \textsc{rels} list of the output as in (\ref{avm:pfv-old}), 
the \type{perfective-lr} in (\ref{avm:pfv-new}) allows an underspecified list to be appended at the end of the \textsc{rels} list.
The \textsc{phon} value of the output makes it possible for further phonological material to occur both before and after \phonliste{ le }.

\ea\label{avm:pfv-new}
Type constraints on the type \type{perfective-lr} from which other subtypes inherit:\\
%\scalebox{.9}{%
    \avm{
        [\type*{perfective-lr}\\
        phon & \etag $\oplus$ \phonliste{ le } $\oplus$ \etag\\
        \punk{synsem\textbar loc \textbar cont}{ [ltop & \2\\
            ind  & \3]}\\ 
        rels & <[\type*{perfective-rel}\\
        lbl  & \2\\
        arg0 & \3\\
        arg1 & \4 ]> $\oplus$ \5 $\oplus$ \etag\smallskip\\
        lex-dtr & [%phon & \1\\
        \punk{synsem\textbar loc}{[cat & [ head & verb ]\\
            cont & [ ltop & \4\\
            ind  & \3]]}\\
        rels & \5]
        ]
}
%}
\z

\type{v-le-lr} with all inherited constraints as given in (\ref{avm:vle}) inherits from \type{perfective-lr} and specifies that the first element in the output \textsc{phon} list is identified with the \textsc{phon} value of the input verb
and that nothing else comes after \phonliste{ le }.
Furthermore, no other list can be appended at the end of the \textsc{rels} list of the output anymore.
This corresponds to the proposal of \citet[246]{MuellerLipenkova2013} shown in (\ref{avm:pfv-old}), 
which accounts for the simple perfective marking of verbs.

%\newpage
\ea\label{avm:vle}
Structure of type \type{v-le-lr} with constraints inherited from \type{perfective-lr}:\\
%\scalebox{.9}
\z

\type{perfective-reduplication-lr} inherits from both \type{verbal-reduplication-lr} and \type{per\-fec\-tive\hyp{}lr}.
The \textsc{phon} value of the output reduplicates the phonology of the input verb and states that there is \phonliste{ le } in between, 
as well as potentially further phonological material.
The \textsc{rels} list of the output appends the \type{delimitative-rel} to the \type{perfective-rel} and the \textsc{rels} value of the input verb.
The arguments of both \type{perfective-rel} and \type{delimitative-rel} share the event index of the
input verb \iboxb{3} to ensure that they apply to the same event denoted by the input verb. 
The label of the \type{delimitative-rel} and the input verb are identified (\type{delimitative-rel} is a modifier) 
and this shared label is embedded under the \type{perfective-rel}.

\ea\label{avm:pfv-redup}
Perfective and reduplication combined: type \type{perfective-reduplication-lr} with
constraints inherited from \type{perfective-lr} and \type{verbal-redu\-pli\-ca\-tion-lr}:\\
%\resizebox{\linewidth}{!}{
 %   \scalebox{.9}{%
        \avm{
            [\type*{perfective-reduplication-lr}\\
            phon & \1 $\oplus$ \phonliste{ le } $\oplus$ \etag $\oplus$ \1\\
            \punk{synsem\textbar loc\textbar cont\textbar ltop}{ \2}\\ 
            rels & <[\type*{perfective-rel}\\
            lbl & \2\\
            arg0 & \3\\
            arg1 & \4]>
            $\oplus$ \5 $\oplus$
            <[\type*{delimitative-rel}\\
            lbl  & \4\\
            arg0 & \3]>\\
            lex-dtr & [phon & \1\\
            synsem\textbar loc & [cat  & [ head & verb ]\\
            cont & [ ltop & \4\\
            ind  & \3]]\\
            rels & \5]
            ]
    }
%}
    %}
\z
For example %(\ref{ex-he-tasted-the-soup-a-little-bit}), repeated here as
 (\ref{ex-he-tasted-the-soup-a-little-bit2}), we get the  MRS representation in (\ref{ex:mrs-soup}),
where h1 and h2 correspond to the handles \ibox{2} and \ibox{4} and e1 to the event variable \ibox{3}:
\ea\label{ex-he-tasted-the-soup-a-little-bit2}
    \gll \objex{ta1} \obj{chang2}-\obj{le}-\obj{chang2} \objex{tang1}.\\
    he taste-\textsc{pfv}-taste soup\\
    \glt `He tasted the soup a little bit.'
 \z
\ea\label{ex:mrs-soup}
h1 \sliste{ h1:perfective(e1,h2), h2:taste(e1,he,soup), h2:delimitative(e1) }
\z
So the delimitative relation is treated as an adjunct to the main relation of the verb, 
and the perfective relation scopes over both the main relation and the delimitative relation.

Two subtypes of \type{perfective-reduplication-lr} are posited:
\type{a\hyp{}le\hyp{}yi\hyp{}a\hyp{}lr} %as in (\ref{avm:AleyiA}) 
and \type{a-le-a-lr}, as shown in (\ref{avm:AleA}). %(\ref{avm:AleA}).
They take over the semantic change to the input from \type{perfective\hyp{}reduplication\hyp{}lr}, but specify the \textsc{phon} value differently.
Specifically, \type{a\hyp{}le\hyp{}yi\hyp{}a\hyp{}lr} specifies the middle phonological material as
\phonliste{ le, yi }, while \type{a-le-a} specifies it as \phonliste{ le } only.\footnote{
A reviewer wants to know what prevents phonological material other than \obj{yi} and \obj{le} to appear in between the reduplicated elements.
He/she states that this aspect seems to be somewhat arbitrary, and it is not evident why certain elements are permissible while others are not.

\obj{le} `\textsc{pfv}' marks the perfective aspect.
The reason for \obj{le} `\textsc{pfv}' being the only aspect marker compatible with reduplication is explained in Section~\ref{sec:aspM}.

As for \obj{yi} `one', its use stems from the historical development of delimitative verbal reduplication but is synchronically opac.
\citet[13--15]{Zhang2000} shows that delimitative verbal reduplication originates from the ``V + numeral + verbal classifier'' phrase,
where the verbal classifier is borrowed from a verb of the same form, as shown in (\ref{ex:hist1}) from \obj{Song4dai4} \obj{juan4}: \obj{Xu1tang2} \obj{He2shang} \obj{yu3lu4} \textit{[The Song dynasty volume: Quotations from Abbot Xutang]} p. 387 as cited in \citet[12]{Zhang2000}.
\ea\label{ex:hist1}
%明又喝,岐亦喝,明连喝两喝,岐便礼拜。(《宋代卷·虚堂和尚语录》3 87 页) as cited in \citet[12]{Zhang2000}
\gll \objex{ming2} \objex{you4} \objex{he4}, \objex{qi2} \objex{yi4} \objex{he4}, \objex{ming2} \objex{lian2} \objex{he4} \objex{liang3} \objex{he4}, \objex{qi2} \objex{bian4} \objex{li3bai4}\\
Ming again yell Qi also yell Ming in.succession yell two yell Qi thus bow\\
\glt `Ming yelled again. Qi yelled too. Ming yelled twice. Qi thus bowed.'
\z
Here the numeral refers to the actual number of the action taking place
and can be any number.
This evolved into the structure of V-\obj{yi}-V (or A-\obj{yi}-A), which does not express the actual number of the action anymore,
but simply conveys that the action happens in a short period of time or for only few times. 
Consider (\ref{ex:hist2}) from \obj{Song4dai4} \obj{juan4}: \obj{Zhang1} \obj{Xie2} \obj{zhuang4yuan2} \textit{[The Song dynasty volume: Top graduate Zhang Xie]} p. 519 as cited in \citet[13]{Zhang2000}.
\ea\label{ex:hist2}
%且歇一歇了,去坐地。(《宋代卷·张协状元》5 19 页) as cited in \citet[13]{Zhang2000}
\gll \objex{qie3} \obj{xie1-yi-xie1} \objex{le}, \objex{qu4} \objex{zuo4} \objex{di4}.\\
just rest-one-rest \textsc{ptc} go sit ground\\
\glt `(Let me) just take a rest and go sit on the ground.'
\z
In this case, \obj{yi1} `one' cannot be replaced by other numerals.
Since V-\obj{yi}-V does not express the actual number of the action anymore,
it became possible to delete the \obj{yi1} `one',
hence the use of the VV (or AA) form, too.

Synchronically, A-\obj{yi}-A has the same meaning as AA and both do not refer to the actual number of the event denoted by the verb.
This means that \obj{yi1} `one' in A-\obj{yi}-A does not contribute any specific meaning to the structure
and is only there as a historical remnant.
In this sense,  \obj{yi1} `one' appearing in between the reduplication is synchronically arbitrary. %similar to linking elements in compounds

Besides the obvious conceptual reason that  \obj{yi1} `one' is taken to mean `little/few',
there might be a phonological explanation for  \obj{yi1} `one' rather than other numerals is used in the A-\obj{yi}-A structure.
In Mandarin Chinese, one form of intensifying reduplication is A-\obj{li}-AB e.g. \obj{hu2-li-hu2tu2} `confused-li-confused',
where the \obj{li} is fixed and also does not bear any meaning.
\citet[137]{Sui2018} assumes that this syllable is filled with \obj{li} because the open syllable \obj{li} is a relatively unmarked phonological constituent \citep{Yip1992}
and the second syllable in A-\obj{li}-AB occupies an unstressed position.
We can see the similarities between \obj{li} in A-\obj{li}-AB and \obj{yi} in A-\obj{yi}-A: they are both relatively unmarked and occupies an unstressed position.
This can also make it easier for  \obj{yi1} `one' rather than other numerals to become part of a fixed structure.
But a phonological account is out of the scope of this paper and has to be left for further research.

The subtypes of \type{verbal-reduplication-lr} (\ref{avm:AyiA}, \ref{avm:AleA}) prevent phonological materials other than \phonliste{ le } and \phonliste{ yi } from appearing in between the reduplication by specifying what can appear in the \etag list and what not.
}


\ea\label{avm:AleA}
%\begin{tabular}{@{}l@{\hspace{2cm}}l@{}}
\ea    %\scalebox{.9}{%
        \type{a-le-yi-a-lr} \impl\\
        \avm{
            [%\type*{a-le-yi-a-lr}\\
            phon & \1 \+ \phonliste{ le, yi } \+ \1\\
            lex-dtr & [phon & \1
            ]
            ]
    }
%}%&
\ex   % \scalebox{.9}{%
\label{avm:a-le-a-lr}
    \type{a-le-a-lr} \impl\\
        \avm{
            [%\type*{a-le-a-lr}\\
            phon & \1 \+ \phonliste{ le } \+ \1\\
            lex-dtr & [phon & \1
            ]
            ]
    }
%}
%\end{tabular}
\z
\z

The analysis of (\ref{ex-he-tasted-the-soup-a-little-bit2}) is shown in tree format in
Figure~\ref{fig-he-tasted-the-soup-a-little-bit}. The \textsc{lex-dtr} is the daughter in the tree.
\begin{figure}[htbp]
\centering
\begin{forest}
[        \avm{
            [\type*{a-le-a-lr}\\
            phon & \1 $\oplus$ \phonliste{ le } $\oplus$ \1\\
            synsem\textbar loc & [ cat  & [ head  & verb\\
                                            spr   & < NP$_{\6}$ >\\
                                            comps & < NP$_{\7}$ > ]\\
                                   cont & [ ltop & \2\\
                                            ind  & \ibox{3} ]] \\
            rels & <[\type*{perfective-rel}\\
            lbl & \2\\
            arg0 & \3\\
            arg1 & \4]> $\oplus$ \5 $\oplus$
            <[\type*{delimitative-rel}\\
            lbl  & \4\\
            arg0 & \3]> ]
    }
[\avm{ [phon & \1 \phonliste{ chang2 }\\
       synsem\textbar loc & [cat  & [ head  & verb\\
                                      spr   & < NP$_{\6}$ >\\
                                      comps & < NP$_{\7}$ > ]\\
                             cont & [ ltop & \4\\
                                      ind  & \3]]\\
        rels & \5 < [\type{taste}\\
                     lbl  & \4\\
                     arg0 & \3\\
                     arg1 & \6\\
                     arg2 & \7 ] > ]
}]]
\end{forest}

\caption{Analysis of \obj{chang2-le-chang2} taste-\textsc{pfv}-taste `taste the soup a little bit'}\label{fig-he-tasted-the-soup-a-little-bit}
\end{figure}

The figure shows how the lexical item for \obj{chang2} `to taste' is inserted as a daughter into
the \type{a-le-a-lr} lexical rule. The arguments of \obj{chang2} `to taste' are represented within the \spr and
the \compsl (see \citealt{GSag2000a-u} for English) and the respective argument NPs are linked to the
arguments of \type{taste}: the subject is \textsc{arg1}, the agent, and the object is \textsc{arg2},
the stimulus. See \citet{DKW2021a} for more on linking. The part of speech of the lexical item
(\type{verb}) and the valence information is carried over from the daughter to the mother
unchanged. The semantic contribution of the daughter verb, the value of \ibox{5}, is inserted into
the \relsl of the mother as it was specified in the constraints on the type
\type{perfective-reduplication-lr} in (\ref{avm:pfv-redup}). The \textsc{phon} value of the mother
is the concatentation of \obj{chang2}, \obj{le} and \obj{chang2}, as specified in the constraint
on \type{a-le-a-lr} in (\ref{avm:a-le-a-lr}). The resulting unit \obj{chang2-le-chang2} then
combines with its object forming a VP. This VP is combined with the subject resulting in a complete
verbal projection, a sentence. \obj{chang2-le-chang2} behaves in the same way as the simple \obj{chang2}.

Since the above-described lexical rules do not constrain the number of syllables of the input verb, but simply reduplicate its phonology  as a whole,
they can also account for the ABAB and the AB-\textit{le}-AB forms of reduplication,
as long as the input verb is disyllabic.
Notice that  the lexical rules above also produce AB-\textit{yi}-AB and AB\hyp{}\textit{le}\hyp{}\textit{yi}\hyp{}AB for disyllabic input verbs.
Although these forms are considered unacceptable by some authors \citetext{\citealp[30]{LiThompson1981}; \citealp[275--276]{Hong1999};  \citealp[160]{BascianoMelloni2017}; \citealp[239]{YangWei2017}}, 
\citet[269]{Fan1964} and \citet[143]{Sui2018} consider AB-\textit{yi}-AB and AB-\textit{le}-\textit{yi}-AB to be possible, even though they both recognize that these two forms are rare.
Indeed, a few examples of AB\hyp{}\emph{yi}\hyp{}AB and AB\hyp{}\emph{le}\hyp{}\emph{yi}\hyp{}AB in Early Mandarin %(12th to 20th centuries)
 (\ref{ex:AByiAB-lu}--\subref{ex:AByiAB-pu}) and Modern Mandarin (\ref{ex:AByiAB-rou1}--\subref{ex:ABleyiAB-ccl}) were found.

\settowidth\jamwidth{(CCL)}

\begin{sloppypar}
\ea\label{ex:ABleyiAB}
\ea\label{ex:AByiAB-lu}
\gll \objex{ni3} \objex{yu3} \objex{wo3} \obj{zheng3li3-yi-zheng3li3}.\footnotemark\\
you let me arrange-one-arrange\\
\glt `Let me arrange it a little bit!'
\footnotetext{\obj{Yuan2qu3} \obj{xuan3}: \obj{Lu3zhai1lang2} \textit{[Selected Yuanqu: Luzhailang]}, as cited in \citet[15]{Zhang2000}}

\ex\label{ex:AByiAB-pu}
\gll \objex{ni3} \obj{da3ting1-yi-da3ting1}.\footnotemark\\
you inquire-one-inquire\\
\glt `Inquire about it a little bit!'
\footnotetext{\obj{Yuan2} \obj{Ming2} \obj{juan4}: \obj{Piao2tong1shi4} \textit{[Yuan and Ming volume: Piaotongshi]}, 308, as cited in \citet[15]{Zhang2000}}

\ex\label{ex:AByiAB-rou1}
\gll \objex{ge4} \objex{ge4} \objex{dian3-tou2} \obj{wei1xiao4-yi-wei1xiao4}.\footnotemark\\
\textsc{clf} \textsc{clf} nod-head smile-one-smile\\
\glt `Each one nodded his head and smiled a little bit.'
\footnotetext{\objex{Rou2}, \objex{Shi2}. 1975. \obj{Rou2} \obj{Shi2} \obj{xiao3shuo1} \obj{xuan3ji2} \textit{[Selected novels of Roushi]}, 31. Beijing: People's Literature Publishing House.}

\ex\label{ex:AByiAB-rou2}
\gll \objex{ta1} \obj{wei1xiao4-le-yi-wei1xiao4}, you \obj{ming2xiang3}-\obj{le-yi}-\obj{ming2xiang3}.\footnotemark\\
he smile-\textsc{pfv}-one-smile and meditate-\textsc{pfv}-one-meditate\\
\glt `He smiled a little bit and meditated a little bit.'
\footnotetext{\objex{Rou2}, \objex{Shi2}. 1975. \obj{Rou2} \obj{Shi2} \obj{xiao3shuo1} \obj{xuan3ji2} \textit{[Selected novels of Roushi]}, 31. Beijing: People's Literature Publishing House.}

\ex\label{ex:AByiAB-li}
\gll \objex{fei1chang2} \objex{yan2su4} de \objex{ba3} \objex{jin4shi4} \objex{yan3jing4} \obj{duan1zheng4-le-yi-duan1zheng4}.\footnotemark\\
very seriously \textsc{de} \textsc{ba} nearsighted glasses straighten-\textsc{pfv}-one-straighten\\
\glt `[He] very seriously straightened the nearsighted glasses quickly.'
\footnotetext{\objex{Li3}, \objex{Jie2ren2}. 1962. \obj{Da4} \obj{bo1} \textit{[Great wave]}, 3rd band, 171. Beijing: The Writers Publishing House.}

\ex\label{ex:ABleyiAB-ccl}
\gll \objex{ji3duo1} \objex{san3luan4}-zhe de \objex{chuan2} \objex{li3} de \objex{deng1guang1}, \objex{ye3} \objex{hu1yin1hu1mie4} de \obj{bian4huan4-le-yi-bian4huan4} \objex{wei4zhi}.\\
many scattered-\textsc{dur} \textsc{de} boat in \textsc{de} light also flicker \textsc{de} change-\textsc{pfv}-one-chang position\\ \jambox{(CCL)}
\glt `Many scattered lights in the boats also changed their positions a little bit, flickering.'
\z
\z
\end{sloppypar}

This suggests that  even though AB-\textit{yi}-AB and AB-\textit{le}-\textit{yi}-AB might be degraded, they are not ungrammatical \textit{per se}.
The reason for this degradedness is probably phonological, since AB-\textit{yi}-AB and AB-\textit{le}-\textit{yi}-AB contain too many syllables \citetext{\citealp[274]{Fan1964}; \citealp[15]{Zhang2000}; \citealp[239]{YangWei2017}; \citealp[143]{Sui2018}}, but we argue that it is not an issue of grammaticality.
Thus, they can still be produced via the lexical rules posited above, but are ruled out or degraded due to a general phonological constraint.\footnote{
One reviewer suggests that given that AB-\textit{yi}-AB and AB-\textit{le}-\textit{yi}-AB were possible in previous stages of the language (see \citealt[15]{Zhang2000}, \citealt[160--161]{BascianoMelloni2017}),
these rare occurrences can be seen as relics of this usage.
}


AAB, A-\textit{yi}-AB, A-\textit{le}-AB, AA-\obj{kan4} and A-\obj{kan4}-\obj{kan4} can also be accounted for by the lexical rules proposed in this section.
They can be analyzed as compounds consisting of a reduplicated monosyllabic verb and another element.
Specifically, AAB, A-\textit{yi}-AB and A-\textit{le}-AB can be considered as the compound of a reduplicated monosyllabic verb (A) and a noun (B).\footnote{
    \citet[64--65]{Huang1984} and \citeauthor{Her1996} (\citeyear[Sec.\,2]{Her1996}; \citeyear[Sec.\,3.1]{Her2010}) argue that some of these V-O combinations are compounds, some are phrases, and some have dual status (both compounds and phrases).
    Following this approach, AAB, A-\textit{yi}-AB and A-\textit{le}-AB can (also) be considered as the phrasal combination of a reduplicated verb and its object.
}
AA-\obj{kan4} can be regarded as the compound of a reduplicated monosyllabic verb (A) and the verb \obj{kan4} `look',
whereas A-\obj{kan4}-\obj{kan4} is the compound of a monosyllabic verb (A) and the reduplication of \obj{kan4} `look'.
A-\obj{yi}-A-\obj{kan4} is also possible, though rare, presumably also due to its length.
An inquiry in CCL found 55 hits of A-\obj{yi}-A-\obj{kan4}.
A sample is listed in (\ref{ex:AyiAkan}).

\begin{sloppypar}
\ea\label{ex:AyiAkan}
\ea
\gll \objex{te4yi4} \objex{gong1neng2} de \objex{yan2jiu1zhe3}-men \objex{bu4fang2} \objex{ru2ci3} \obj{shi4-yi-shi4-kan4} \ldots\\
special power \textsc{de} researcher-\textsc{pl} may.as.well such try-one-try-look\\ \jambox{(CCL)}
\glt `Researchers of special power may as well have a try as such and see \ldots'

\ex 
\gll \objex{dan4shi4} \objex{dui4} \objex{fa1} \objex{mei2} \objex{fa1}-\objex{guo} \objex{he2ge2-zheng4}, \objex{yi3jing1} \objex{shuo1} \objex{bu4} \objex{qing1} le, \objex{xu1yao4} \obj{cha2-yi-cha2-kan4}.\\
but about issue not issue-\textsc{exp} conformity-certificate already say not clealy \textsc{ptc} need check-one-chek-look\\ \jambox{(CCL)}
\glt `But one already cannot say it clearly anymore, whether a certificate of conformity is issued or not. One needs to have a check and see.'

\ex
\gll \objex{rang4} \objex{wo3} \objex{lai2} \obj{cai1-yi-cai1-kan4}.\\
let I come guess-one-guess-look\\ \jambox{({CCL})}
\glt `Let me have a guess.'

\ex
\gll \objex{da4-lao3ban3}-men \objex{yao4} \obj{deng3-yi-deng3-kan4}\\
big-boss-\textsc{pl} need wait-one-wait-look\\ \jambox{(CCL)}
\glt `Big bosses need to wait a little bit and see.'

\ex
\gll \objex{fu1ren2} \objex{ni3} \objex{dao4} \obj{shu3-yi-shu3-kan4}, \objex{zhe4} \objex{zhu1} \objex{hua1} de \objex{hua1duo3} \objex{gong4} \objex{you3} \objex{ji3} \objex{zhong3} \objex{yan2se4}.\\
madam you just count-one-count-look this \textsc{clf} flower \textsc{de} blossom in.total have how.many \textsc{clf} color\\ \jambox{(CCL)}
\glt `Madam, just try to count and see how many colors the blossom of this flower has in total.'
\z
\z
Due to the prominent tentative, trying meaning of AA-\obj{kan4} and A-\obj{kan4}-\obj{kan4}, they are not compatible with the perfective aspect marker \obj{le} semantically,
as one usually cannot try something that is already realized.
Thus, structures such as  A\hyp{}\obj{le}\hyp{}A\hyp{}\obj{kan4} and A\hyp{}\obj{kan4\hyp{}le\hyp{}kan4} are considered pragmatically infelicitous.
\end{sloppypar}

The current analysis provides a unified account for all forms of delimitative verbal reduplication in Mandarin Chinese.
Like in \citet{FanSongBond2015}, \textit{yi} is handled as a phonological element which does not make any contribution to the semantics,
and an inheritance hierarchy is used to capture the commonalities among different forms of reduplication.
But the present proposal also reflects the connection between the reduplication and aspect marking via multiple inheritance.
This account makes use of a semantic mechanism, which correctly rules out aspect marking with forms other than \textit{le}.
By providing a semantic explanation, this mechanism seems less \textit{ad hoc} than the one used in \citet{FanSongBond2015}, which simply assumed that the reduplication cannot combine with aspect information.
The present approach also has a broader coverage of the forms of verbal reduplication than the one in \citet{FanSongBond2015}.
Furthermore, all the forms are derivable from the lexical rules proposed here, so that there is no need to resort to irregular lexicon entries, and the productivity of these forms is correctly captured.
In sum, the analysis proposed in this paper possesses greater explanatory power and resolves the
problems of previous studies.


%      <!-- Local IspellDict: en_US-w_accents --> 
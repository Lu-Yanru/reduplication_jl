\section{A new HPSG analysis}\label{sec:analysis}

In what follows, we suggest a new lexical-rule-based analysis of aspect marking and reduplication
using Minimal Recursion Semantics (MRS; \citealt{Copestakeetal2005}).%, as proposed in \citet{LuMueller2021}.

The implicational constraint in  (\ref{avm:redup}) shows the constraints on all structures of type
\textit{verbal-reduplication-lr} for Mandarin Chinese. 
Such structures take a verb as \textsc{lexical-daughter (lex-dtr)}.  
The output reduplicates the \textsc{phonology (phon)} of the input verb with the possibility to have further phonological material in between.
\etag{} indicates an underspecified list which could be empty or not. 
A delimitative relation is appended to the \textsc{relations (rels)} value of the input verb, 
and it takes the event index of the input verb as argument. 
The label of the output \iboxb{2} is identified with the label of the input and with the label of the delimitative relation, 
hence \type{delimitative-rel} is treated as a modifier.
 Further relations can be added at the beginning of the \textsc{rels} list to allow for
the additional perfective meaning in A-\textit{le}-A and A-\textit{le}-\textit{yi}-A. 
The combination with the perfective will be elaborated on in the following paragraphs.

\ea
\type{verbal-reduplication-lr} \impl\\
%\scalebox{.9}
\label{avm:redup}
\z

To account for the variations in the phonology of the reduplication as well as 
the combination with the phonology and semantics of the perfective aspect marker \textit{le}, 
the type hierarchy of lexical rules in \figref{fig:typehi} is put forward. 
\begin{figure}
    \centering
    \scalebox{.9}{\begin{forest}
        type hierarchy
        [,phantom
        [,phantom
        [verbal-reduplication-lr,name=verbal-reduplicaiton-lr
        [non-perfective-reduplicaiton-lr
        [a-a-lr]
        [a-yi-a-lr]]
        [,identify=!r211]]]
        [aspect-marking-lr
        [perfective-lr
        [perfective-reduplicaiton-lr%,edge to=verbal-reduplicaiton-lr
        [a-le-yi-a-lr]
        [a-le-a-lr]]
        [v-le-lr]]
        [durative-lr]
        [\ldots]]]
    \end{forest}
}
    \caption{Type hierarchy for lexical rules of verbal reduplication and \textit{le}}
    \label{fig:typehi}
\end{figure}
Apart from the type \type{perfective-reduplication-lr}, 
which adds the inherited perfective relation, 
there is a subtype \type{non-perfective-reduplication-lr}, 
which does not add further relations. 
Hence, what is \etag in the \textsc{rels} list in (\ref{avm:redup}) is the empty list in (\ref{avm:nonperf}):
\ea\label{avm:nonperf}
\type{non-perfective-verbal-reduplication-lr} \impl\\
%\scalebox{.9}
\z
The \textsc{rels} list of the output of the lexical rule is the \textsc{rels} list of
the daughter \iboxb{1} plus a list with one element. 
Since this element is specified in the supertype, it is not specified in (\ref{avm:nonperf}) again. 

\type{non-perfective-verbal-reduplication-lr} has \type{aa-lr} and \type{a-yi-a-lr} as direct subtypes.
(\ref{avm:AA}) and (\ref{avm:AyiA}) show \type{aa-lr} and \type{a-yi-a-lr}, respectively.
As subtypes of \type{ver\-bal\hyp{}re\-dup\-li\-ca\-tion\hyp{}lr} illustrated in (\ref{avm:redup}), 
both inherit the constraints on the \textsc{lex-dtr} and on the semantics of the output, 
and because of (\ref{avm:nonperf}), no extra material is appended to the \textsc{rels} value of the input verb
and the list containing the \type{delimitative-rel}.
 In addition to the inherited constraints,
\type{aa-lr} and \type{a-yi-a-lr} specify the phonology of the output differently.
\type{aa-lr} determines that the \etag between the two phonological copies in (\ref{avm:redup}) is the empty list, 
whereas \type{a-yi-a-lr} specifies this list of phonological material as \phonliste{ yi }:

\ea\label{avm:constr-AA}
Constraints on lexical rules of type \type{aa-lr} and \type{a-yi-a-lr}:
%\begin{tabular}{@{}l@{\hspace{2cm}}l@{}}
 \ea   \type{aa-lr} \impl \\%& 
  %  \scalebox{.9} %& 
\ex	\type{a-yi-a-lr} \impl\\
 %    \scalebox{.9}
%\end{tabular}
\z
\z


\noindent
The lexical rules with all inherited constraints are given in (\ref{avm:AA}) and (\ref{avm:AyiA}):

\ea\label{avm:AA}
The AA lexical rule with all constraints inherited from the supertypes:
%\scalebox{.9}
\z

\ea\label{avm:AyiA}
The A-\textit{yi}-A lexical rule with all constraints inherited from the supertypes:
%\scalebox{.9}
\z

\type{v-le-lr} is a direct subtype of the \type{perfective-lr}.
\type{perfective-reduplication-lr} inherits from both \type{verbal-reduplication-lr} and \type{per\-fec\-tive\hyp{}lr}
and has two subtypes, \type{a-le-yi-a-lr} and \type{a-le-a-lr} itself.
\type{verbal-reduplication-lr} is already presented in (\ref{avm:redup}). 
We now turn to the constraints on \type{perfective-lr} and its subtypes.


\citet[246]{MuellerLipenkova2013} proposed the perfective lexical rule given in (\ref{avm:pfv-old}), 
adapted to the formalization adopted in the current paper.
It takes a verb as \textsc{lex-dtr}
and appends \phonliste{ le } to its phonology.
Further, it accounts for the change in semantics by appending the \textsc{rels} value of the input verb to a \type{perfective-rel}.

\ea\label{avm:pfv-old}
Perfective lexical rule adapted from \citet[246]{MuellerLipenkova2013}:\\
%\scalebox{.9}
\z
The event variables \iboxb{3} of the input and the output verb are shared. 
The \textsc{ltop} of the output of the lexical rule \iboxb{2} is the label of the perfective relation, 
and this relation scopes over the embedded verb. 
The handle of the embedded verb \iboxb{4} is the argument of the \type{perfective-rel}. 

The lexical rule suggested in (\ref{avm:pfv-old}) only explains simple perfective aspect marking with \textit{le}, 
where \textit{le} immediately follows the verb.
But it cannot account for the perfective aspect marking of a reduplicated verb,
 as \textit{le} does not occur after the reduplication, 
nor can \textit{le} be reduplicated together with the verb.
It can only appear between the verb and the reduplicant.
In order to accommodate \textit{le} marking for both simple and reduplicated verbs, 
a general perfective lexical rule as in (\ref{avm:pfv-new}) and 
a subtype \type{v-le-lr} as in (\ref{avm:vle}) are posited here.
Besides adding a \type{perfective-rel} in the \textsc{rels} list of the output as in (\ref{avm:pfv-old}), 
the \type{perfective-lr} in (\ref{avm:pfv-new}) allows an underspecified list to be appended at the end of the \textsc{rels} list.
The \textsc{phon} value of the output makes it possible for further phonological material to occur both before and after \phonliste{ le }.

\ea\label{avm:pfv-new}
Type constraints on the type \type{perfective-lr} from which other subtypes inherit:\\
%\scalebox{.9}{%
    \avm{
        [\type*{perfective-lr}\\
        phon & \etag $\oplus$ \phonliste{ le } $\oplus$ \etag\\
        \punk{synsem\textbar loc \textbar cont}{ [ltop & \2\\
            ind  & \3]}\\ 
        rels & <[\type*{perfective-rel}\\
        lbl  & \2\\
        arg0 & \3\\
        arg1 & \4 ]> $\oplus$ \5 $\oplus$ \etag\smallskip\\
        lex-dtr & [%phon & \1\\
        \punk{synsem\textbar loc}{[cat & [ head & verb ]\\
            cont & [ ltop & \4\\
            ind  & \3]]}\\
        rels & \5]
        ]
}
%}
\z

\type{v-le-lr} with all inherited constraints as given in (\ref{avm:vle}) inherits from \type{perfective-lr} and specifies that the first element in the output \textsc{phon} list is identified with the \textsc{phon} value of the input verb
and that nothing else comes after \phonliste{ le }.
Furthermore, no other list can be appended at the end of the \textsc{rels} list of the output anymore.
This corresponds to the proposal of \citet[246]{MuellerLipenkova2013} shown in (\ref{avm:pfv-old}), 
which accounts for the simple perfective marking of verbs.

\ea\label{avm:vle}
Structure of type \type{v-le-lr} with constraints inherited from \type{perfective-lr}:\\
%\scalebox{.9}
\z

\type{perfective-reduplication-lr} inherits from both \type{verbal-reduplication-lr} and \type{per\-fec\-tive\hyp{}lr}.
The \textsc{phon} value of the output reduplicates the phonology of the input verb and states that there is \phonliste{ le } in between, 
as well as potentially further phonological material.
The \textsc{rels} list of the output appends the \type{delimitative-rel} to the \type{perfective-rel} and the \textsc{rels} value of the input verb.
The arguments of both \type{perfective-rel} and \type{delimitative-rel} share the event index of the
input verb \iboxb{3} to ensure that they apply to the same event denoted by the input verb. 
The label of the \type{delimitative-rel} and the input verb are identified (\type{delimitative-rel} is a modifier) 
and this shared label is embedded under the \type{perfective-rel}.

\ea\label{avm:pfv-redup}
Perfective and reduplication combined: type \type{perfective-reduplication-lr} with
constraints inherited from \type{perfective-lr} and \type{verbal-reduplication-lr}:
%\resizebox{\linewidth}{!}{
 %   \scalebox{.9}{%
        \avm{
            [\type*{perfective-reduplication-lr}\\
            phon & \1 $\oplus$ \phonliste{ le } $\oplus$ \etag $\oplus$ \1\\
            \punk{synsem\textbar loc \textbar cont\textbar ltop}{ \2}\\ 
            rels & <[\type*{perfective-rel}\\
            lbl & \2\\
            arg0 & \3\\
            arg1 & \4]>
            $\oplus$ \5 $\oplus$
            <[\type*{delimitative-rel}\\
            lbl  & \4\\
            arg0 & \3]>\\
            lex-dtr & [phon & \1\\
            synsem\textbar loc & [cat  & [ head & verb ]\\
            cont & [ ltop & \4\\
            ind  & \3]]\\
            rels & \5]
            ]
    }
%}
    %}
\z
For example (\ref{ex-he-tasted-the-soup-a-little-bit}), repeated here as (\ref{ex-he-tasted-the-soup-a-little-bit2}), we get the  MRS representation in (\ref{ex:mrs-soup}),
where h1 and h2 correspond to the handles \ibox{2} and \ibox{4} and e1 to the event variable \ibox{3}:
\ea\label{ex-he-tasted-the-soup-a-little-bit2}
    \gll ta \textit{chang}-\textit{le}-\textit{chang} tang.\\
    he taste-\textsc{pfv}-taste soup\\
    \glt `He tasted the soup a little bit.'
 \z
\ea\label{ex:mrs-soup}
h1 \sliste{ h1:perfective(e1,h2), h2:taste(e1,he,soup), h2:delimitative(e1) }
\z
So the delimitative relation is treated as an adjunct to the main relation of the verb, 
and the perfective relation scopes over both the main relation and the delimitative relation.

Two subtypes of \type{perfective-reduplication-lr} are posited:
\type{a\hyp{}le\hyp{}yi\hyp{}a\hyp{}lr} %as in (\ref{avm:AleyiA}) 
and \type{a-le-a-lr}, as shown in (\ref{avm:AleA}). %(\ref{avm:AleA}).
They take over the semantic change to the input from \type{perfective\hyp{}reduplication\hyp{}lr}, but specify the \textsc{phon} value differently.
Specifically, \type{a\hyp{}le\hyp{}yi\hyp{}a\hyp{}lr} specifies the middle phonological material as
\phonliste{ le, yi }, while \type{a-le-a} specifies it as \phonliste{ le } only.


\ea\label{avm:AleA}
%\begin{tabular}{@{}l@{\hspace{2cm}}l@{}}
\ea    %\scalebox{.9}{%
        \type{a-le-yi-a-lr} \impl\\
        \avm{
            [%\type*{a-le-yi-a-lr}\\
            phon & \1 \+ \phonliste{ le, yi } \+ \1\\
            lex-dtr & [phon & \1
            ]
            ]
    }
%}%&
\ex   % \scalebox{.9}{%
    \type{a-le-a-lr} \impl\\
        \avm{
            [%\type*{a-le-a-lr}\\
            phon & \1 \+ \phonliste{ le } \+ \1\\
            lex-dtr & [phon & \1
            ]
            ]
    }
%}
%\end{tabular}
\z
\z



Since the above-described lexical rules do not constrain the number of syllables of the input verb, but simply reduplicate its phonology  as a whole,
they can also account for the ABAB and the AB-\textit{le}-AB forms of reduplication,
as long as the input verb is disyllabic.
Notice that  the lexical rules above also produce AB-\textit{yi}-AB and AB\hyp{}\textit{le}\hyp{}\textit{yi}\hyp{}AB for disyllabic input verbs.
Although these forms are generally considered unacceptable \citetext{\citealp[160]{BascianoMelloni2017}, \citealp[275--276]{Hong1999}, \citealp[30]{LiThompson1981}, \citealp[239]{YangWei2017}}, 
\citet[269]{Fan1964} and \citet[143]{Sui2018} considered AB-\textit{yi}-AB and AB-\textit{le}-\textit{yi}-AB to be possible, even though they both recognized that these two forms are rare.
Indeed, a few examples of AB\hyp{}\emph{yi}\hyp{}AB and AB\hyp{}\emph{le}\hyp{}\emph{yi}\hyp{}AB in Early Mandarin (12th to 20th centuries) (\ref{ex:AByiAB-lu}--\subref{ex:AByiAB-pu}) and Modern Mandarin (\ref{ex:AByiAB-rou1}--\subref{ex:ABleyiAB-ccl}) were found.

\settowidth\jamwidth{(CCL)}

\footnotetext[17]{\textit{Yuanqu xuan: Luzhailang [Selected Yuanqu: Luzhailang]}, as cited in \citet[15]{Zhang2000}}
\footnotetext[18]{\textit{Yuan Ming juan: Piaotongshi [Yuan and Ming volume: Piaotongshi]}, 308, as cited in \citet[15]{Zhang2000}}
\footnotetext[19]{Rou, Shi. 1975. \textit{Roushi xiaoshuo xuanji [Selected novels of Roushi]}, 31. Beijing: People's Literature Publishing House.}
\footnotetext[20]{Li, Jieren. 1962. \textit{Da bo [Great wave]}, vol. 3, 171. Beijing: The Writers Publishing House.}
\begin{sloppypar}
\ea\label{ex:ABleyiAB}
\ea\label{ex:AByiAB-lu}
\gll ni yu wo \textit{zhengli-yi-zhengli}.\footnotemark\\
you let me arrange-one-arrange\\
\glt `Let me arrange it a little bit!'

\ex\label{ex:AByiAB-pu}
\gll ni \textit{dating-yi-dating}.\footnotemark\\
you inquire-one-inquire\\
\glt `Inquire about it a little bit!'

\ex\label{ex:AByiAB-rou1}
\gll ge ge dian-tou \textit{weixiao-yi-weixiao}.\footnotemark\\
\textsc{clf} \textsc{clf} nod-head smile-one-smile\\
\glt `Each one nodded his head and smiled a little bit.'

\ex\label{ex:AByiAB-rou2}
\gll ta \textit{weixiao-le-yi-weixiao}, you \textit{mingxiang-le-yi-mingxiang}.\footnotemark\\
he smile-\textsc{pfv}-one-smile and meditate-\textsc{pfv}-one-meditate\\
\glt `He smiled a little bit and meditated a little bit.'

\ex\label{ex:AByiAB-li}
\gll feichang yansu de ba jinshi yanjing \textit{duanzheng-le-yi-duanzheng}.\footnotemark\\
very seriously \textsc{de} \textsc{ba} nearsighted glasses straighten-\textsc{pfv}-one-straighten\\
\glt `[He] very seriously straightened the nearsighted glasses quickly.'

\ex\label{ex:ABleyiAB-ccl}
\gll jiduo sanluan-zhe de chuan li de dengguang, ye huyinhumie de \textit{bianhuan-le-yi-bianhuan} weizhi.\\
many scattered-\textsc{dur} \textsc{de} boat in \textsc{de} light also flicker \textsc{de} change-\textsc{pfv}-one-chang position\\ \jambox{(CCL)}
\glt `Many scattered lights in the boats also changed their positions a little bit, flickering.'
\z
\z
\end{sloppypar}

This suggests that  even though AB-\textit{yi}-AB and AB-\textit{le}-\textit{yi}-AB might be degraded, they are not ungrammatical \textit{per se}.
The reason for this degradedness is probably phonological, since AB-\textit{yi}-AB and AB-\textit{le}-\textit{yi}-AB contain too many syllables \citetext{\citealp[274]{Fan1964}, \citealp[143]{Sui2018}, \citealp[239]{YangWei2017}, \citealp[15]{Zhang2000}}, but we argue that it is not an issue of grammaticality.
Thus, they can still be produced via the lexical rules posited above, but are ruled out or degraded due to a general phonological constraint.


AAB, A-\textit{yi}-AB, A-\textit{le}-AB, AA-\textit{kan} and A-\textit{kan}-\textit{kan} can also be accounted for by the lexical rules proposed in this section.
They can be analyzed as compounds consisting of a reduplicated monosyllabic verb and another element.
Specifically, AAB, A-\textit{yi}-AB and A-\textit{le}-AB can be considered as the compound of a reduplicated monosyllabic verb (A) and a noun (B).\footnote{
    \citet{Huang1984} and \citet{Her1996, Her2010} argued that some of these V-O combinations are compounds, some are phrases, and some have dual status (both compounds and phrases).
    Following this approach, AAB, A-\textit{yi}-AB and A-\textit{le}-AB can (also) be considered as the phrasal combination of a reduplicated verb and its object.
}
AA-\textit{kan} can be regarded as the compound of a reduplicated monosyllabic verb (A) and the verb \textit{kan} `look',
whereas A-\textit{kan}-\textit{kan} is the compound of a monosyllabic verb (A) and the reduplication of \textit{kan} `look'.
A-\textit{yi}-A-\textit{kan} is also possible, though rare, presumably also due to its length.
An inquiry in CCL found 55 hits of A-\textit{yi}-A-\textit{kan}.
A sample is listed in (\ref{ex:AyiAkan}).

\begin{sloppypar}
\ea\label{ex:AyiAkan}
\ea
\gll teyi gongneng de yanjiuzhe-men bufang ruci \textit{shi-yi-shi-kan} \ldots\\
special power \textsc{de} researcher-\textsc{pl} may.as.well such try-one-try-look\\ \jambox{(CCL)}
\glt `Researchers of special power may as well have a try as such as see \ldots'

\ex 
\gll danshi dui fa mei fa-guo hege-zheng, yijing shuo bu qing le, xuyao \textit{cha-yi-cha-kan}.\\
but about issue not issue-\textsc{exp} conformity-certificate already say not clealy \textsc{ptc} need check-one-chek-look\\ \jambox{(CCL)}
\glt `But one already cannot say it clearly anymore, whether a certificate of conformity is issued or not. One needs to have a check and see.'

\ex
\gll rang wo lai \textit{cai-yi-cai-kan}.\\
let I come guess-one-guess-look\\ \jambox{({CCL})}
\glt `Let me have a guess.'

\ex
\gll da-laoban-men yao \textit{deng-yi-deng-kan}\\
big-boss-\textsc{pl} need wait-one-wait-look\\ \jambox{(CCL)}
\glt `Big bosses need to wait a little bit and see.'

\ex
\gll furen ni dao \textit{shu-yi-shu-kan}, zhe zhu hua de huaduo gong you ji zhong yanse.\\
madam you just count-one-count-look this \textsc{clf} flower \textsc{de} blossom in.total have how.many \textsc{clf} color\\ \jambox{(CCL)}
\glt `Madam, just try to count and see how many colors the blossom of this flower has in total.'
\z
\z
Due to the prominent tentative, trying meaning of AA-\textit{kan} and A-\textit{kan}-\textit{kan}, they are not compatible with the perfective aspect marker \textit{le} semantically,
as one usually cannot try something that is already realized.
Thus, structures such as  A\hyp{}\textit{le}\hyp{}A\hyp{}\textit{kan} and A\hyp{}\textit{kan\hyp{}le\hyp{}kan} are considered pragmatically infelicitous.
\end{sloppypar}

The current analysis provides a unified account for all forms of delimitative verbal reduplication in Mandarin Chinese.
Like in \citet{FanSongBond2015}, \textit{yi} is handled as a phonological element which does not make any contribution to the semantics,
and an inheritance hierarchy is used to capture the commonalities among different forms of reduplication.
But the present proposal also reflects the connection between the reduplication and aspect marking via multiple inheritance.
This account makes use of a semantic mechanism, which correctly rules out aspect marking with forms other than \textit{le}.
By providing a semantic explanation, this mechanism seems less \textit{ad hoc} than the one used in \citet{FanSongBond2015}, which simply assumed that the reduplication cannot combine with aspect information.
The present approach also has a broader coverage of the forms of verbal reduplication than the one in \citet{FanSongBond2015}.
Furthermore, all the forms are derivable from the lexical rules proposed here, so that there is no need to resort to irregular lexicon entries, and the productivity of these forms is correctly captured.
In sum, the analysis proposed in this paper possesses greater explanatory power and resolves the problems of previous studies.
\section{Analysis based on the aspectual system by \citet{Tsai2008}}\label{app:tsai}

\citet[229]{YangWei2017} claim that reduplication can be analyzed as an aspect marker 
following the structure of Mandarin Chinese aspects proposed by \citet{Tsai2008}.
 \citet{Tsai2008} provides the syntactic analysis for aspect markers in Mandarin Chinese as shown in \figref{tree:tsai}.\footnote{Asp = Aspect}
He observes the so-called incompleteness effect, namely that a minimal sentence, 
which only contains a verb marked by \obj{zhe} `\textsc{dur}',
\obj{le} `\textsc{pfv}' or \obj{wan2} `\textsc{compl}' 
and its arguments, 
seems incomplete without further sentential elements 
such as the sentence final particle \obj{le} 
or a temporal adverbial like \obj{gang1cai2} `just now' (\ref{ex:incomplete}). 
In contrast, a minimal sentence with a verb marked by  \obj{zai4} `\textsc{prog}' or \obj{guo}
`\textsc{exp}'  and its arguments can stand alone (\ref{ex:complete}).

\ea\label{ex:incomplete}
\gll \objex{Xiao3li3} \objex{chi1}-zhe/le/\objex{wan2} \objex{fan4} *(le).\\
Xiaoli eat-\textsc{dur}/\textsc{pfv}/\textsc{compl} meal \textsc{ptc}\\
\glt `Xiaoli is eating/ate/finished eating the meal.'
\z

\ea\label{ex:complete}
\gll \objex{Xiao3li3} (\objex{gang1cai2}) \objex{zai4} \objex{ku1}/\objex{ku1}-\objex{guo}.\\
Xiaoli just.now \textsc{prog} cry/cry-\textsc{exp}\\
\glt `Xiaoli was crying/cried just now.'
\z
He thus proposes three aspect positions under TP. \obj{zai4} `\textsc{prog}' and \obj{guo} `\textsc{exp}' reside under Asp$_1$, 
while \obj{zhe} `\textsc{dur}' and \obj{le} `\textsc{pfv}' under Asp$_2$, as illustrated in \figref{tree:tsai}.\footnote{
    \citet{Tsai2008} differentiated the middle and the inner aspect based on the fact that 
    \obj{wan2} can only occur with certain types of predicate.
    This differentiation does not play a role for our purpose and will not be further discussed here.}

\begin{figure}
    \centering
    \begin{forest}
        [TP [T]
        [...
        [AspP$_1$ (outer aspect) [Asp$_1$\\\obj{zai4/guo}]
        [\textit{v}P [\textit{v}]
        [AspP$_2$ (middle aspect) [Asp$_2$\\\obj{zhe/le}]
        [VP [V-Asp$_3$ (inner aspect)\\\obj{wan2}]
        ]
        ]
        ]
        ]
        ]
        ]
    \end{forest}
    \caption{Structure of the aspectual system in Mandarin Chinese according to \citet[683]{Tsai2008}}
    \label{tree:tsai}
\end{figure}

Turning to reduplication, a minimal sentence with reduplication also seems incomplete (\ref{ex:redup-tsai}).
Based on this, the reduplicant should reside under Asp$_2$, as illustrated in \figref{tree:redupasp}.

\ea\label{ex:redup-tsai}
\ea[*] {\gll \objex{ta1} \obj{xiao4}-\obj{xiao}\\
    he laugh-laugh\\}
\ex[] {\gll \objex{ta1} \obj{xiao4}-\obj{le}-\obj{xiao4}\\
    he laugh-\textsc{pfv}-laugh\\
    \glt `He laughed a little.'}
\ex[] {\gll \objex{ta1} \obj{xiao4}-\obj{xiao}, \objex{bu4} \objex{shuo1hua4}\\
    he laugh-laugh not speak\\
    \glt `He laughed a little, and didn't speak.'}\label{ex:redup-tsai-co}
\z
\z

\begin{figure}
    \centering
    \begin{forest}
        [TP [T]
        [...
        [AspP$_1$ (outer aspect) [Asp$_1$\\\obj{zai4/guo}]
        [\textit{v}P [\textit{v}]
        [AspP$_2$ (middle aspect) [Asp$_2$\\\obj{zhe}/\obj{le}/reduplication]
        [VP [V-Asp$_3$ (inner aspect)\\\obj{wan2}]
        ]
        ]
        ]
        ]
        ]
        ]
    \end{forest}
    \caption{Position of reduplication according to the aspectual system in \citet{Tsai2008}}
    \label{tree:redupasp}
\end{figure}

This analysis would result in a mismatch between syntax and semantics, 
in the sense that the aspect markers that belong to the same semantic group do not occur in the same syntactic position. 
Even though \obj{le} `\textsc{pfv}', \obj{guo} `\textsc{exp}' and reduplication all mark perfective aspects (\sectref{sec:aspM}, \citealt{Dai1997, XiaoMcEnery2004}),
\obj{guo} `\textsc{exp}' is situated under Asp$_1$, while \obj{le} `\textsc{pfv}' and reduplication are under Asp$_2$.
Similarly, \obj{zai4} `\textsc{prog}' and \obj{zhe} `\textsc{dur}' are both imperfective aspects but also occur in different syntactic positions.

\section{\obj{yi} and \obj{le}}\label{app:hist}
A reviewer suggests that 
the fact that only  \obj{yi} and \obj{le}  can appear in between reduplication 
seems somewhat arbitrary, 
and it is not evident why certain elements are permissible while others are not.
We think that there might be historical and phonological reasons.

\obj{le} `\textsc{pfv}' marks the perfective aspect.
The reason for \obj{le} `\textsc{pfv}' being the only aspect marker compatible with reduplication is explained in Section~\ref{sec:aspM}.

As for \obj{yi} `one', its use stems from the historical development of delimitative verbal reduplication but is synchronically opac.
\citet[13--15]{Zhang2000} shows that delimitative verbal reduplication originates from the ``V + numeral + verbal classifier'' phrase,
where the verbal classifier is borrowed from a verb of the same form, as shown in (\ref{ex:hist1}) from \obj{Song4dai4} \obj{juan4}: \obj{Xu1tang2} \obj{He2shang} \obj{yu3lu4} \textit{[The Song dynasty volume: Quotations from Abbot Xutang]} p. 387 as cited in \citet[12]{Zhang2000}.
\ea\label{ex:hist1}
%明又喝,岐亦喝,明连喝两喝,岐便礼拜。(《宋代卷·虚堂和尚语录》3 87 页) as cited in \citet[12]{Zhang2000}
\gll \objex{ming2} \objex{you4} \objex{he4}, \objex{qi2} \objex{yi4} \objex{he4}, \objex{ming2} \objex{lian2} \objex{he4} \objex{liang3} \objex{he4}, \objex{qi2} \objex{bian4} \objex{li3bai4}\\
Ming again yell Qi also yell Ming in.succession yell two yell Qi thus bow\\
\glt `Ming yelled again. Qi yelled too. Ming yelled twice. Qi thus bowed.'
\z
Here the numeral refers to the actual number of the action taking place
and can be any number.
This evolved into the structure of V-\obj{yi}-V (or A-\obj{yi}-A), which does not express the actual number of the action anymore,
but simply conveys that the action happens in a short period of time or for only few times. 
Consider (\ref{ex:hist2}) from \obj{Song4dai4} \obj{juan4}: \obj{Zhang1} \obj{Xie2} \obj{zhuang4yuan2} \textit{[The Song dynasty volume: Top graduate Zhang Xie]} p. 519 as cited in \citet[13]{Zhang2000}.
\ea\label{ex:hist2}
%且歇一歇了,去坐地。(《宋代卷·张协状元》5 19 页) as cited in \citet[13]{Zhang2000}
\gll \objex{qie3} \obj{xie1-yi-xie1} \objex{le}, \objex{qu4} \objex{zuo4} \objex{di4}.\\
just rest-one-rest \textsc{ptc} go sit ground\\
\glt `(Let me) just take a rest and go sit on the ground.'
\z
In this case, \obj{yi1} `one' cannot be replaced by other numerals.
Since V-\obj{yi}-V does not express the actual number of the action anymore,
it became possible to delete the \obj{yi1} `one',
hence the use of the VV (or AA) form, too.

Synchronically, A-\obj{yi}-A has the same meaning as AA and both do not refer to the actual number of the event denoted by the verb.
This means that \obj{yi1} `one' in A-\obj{yi}-A does not contribute any specific meaning to the structure
and is only there as a historical remnant.
In this sense,  \obj{yi1} `one' appearing in between the reduplication is synchronically arbitrary. %similar to linking elements in compounds

Besides the obvious conceptual reason that  \obj{yi1} `one' is taken to mean `little/few',
there might be a phonological explanation for  \obj{yi1} `one' rather than other numerals is used in the A-\obj{yi}-A structure.
In Mandarin Chinese, one form of intensifying reduplication is A-\obj{li}-AB e.g. \obj{hu2-li-hu2tu2} `confused-li-confused',
where the \obj{li} is fixed and also does not bear any meaning.
\citet[137]{Sui2018} assumes that this syllable is filled with \obj{li} because the open syllable \obj{li} is a relatively unmarked phonological constituent \citep{Yip1992}
and the second syllable in A-\obj{li}-AB occupies an unstressed position.
We can see the similarities between \obj{li} in A-\obj{li}-AB and \obj{yi} in A-\obj{yi}-A: they are both relatively unmarked and occupies an unstressed position.
This can also make it easier for  \obj{yi1} `one' rather than other numerals to become part of a fixed structure.
But a phonological account is out of the scope of this paper and has to be left for further research.

%The subtypes of \type{verbal-reduplication-lr} (\ref{avm:AyiA}, \ref{avm:AleA}) prevent phonological materials other than \phonliste{ le } and \phonliste{ yi } from appearing in between the reduplication by specifying what can appear in the \etag list and what not.

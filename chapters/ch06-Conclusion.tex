\section{Conclusions}\label{sec:conclu}

The current study provides a new HPSG account for verbal reduplication in Mandarin Chinese.
We present empirical evidence that reduplication is possible with all verb classes.
We give a semantic explanation for the incompatibility of reduplication with aspect markers other than \obj{le}.
We argue that reduplication is a morphological rather than a syntactic process.
We model reduplication as a lexical rule,
and the different forms of reduplication are captured in an inheritance hierarchy using underspecified lists.
The interaction between verbal reduplication and aspect marking is handled by multiple inheritance.
This analysis is compatible with both mono- and disyllabic verbs, 
so that all productive forms of reduplication are derivable by lexical rules.
The analysis is implemented as part of a computer-processable grammar of Mandarin Chinese.
%The analysis is implemented as part of the CoreGram project \citep{MuellerCoreGram} in a Chinese
%grammar in the TRALE system \citep{MeurersEtAl2002, Penn2004}.

%% -*- coding:utf-8 -*-
\documentclass[fleqn,twoside]{article}


\usepackage[
style = langsci-unified,
backend = biber,
natbib = true,
]{biblatex}
\addbibresource{Bib.bib}

\usepackage{xcolor}
\definecolor{lsDOIGray}{cmyk}{0,0,0,0.45}

\usepackage{csquotes}

%% Language
\usepackage[ngerman, english]{babel}
	

\usepackage{microtype}


%%%
%% Tables, lists, columns
% Text in columns: \begin{}{n} \columnbreak \end{multicols}
\usepackage{tabularx}
\usepackage{multicol}
%	\setlength{\columnsep}{.5cm}	


\usepackage{float}

\usepackage{subcaption}



%%%
%% Linguistic graphs
% tikz
\usepackage{tikz}
\usetikzlibrary{patterns, matrix}

% tree
%\usepackage{forest}
% Needed for the "actual forest version"
%\useforestlibrary{linguistics}
%\forestapplylibrarydefaults{linguistics}
\usepackage{langsci-forest-setup}

% avm
\usepackage{langsci-avm}




	
%% hyperref & url
\usepackage[
bookmarksnumbered, %For numbered bookmarks in PDF
hidelinks, %For links without colored borders
%breaklinks=true,
%pdfborder={0 0 0}
]{hyperref}
%\hypersetup{pdfborder={0 0 0}}
% line break url
%\usepackage{breakurl}


%hyphenation
\usepackage{german}
\selectlanguage{USenglish}
\usepackage{hyphenat}


\usepackage{etoolbox}

% examples and glossing
\usepackage{refcount}
\usepackage{xstring}

\usepackage{gb4e}
\exewidth{(222)}
\judgewidth{**}

\usepackage{jambox}


% disambiguate bib names
\newrobustcmd{\disambiguate}[3]{#2~#3}

% refs
\newcommand{\sectref}[1]{Section~\ref{#1}}
\newcommand{\figref}[1]{Figure~\ref{#1}}
\newcommand{\cioktegen}[2][]{\citeauthor{#2}'s (\citeyear*[#1]{#2})}

%%% Linguistics graphs related
%% indexed box
\newlength{\MyetagLength}
\settowidth{\MyetagLength}{{$\scriptstyle 1$}}

%% indexed box
\newcommand{\iboxt}[1]{{%
        \setlength{\fboxsep}{1.25pt}%
        \fbox{$\scriptstyle #1$}%
}}


\newcommand{\ibox}[1]{%
    %  \iboxt{#1}\,%       why this extra space???? 20.02.2018
    \iboxt{#1}%
}

\newcommand{\iboxb}[1]{(\,\iboxt{#1}\,)}

%% empty box
\newcommand{\etag}{\ibox{\rule{0ex}{1.1ex}\hspace{\MyetagLength}}\xspace}

%% type
\newcommand{\type}[1]{{\normalfont\itshape #1\/}}

%% list of phons
\newcommand{\phonliste}[1]{%
    \mbox{%
        $%
        %
        \left\langle \mbox{\normalfont\itshape#1} \right\rangle%
        $%
        %\\[-1.5mm]
    }%
}

\newcommand{\sliste}[1]{%
    \mbox{%
        $\left\langle\mbox{\upshape #1}\right\rangle$}%
}

\newcommand{\impl}{$\Rightarrow$\xspace}

%\usepackage{fontspec}

%\newtoggle{draft}\togglefalse{draft}
\newtoggle{draft}\toggletrue{draft}



%\usepackage{hyperref}
%\hypersetup{colorlinks=false, pdfborder={0 0 0}}

%\usepackage{unified-biblatex}

%\let\citew\citealp
%\newcommand{\page}{}
%\bibliography{Bib,bib-abbr,biblio}


%\usepackage{langsci-forest-setup}

%\usepackage{german}%

%\selectlanguage{USenglish}

%\usepackage{abbrev,merkmalstruktur,article-ex,makros.2e,mycommands,eng-date,my-xspace}
%\usepackage{my-gb4e-article}

%\usepackage{soul}

%\usepackage{forest}


\let\citew\cite

\newcommand{\is}[1]{}

\begin{document}
\noindent
{\large\bf Replies to 3rd round of reviews of \emph{Deliminative verbal reduplication in Mandarin Chinese}}\\

\bigskip
\noindent
Dear editors,\\

\noindent
Thank you very much again for the positive reviews. The following text is a repetition of the points the
reviewer raised and an explanation of how we addressed open issues.


\begin{enumerate}

\item p. 6--7, example (11). While example (11b) is acceptable, (11a) does not sound natural. It would be advisable to consult additional native speakers to confirm its acceptability.

\item p. 8, example (14b). In the English translation, it is unclear why the meaning `a bit' is associated with `dangerous' instead of `read.' Clarification or correction may be needed.

\textbf{Answer:} The translation `a bit dangerous' comes from \obj{you3xie1} \obj{wei1xian3} in the original sentence. We changed the translation to `If lectures do not read a bit of books, then it is somewhat dangerous' to avoid confusion.

\item p. 9, lines 1--6. ``In sum, we consider that the reduplication has a similar syntactic distribution as an unreduplicated verb, and the incompatibility of the reduplication with duration and frequency phrases as well as aspect markers other than \textit{le} can be explained semantically, as we will show in the next section.''

The author(s) stated that reduplication has a similar syntactic distribution to an unreduplicated verb. While some examples support this claim, previous studies have documented distributional restrictions in particular environments. It would be useful to briefly comment on potential sources of these different observations.

\textbf{Answer: } In p. 6--8, we discussed some distributional restrictions claimed in previous studies and used examples to show that reduplication has a broader distribution than previously assumed.

\item p. 19, lines 33--35. ``Tense inflections in English such as -ed change neither the category nor the valency of the input verb.''

The author(s) used the English past tense marker ed as an example of a morphological process that does not change the category or valency of the verb. However, in generative syntax, tense is typically introduced by a functional projection, and ed is viewed as a surface- level morphological realization. Therefore, it may be worth considering whether ed is the most appropriate example, especially as an example of a morphological process.

\todo[inline]{Stefan, say something.}

\item p. 20, lines 16--20. Example (38) is intended to demonstrate that \textit{le} can be inserted between V-\obj{gei3}. However, since the verb \obj{ji4} `send' can be used independently without \obj{gei3} (e.g., ``\obj{wo3} \obj{ji4} \obj{le} \obj{yi4} \obj{xiang1} \obj{ping2guo3} (\obj{gei3} \obj{ta1})'' `I sent a box of apples (to her)'), the insertion of \textit{le} may reflect the separability of an optional prepositional phrase, rather than a property of the V-\obj{gei3} unit. Thus, this example may not sufficiently support the claim.

\textbf{Answer: } The \obj{ji4-gei3} example is suggested by another reviewer, as \citet[1282]{Her2006} claims that the V-\obj{gei3} sequence forms a single lexical item. \citegen{Her2006} analysis cannot be elaborated here.

\item p. 21, example (41) and footnote [10]. Footnote [10] is helpful in explaining the purpose of using example (41). The author(s) mentioned that ``This word order constraint is not a problem for the test in (41), as the adverb can, in principle, be interpreted as modifying the second verb'' (line 44--45). However, the use of an adverb to modify the second occurrence in a reduplicated verb appears problematic, as the second occurrence is typically not analyzed as a syntactic head or the main part/verb of the reduplication (as evidenced by the acceptability
of \obj{kan4} \obj{le} \obj{kan4} but not \obj{kan4} \obj{kan4} \obj{le}). This raises concerns about whether (41), as a diagnostic test, presents a reliable example, where the adverb is used to modify the second occurrence.

\textbf{Answer: } non-head can be modified too.

\item p. 24, example (47b). The author(s) suggested that verb reduplications differ from verb classifier phrases based on their separability. While example (46) convincingly demonstrates that a verb classifier phrase can be separated, the claim that reduplications do not allow such separations may be overstated. Corpus data from the BCC show expressions such as (i) ``\obj{deng3} \obj{wo3} \obj{yi} \obj{deng3}'' `wait for me a bit,' (ii) ``\obj{kan4} \obj{shu1} \obj{yi} \obj{kan4}'' `read a book for a bit,' (iii) ``\obj{kan4} \obj{ta1} \obj{yi} \obj{kan4}'' `take a look at him' do occur. This means that separation is possible in certain cases, even if limited.
    
    
\end{enumerate}
    

{\sloppy
\printbibliography[heading=subbibliography,notkeyword=this]
}
\end{document}



%      <!-- Local IspellDict: en_US-w_accents -->

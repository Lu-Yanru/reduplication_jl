%% -*- coding:utf-8 -*-
\documentclass[fleqn,twoside]{article}


\usepackage[
style = langsci-unified,
backend = biber,
natbib = true,
]{biblatex}
\addbibresource{Bib.bib}

\usepackage{xcolor}
\definecolor{lsDOIGray}{cmyk}{0,0,0,0.45}

\usepackage{csquotes}

%% Language
\usepackage[ngerman, english]{babel}
	

\usepackage{microtype}


%%%
%% Tables, lists, columns
% Text in columns: \begin{}{n} \columnbreak \end{multicols}
\usepackage{tabularx}
\usepackage{multicol}
%	\setlength{\columnsep}{.5cm}	


\usepackage{float}

\usepackage{subcaption}



%%%
%% Linguistic graphs
% tikz
\usepackage{tikz}
\usetikzlibrary{patterns, matrix}

% tree
%\usepackage{forest}
% Needed for the "actual forest version"
%\useforestlibrary{linguistics}
%\forestapplylibrarydefaults{linguistics}
\usepackage{langsci-forest-setup}

% avm
\usepackage{langsci-avm}




	
%% hyperref & url
\usepackage[
bookmarksnumbered, %For numbered bookmarks in PDF
hidelinks, %For links without colored borders
%breaklinks=true,
%pdfborder={0 0 0}
]{hyperref}
%\hypersetup{pdfborder={0 0 0}}
% line break url
%\usepackage{breakurl}


%hyphenation
\usepackage{german}
\selectlanguage{USenglish}
\usepackage{hyphenat}


\usepackage{etoolbox}

% examples and glossing
\usepackage{refcount}
\usepackage{xstring}

\usepackage{gb4e}
\exewidth{(222)}
\judgewidth{**}
\usepackage{cgloss}

\usepackage{jambox}

\usepackage{xspace}

% pinyin
\usepackage{xpinyin}
\xpinyinsetup {pysep = {}}


% disambiguate bib names
\newrobustcmd{\disambiguate}[3]{#2~#3}

% refs
\newcommand{\sectref}[1]{Section~\ref{#1}}
\newcommand{\figref}[1]{Figure~\ref{#1}}
\newcommand{\citegen}[2][]{\citeauthor{#2}'s (\citeyear*[#1]{#2})}

\renewcommand{\subref}[1]{\StrRight{\getrefnumber{#1}}{1}}

%%% Linguistics graphs related
%% indexed box
\newlength{\MyetagLength}
\settowidth{\MyetagLength}{{$\scriptstyle 1$}}

%% indexed box
\newcommand{\iboxt}[1]{{%
        \setlength{\fboxsep}{1.25pt}%
        \fbox{$\scriptstyle #1$}%
}}


\newcommand{\ibox}[1]{%
    %  \iboxt{#1}\,%       why this extra space???? 20.02.2018
    \iboxt{#1}%
}

\newcommand{\iboxb}[1]{(\,\iboxt{#1}\,)}

%% empty box
\newcommand{\etag}{\ibox{\rule{0ex}{1.1ex}\hspace{\MyetagLength}}\xspace}

%% type
\newcommand{\type}[1]{{\normalfont\itshape #1\/}}

%% list of phons
\newcommand{\phonliste}[1]{%
    \mbox{%
        $%
        %
        \left\langle \mbox{\normalfont\itshape#1} \right\rangle%
        $%
        %\\[-1.5mm]
    }%
}

\newcommand{\sliste}[1]{%
    \mbox{%
        $\left\langle\mbox{\upshape #1}\right\rangle$}%
}

\newcommand{\impl}{$\Rightarrow$\xspace}

% does not work properly St.Mü. 2023-06-22
\makeatletter
\def\eas{\ifnum\@xnumdepth=0\begin{exe}[(34)]\else\begin{xlist}[iv.]\fi\ex\begin{tabular}[t]{@{}p{\linewidth}@{}}}
\def\zs{\end{tabular}\ifnum\@xnumdepth=1\end{exe}\else\end{xlist}\fi}
\makeatother


%\def\ea{\ifnum\@xnumdepth=0\begin{exe}[(234)]\else\begin{xlist}[iv.]\fi\ex}
%\def\z{\ifnum\@xnumdepth=1\end{exe}\else\end{xlist}\fi}

% pinyin
\newcommand{\obj}[1]{\pinyin[format = \itshape]{#1}}
\newcommand{\objex}[1]{\pinyin{#1}}



\usepackage{todonotes}

\newcommand{\todostefan}[1]{\todo[color=orange!80]{\footnotesize #1}\xspace}
\newcommand{\todosatz}[1]{\todo[color=red!40]{\footnotesize #1}\xspace}

\newcommand{\inlinetodostefan}[1]{\todo[color=green!40,inline]{\footnotesize #1}\xspace}

\newcommand{\inlinetodoopt}[1]{\todo[color=green!40,inline]{\footnotesize #1}\xspace}
\newcommand{\inlinetodoobl}[1]{\todo[color=red!40,inline]{\footnotesize #1}\xspace}

\newcommand{\itd}[1]{\iftoggle{draft}{\inlinetodoobl{#1}}{}}
\newcommand{\itdobl}[1]{\iftoggle{draft}{\inlinetodoobl{#1}}{}}
\newcommand{\itdopt}[1]{\iftoggle{draft}{\inlinetodoopt{#1}}{}}

% for editing, remove later
%\usepackage{xcolor}
\newcommand{\iaddpages}{\iftoggle{draft}{\yel[add pages]{pages}\xspace}}

\newcommand{\addpages}{\iftoggle{draft}{\todostefan{add pages}}\xspace}
\newcommand{\addsource}{\iftoggle{draft}{\todostefan{add source}}\xspace}
\newcommand{\addglosses}{\iftoggle{draft}{\todostefan{add glosses}}\xspace}
%\usepackage{fontspec}

%\newtoggle{draft}\togglefalse{draft}
\newtoggle{draft}\toggletrue{draft}



%\usepackage{hyperref}
%\hypersetup{colorlinks=false, pdfborder={0 0 0}}

%\usepackage{unified-biblatex}

%\let\citew\citealp
%\newcommand{\page}{}
%\bibliography{Bib,bib-abbr,biblio}


%\usepackage{langsci-forest-setup}

%\usepackage{german}%

%\selectlanguage{USenglish}

%\usepackage{abbrev,merkmalstruktur,article-ex,makros.2e,mycommands,eng-date,my-xspace}
%\usepackage{my-gb4e-article}

%\usepackage{soul}

%\usepackage{forest}


\let\citew\cite

\newcommand{\is}[1]{}

\begin{document}
\noindent
{\large\bf Replies to 2nd round of reviews of \emph{An HPSG account for verbal reduplication in Mandarin Chinese}}\\

\bigskip
\noindent
Dear editors,\\

\noindent
Thank you very much again for the positive reviews. The following text is a repetition of the points the
reviewers raised and an explanation of how we addressed open issues.

\section{Reviewer 1}\label{sec:1}

\begin{enumerate}
%%%%% YL
\item p. 2-3, ex. (3). Regarding the verb 来往 \textit{lai-wang}, it actually refers to two different verbs: (1) `come
and go,' a coordinate verb, and (2) `communicate.' In some dictionaries, the second syllable of the
latter (`communicate') is marked as having a neutral tone. The first verb, which has a coordinate
structure, reduplicates as AABB, while the second reduplicates as ABAB. In principle, the verb
meaning `communicate' should not reduplicate as AABB, just as the verb meaning `come and go'
should not reduplicate as ABAB.

%%%%%%%%%%%%%%%%%%%
%%%%%%%%%%%%%%%%%%%
\noindent
\textbf{Answer:} Example (3)  in the manuscript only serves to
 illustrate the different forms of verbal reduplication in Mandarin Chinese
  without discussing their meaning. 
  We included the translation to the examples 
  to make the distinction in meaning clearer, 
  as shown below in (\ref{ex:laiwang}).

\ea\label{ex:laiwang} for disyllabic verbs: \obj{lai2-wang3} come-go `come and go/communicate'
\ea \gll \objex{lai2-wang3-lai2-wang3}\\
come-go-come-go\\ \jambox{ABAB}
\glt `communicate a little bit'

\ex \gll \objex{lai2-wang3-le-lai2-wang3}\\
come-go-\textsc{pfv}-come-go\\ \jambox{AB-\textit{le}-AB}
\glt `communicated a little bit'

\ex \gll \objex{lai2-lai2-wang3-wang3}\\
come-come-go-go\\ \jambox{AABB}
\glt `coming and going'

\z\z

\item p. 7, lines 29--31: ``Finally, \citet[322]{SuiHu2016} claim that a reduplicated verb cannot combine with
a quantized object.''

This statement holds true only for accomplishments, not activities. With accomplishments, verbs
become telic when followed by a quantized object (see \citealt{Verkuyl1972}). This is not the case for
activities. The examples provided in (15) do not contain accomplishment verbs, so their acceptability
is expected. Please double-check the statement by  \citet[322]{SuiHu2016}.

%%%%%%%%%%%%%%%%%%%
%%%%%%%%%%%%%%%%%%%
\noindent
\textbf{Answer:} After double-checking,
 \citet[322]{SuiHu2016} indeed argue that verbal reduplication cannot take a quantized object, 
 claiming that the following example (\ref{ex:qobj}) is ungrammatical.

\ea\label{ex:qobj}
\gll \objex{wo3} \obj{kan4-kan4} \objex{san1} \objex{ben3} \objex{shu1}.\\
I look-look three \textsc{clf} book\\ \jambox{\citep[322]{SuiHu2016}} 
\glt `I will read three books a little bit.'
\z

We can also find examples of reduplicated accomplishments followed by quantized objects in CCL and BCC, as shown in (\ref{ex:qobj1}--\ref{ex:qobj2}) below.
% 吃?吃三种?局,一是吃口味,一是吃氛围,一是吃免费。(CCL 2010s\201x\网络语料\微博\weibo_062.txt)
\ea\label{ex:qobj1}
% 其实他一直想写写两位老师 (CCL 2010s\201x\网络语料\微信公众号\Wechat_123.txt)
\gll \objex{qi2shi2} \objex{ta1} \objex{yi4zhi2} \objex{xiang3} \obj{xie3-xie3} \objex{liang3} \objex{wei4} \objex{lao3shi1} \ldots\\
actually he always want write-write two \textsc{clf} teacher\\ \jambox{(CCL)}
\glt `Actually, he has always wanted to write a bit about two teachers.'

\ex\label{ex:qobj2}% 男人跟男人的沟通比较快,打打球、喝喝两 杯就行了。(BCC 微博)
\gll \objex{nan2ren2} \objex{gen1} \objex{nan2ren2} \objex{de} \objex{gou1tong1} \objex{bi3jiao4} \objex{kuai4}, \ldots\, \obj{he1-he1} \objex{liang3} \objex{bei1} \objex{jiu4} \objex{xing2} \objex{le}.\\
man and man \textsc{de} communication relatively fast {} drink-drink two cup just ok \textsc{ptc}\\ \jambox{(BCC)}
\glt `The communication between men is relatively fast, \ldots\, they just need to drink two cups together.'

% 可我还是壮着胆子想写写一 个同行。 (BCC 人民日报1996)
%农民们就卖卖五样东西,大酱、葱、煎饼、鸡蛋、苞米。(CCL 2010s\201x\网络语料\微信公众号\Wechat_157.txt)
\z

\item p. 14, lines 11--14: ``The reduplication does not seem to cancel the \textit{telos} of achievement'' and related
answer to reviewer1’s comment.

It appears the author(s) misunderstood the reviewer’s comment. What the reviewer meant was that in
certain contexts, achievements can act like activities (as if the \textit{telos} is cancelled), allowing them to be
reduplicated. The reviewer did not suggest that reduplication cancels the \textit{telos}; on the contrary,
reduplication sets a boundary for the event, which is why it is mainly compatible with activities. This
explains why reduplicated verbs are not compatible with ``for-X-time'' expressions, as both
reduplication and ``for-X-time'' expressions set boundaries on the event, making them incompatible
together. In fact, while the base verb can appear with ``for-X-time'' expressions, such as 有一次,我看了一天的书,既累又困 (BCC corpus), but the reduplicated verb cannot: *我看了看一天的书.
The author(s) correctly state on p. 16, lines 21: ``Reduplication is incompatible with expressions
quantifying the duration or extent of the event expressed in the sentence (\citealt[83--84]{Li1998}; \citealt[114--115]{Chen2005})''. Indeed, both reduplication and duration expressions set boundaries on the event.
The author(s) conclude in lines 37--38: ``These tests suggest that reduplicated verbs are indeed
achievements, rather than being used as activities.'' I fail to see why reduplicated verbs should be
considered activities, given their semantics. Also, note that in example (26), the verb \textit{mai} (`buy') is
not an achievement but rather an accomplishment. As for example (26b), \textit{liang tian hou} (`two days
later') is not an ``in X-time'' expression. An ``in X-time'' expression would be something like \textit{he read
three books \ul{in two days}}. Therefore, (26b) is not a good example to illustrate the author(s)' point.
This passage (from line 11 to line 38) must be revised or perhaps eliminated altogether.

%%%%%%%%%%%%%%%%%%%
%%%%%%%%%%%%%%%%%%%
\noindent
\textbf{Answer:} This passage is removed.

\item p. 15, lines 25--28: ``\textit{Le} is compatible with the reduplication, because its dynamicity can relate to not
only the termination or instantiation of an event (a point of change), but also the process of the
situation, just like that of the reduplication (see Section 2.3.1)''. This passage is unclear. Please revise
it for clarity.

%%%%%%%%%%%%%%%%%%%
%%%%%%%%%%%%%%%%%%%
\noindent
\textbf{Answer:} This passage is revised as follows 
and moved before example (27) ((28) in the previous version of the draft) to directly follow the passage before: 
``As shown in Section 2.3.1, reduplication can also express
 dynamicity of both a time point and a time period, 
 just like \obj{le} illustrated here. 
 Therefore, \obj{le} is compatible with reduplication.''


\item P. 16, lines 31--34: ``A reviewer also notes that the reduplication appears frequently in imperative
(example 25a) and conditional sentences (31) as well as causative sentences with \pinyin{rang4/jiao4/shi3}
`let/let/make' (examples 21, 22, 25c).''
This is not exactly what the reviewer stated. The reviewer was not referring to reduplicated verbs in
general. Rather, s/he noted that, as described in various grammars, verbs depicting events not
controlled by an agent (which normally cannot be reduplicated) sometimes undergo reduplication,
especially in imperative, conditional, or causative sentences. I am not sure if reduplication, in general,
tends to appear specifically in these sentence types. It seems to me that it appears frequently in other
contexts as well. The claim ``This explains the general tendency for reduplication to appear in
conditional clauses, imperative, or causative contexts, frequently describing future situations'' (p. 17,
lines 4--6) should be reconsidered.

%%%%%%%%%%%%%%%%%%%
%%%%%%%%%%%%%%%%%%%
\noindent
\textbf{Answer:} This passage is removed. 
Instead, a footnote is added to Section 2.3.2 at the discussion on 
the reduplication of non-volitional verbs in example (18) (example (17) in the previous draft): 
``A reviewer notes that verbs depicting events not controlled 
by an agent sometimes can undergo reduplication, 
especially in imperative, conditional, or causative sentences. 
It is worthwhile to delve into why such sentence structures
 enable reduplication for these verbs. 
 It is also worth noting that the sentences in (18) 
 are not imperative, conditional or causative sentences. 
 This shows that non-volitional verbs can also be reduplicated
  outside of the aforementioned sentence structures''.

\item p. 19, lines 29--31: ``Tense inflections in English such as -ed change neither the category nor the
valency of the input verb''.

Since this is an inflectional affix, it is not appropriate to make this comparison. Inflectional affixes
are obligatory, and they constitute a different phenomenon.

%%%%%%%%%%%%%%%%%%%
%%%%%%%%%%%%%%%%%%%
\noindent
\textbf{Answer:} This example only serves to illustrate our argument 
that a morphological process does not necessarily change 
the category or the valency of the input verb, 
and inflection is a morphological process that 
does not change the category or the valency of the input verb. 
We do not claim reduplication to be either inflection or derivation.


\item p. 20, lines 23--27: ``In any case, since reduplication is not compounding (\citealt[149--150]{Sui2018}; \citealt{GaoEtAl2021} provide psycholinguistic evidence), and the patterns discussed here constitute a different
process than the AABB pattern (see the discussions above, also \citealt[Sec. 4.3]{Deng2013}, \citealt[Sec. 2]{SuiHu2016}, \citealt{Sui2018} and \citealt{Wang2023}), it is not surprising that le occurs at a different position''.
However, le insertion represents a case of lexical integrity violation (meaning the reduplicated verb
does not form a word), unless it is considered as an instance of infixation.

%%%%%%%%%%%%%%%%%%%
%%%%%%%%%%%%%%%%%%%
\noindent
\textbf{Answer:} It is considered an instance of infixation.

\item p. 21, examples in (44): It would be better to use a pronoun as the object in (44a), as it is undeniable
that pronoun objects appear in that position.

%%%%%%%%%%%%%%%%%%%
%%%%%%%%%%%%%%%%%%%
\noindent
\textbf{Answer:} This part uses the example provided by Reviewer 2. We decided to keep (44a) as it is to maintain the consistency among other minimal pairs. Our native speaker informants consider (44a) to be grammatical, and we can also find similar sentences in corpora.

\item p. 21, lines 21--24: ``Cross-linguistically, verbal reduplication in Mandarin Chinese patterns more with
morphological reduplication (below as reduplication) in other languages than syntactic reduplication
(below as repetition; \citealt[31]{Gil2005}, \citealt[1--2]{Forza2016}).''

Be careful: stating that reduplication is a syntactic phenomenon does not mean it is an instance of
repetition. Arguing that diminishing reduplication is a syntactic phenomenon affecting the vP domain
and proposing a syntactic analysis (as in \citealt{Arcodiaetal2014} and \citealt{BascianoMelloni2017}) does not imply that it is syntactic repetition; they are two different phenomena.

%%%%%%%%%%%%%%%%%%%
%%%%%%%%%%%%%%%%%%%
\noindent
\textbf{Answer:} We do not claim that the syntactic analyses consider reduplication to be repetition. We added a footnote to clarify this point.


\item p.23, lines 29--31: ``Third, idioms (52a) lose their idiomatic meaning when used with verbal classifiers
(52b) but maintain their idiomatic meaning with reduplications (52c) \citep[230--231]{YangWei2017}.''
It would be useful to specify that this applies except when 一下 \textit{yixia} is used, which indicates short
duration.

%%%%%%%%%%%%%%%%%%%
%%%%%%%%%%%%%%%%%%%
\noindent
\textbf{Answer:} We modified footnote 15 (footnote 11 in the previous draft) as follows to make this point clearer: ``This applies except when \obj{yi2 xia4} `once' is used. Because of this exception, one reviewer suggests that the lost of the idiomatic interpretation may be attributed to the use of the numeral \obj{san1} `three' in (50b) (52b in the previous draft). We suggest that it is not \obj{san1} \obj{xia4} `three times' that is special, 
but it is \obj{yi2} \obj{xia4} `once' that is special.
We can replace \obj{san1} `three' with any number above two,
and the distinction still exist.
But \obj{yi2} \obj{xia4} `once' has acquired a duration reading `for a little while' that is not available to the other event quantifiers formed by \obj{xia4}, which only have the `for X times' interpretation (\citealt[77]{Deng2013}, \citealt[16]{Zhang2000}).
We think \obj{liang3} \obj{xia4} `twice' is following this tendency, too.
In this case, it is easy to interpret \obj{yi2} \obj{xia4} and \obj{liang3} \obj{xia4} not as referring to the actual number of action taking place,
but as duration adverbials as a whole,
thus differing them from ``actual'' verbal classifier phrases with other numerals''.


\item p. 25, lines 37--41: ``This suggests that the reduced acceptability of reduplicated achievement and
stative verbs is semantic rather than structural. Their use is possible in specific contexts and should
not be ruled out syntactically. Consequently, this proposal does not seem to offer an appropriate
account for reduplication.''

This depends on whether such verbs can, in certain circumstances, behave differently, acquiring some
aspectual properties of other verb classes. This can happen also with adjectives. For example, the
adjective 高兴 `happy', which typically reduplicates as 高高兴兴 (AABB, adjectival increasing
reduplication), can under certain conditions describe a process, acquiring verbal features, and
reduplicate as ABAB: e.g., 让他高兴高兴 (BCC corpus). It is not the case that all adjectives can
undergo diminishing ABAB reduplication.

%%%%%%%%%%%%%%%%%%%
%%%%%%%%%%%%%%%%%%%
\noindent
\textbf{Answer:} As we explained in our previous reply and also in Section 2.3, 
verb classes are determined solely based on 
the inherent features of the verb itself
(\citealt[52]{XiaoMcEnery2004}).
This means that even though the same verb may express different aspectual properties in different contexts,
its verb class remains the same.
The change in aspectual properties can be attributable to other components at the sentential level, such as the use of reduplication.
We also showed in Section 2.3.2 many examples of achievement and stative verbs being reduplicated.
We argue that all verbs are \textit{syntactically} compatible with reduplication, 
and therefore, this possibility should not be \textit{syntactically} ruled out.
We do recognize that reduplication requires specific situational information, 
namely a situation that is compatible with its deliminative meaning,
but we believe this is a \textit{semantic} constraint.


\item p. 26, line 26: \textit{zhe} is durative rather than progressive.


%%%%%%%%%%%%%%%%%%%
%%%%%%%%%%%%%%%%%%%
\noindent
\textbf{Answer:} We corrected this.

\item P. 26, lines 27--28: ``He observes that \textit{zhe} `PROG', \textit{le} `PFV' and \pinyin{wan2} `COMPL' necessarily occur
with additional information about the event denoted by the sentence (53), while \pinyin{zai4} `dur' and \pinyin{guo4}
`EXP' can occur without further information (54).''

This passage is not completely clear. Please revise and clarify it. Also, the aspect marker 过 \textit{guo}
should be toneless.

%%%%%%%%%%%%%%%%%%%
%%%%%%%%%%%%%%%%%%%
\noindent
\textbf{Answer:} We modified the passage as follows and added more examples to make it clearer: 
``He observes the so-called incompleteness effect, namely that a minimal sentence, 
which only contains a verb marked by \obj{zhe} `\textsc{dur}',
\obj{le} `\textsc{pfv} or \obj{wan2} \textsc{compl} 
and its arguments, 
seems incomplete without further sentential elements 
such as the sentence final particle \obj{le} 
or a temporal adverbial like \obj{gang1cai2} `just now' (51) . 
In contrast, a minimal sentence with a verb marked by  \obj{zai4} `prog' or \obj{guo}
`\textsc{exp}'  and its arguments can stand alone (52).''

We changed the tone of \obj{guo} to be toneless.


\item p. 27, lines 37--40: Please explain the mismatch more thoroughly.

%%%%%%%%%%%%%%%%%%%
%%%%%%%%%%%%%%%%%%%
\noindent
\textbf{Answer:} This passage is modified as follows: 
``This analysis would result in a mismatch between syntax and semantics, 
in the sense that the aspect markers that belong to the same semantic group do not occur in the same syntactic position. 
Even though \obj{le}, \obj{guo} and reduplication all mark perfective aspects (Section 2.3.3, \citealt{Dai1997, XiaoMcEnery2004}),
\obj{guo} is situated under Asp$_1$, while \obj{le} and reduplication are under Asp$_2$.
Similarly, \obj{zai4} and \obj{zhe} are both imperfective aspects but also occur in different syntactic positions.
The former is under Asp$_1$ while the latter is under Asp$_2$.''
%However, on the surface structure, both \obj{le} and \obj{guo} occur in the same position, directly following the verb.
%If we assume that the aspect markers are base-generated in different positions 
%and then move to the same position following the verb, 
%then there needs to be extra explanations given as to why \obj{zai4} occurs before the verb''.
%\todo[inline]{YL to do: check}

\item p. 29, line 29: Indeed, AAB reduplication is basically a reduplication of a monosyllabic verb followed
by an object, so it is productive as AA.

%%%%%%%%%%%%%%%%%%%
%%%%%%%%%%%%%%%%%%%
\noindent
\textbf{Answer:} Yes, and this is also addressed in our analysis in Section 4, p. 40.
%\todo[inline]{YL to do: add page number}

\end{enumerate}

\section{Reviewer 2}\label{sec:2}

\begin{enumerate}
    %%%%% YL
    \item Example (11) is not entirely convincing in its attempt to challenge \citegen{SuiHu2016} proposal. Sui and Hu suggest that verb reduplication can occur in relative clauses when licensed by modal verbs, as these verbs check the relevant features enabling reduplication. Furthermore, they note that certain verbs, such as 打算 \textit{dǎsuàn} (`plan'), 让 \textit{ràng} (`let'), and 叫 \textit{jiào} (`ask') can also license verb
    reduplication. In example (11), the verb 苦留 \textit{kǔliú} (`trying hard to make stay')
    appears to function in a similar licensing role. If this interpretation is accurate, the example does not effectively challenge Su and Hu’s argument. It may be helpful for the authors to clarify why 苦留 \textit{kǔliú} is distinct from other licensing verbs or to provide additional examples that counter Sui and Hu's claims.
    
    %%%%%%%%%%%%%%%%%%%
    %%%%%%%%%%%%%%%%%%%
\noindent
\textbf{Answer:} We modified this passage to include \citegen{SuiHu2016} observation 
that reduplication is possible in relative clauses with 
verbs with modal or mood meanings as follows: ``\citet[319, 332]{SuiHu2016} claim that reduplication cannot appear in a relative clause without a modal verb or a verb with a modal or mood meaning, such as \obj{da3suan4} `plan', \obj{rang4} `let' and \obj{jiao4} `ask'''.
    
However, we also found examples such as the following ones in (\ref{ex:rel1}--\ref{ex:rel2}), 
where reduplication occurs in relative clauses without modal or mood verbs.
    
    % 随便看看的粉丝	群 (BCC 微博)
    
\ea\label{ex:rel1}
    % 你看看的政务 微博2010年9月2号、10月20号和10月21号,应付转发了几十条微博!(BCC 微博)
    \gll [[\objex{ni3} \obj{kan4-kan} \objex{de}] \objex{zheng4wu4}-\objex{wei1bo2}] \ldots\, \objex{zhuan3fa1-le} \objex{ji3}-\objex{shi2} \objex{tiao2} \objex{wei1bo2}!\\
    you look-look \textsc{de} government.affairs-Weibo {} repost-\textsc{pfv} several-ten \textsc{clf} Weibo\\ \jambox{(BCC)}
    \glt `The government affairs Weibo account that you look at \ldots\, reposted dozens of Weibo posts.'
    
    % 大年初一参差光谷,初二去,初三去看看的店 (BCC 微博)
    
    % 像是习惯路过了不进去看看的店 。(BCC 微博)
    
    % 刚在人人看到班上那个喜欢用看政治来装逼,时而偷进我围脖来看看的女人 ,说不定这条你看操…… (BCC 微博)
    
    \ex\label{ex:rel2}
    % 大厅里坐着一些年轻人,他们是逛进来听听的爵士 乐迷。(BCC 为了告别的聚会 A:米兰·昆德拉 Y:UN)
    \gll \objex{ta1men} \objex{shi4} [[\objex{guang4} \objex{jin4lai2} \obj{ting1-ting} \objex{de}] \objex{jue2shi4yue4-mi2}].\\
    they be wander enter listen-listen \textsc{de} jazz-fan\\ \jambox{(BCC)}
    \glt `They are jazz fans who wandered in to have a listen.'
 
 \z  
    
    \item Using example (14), the authors suggest that negation can modify verb reduplication and challenge prior studies that emphasize constraints on their co-occurrence. However, it is important to note that example (14b) is situated in an interrogative context. As \citet{SuiHu2016} highlight, negation modifying verb reduplication is not acceptable in \textit{affirmative} sentences. Similar constraints are reported in earlier literature, such as \citet{Shen1995} and \citet{Liu1983} (see \citealt{Xie2018} for a review).
    For instance, the sentence 你不复习复习功课 \textit{Nǐ bù fùxí fùxí gōngkè} (`You do not review your lessons’) is not acceptable, whereas 你不复习复习功课吗? \textit{Nǐ
    bù fùxí fùxí gōngkè ma}? (`Aren't you going to review your lessons?') is acceptable. This suggests that reduplication with negation is permitted in certain contexts, such as interrogative or emphatic sentences, but not in other contexts, like affirmative statements. The authors should consider addressing these contextual nuances to strengthen their analysis.
    
    In summary, while Section 2.2 effectively demonstrates that the distribution of reduplicated verbs is broader than previously reported, it is essential to acknowledge that some examples represent special cases.
    
    %%%%%%%%%%%%%%%%%%%
    %%%%%%%%%%%%%%%%%%%
    \noindent
    \textbf{Answer:} We also found the following examples with reduplication embedded under negation in affirmative sentences (\ref{ex:neg1}--\ref{ex:neg3}).
    
\ea\label{ex:neg1}
    % 你们只知道责备人家,全不想想自己。(CCL 1910s\1916\文学\小说笔记\1916_5 向恺然 留东外史.txt)
\gll \objex{ni3-men} \objex{zhi3} \objex{zhi1dao4} \objex{ze2bei4} \objex{ren2jia1}, \objex{quan2} \objex{bu4} \obj{xiang3-xiang3} \objex{zi4ji3}.\\
you-\textsc{pl} only know blame others at.all not think-think self\\ \jambox{(CCL)}
\glt `You only blame others, not thinking about yourselves at all.'
        
    % 曾孝一听,登时无名火起,说怎么着,他欺侮我的孩子,可以,也不打听打听,我们家人多,非先去把他家拆个土平,把周家小老婆子,拉在十字路口儿,非撕了他不行。(CCL 1910s\191x\文学\191x_estimated 爱新觉罗·勋锐 评讲聊斋.txt)
    
    % 吊膀子是你的生性使然,你自己曾对我说过,你见了少年生得好的女人,若不转转念头,你心中便像有什么事放不下似的。(CCL 1910s\1916\文学\小说笔记\1916_5 向恺然 留东外史.txt)
    
    \ex\label{ex:neg2}
    % 所谓职业的读书者,譬如学生因为升学,教员因为要讲功课,不翻翻书,就有些危险的就是。(CCL 1920s\1928\文学\1928 鲁迅 而已集.txt)
    \gll \objex{jiao4yuan2} \ldots\, \objex{bu4} \obj{fan1-fan1} \objex{shu1}, \objex{jiu4} \objex{you3xie1} \objex{wei1xian3} \objex{de} \objex{jiu4} \objex{shi4}.\\
    lecturer {} not flip-flip book just somewhat dangerous \textsc{de} just be\\ \jambox{(CCL)}
    \glt `If lecturers do not read books, then it is a bit dangerous.'
    
    \ex\label{ex:neg3}
    % 花儿匠简直不管事了,你看,什么东西也不收拾收拾。(CCL 	1930s\1932\文学\1932 张恨水 金粉世家.txt)
    \gll \objex{hua1erjiang4} \objex{jian3zhi2} \objex{bu4} \objex{guan3} \objex{shi4} \objex{le}, \ldots\, \objex{shen2me} \objex{dong1xi} \objex{ye3} \objex{bu4} \obj{shou1shi-shou1shi}.\\
    florist at.all not care thing \textsc{ptc} {} what thing also not tidy-tidy\\ \jambox{(CCL)}
    \glt `The florist did not care about anything at all, did not tidy up anything.'
\z
    
    We recognize these uses require specific context information.
    
    
    \item On Page 20, the authors use the modification test \citep{Dai1992} to argue that verb reduplication behaves like words rather than phrases. Example (43) illustrates that adverbs cannot be inserted within a verb reduplication. However, this argument could be strengthened by addressing the following point: The rationale for inserting the adverb within the reduplication is not very clear. Even for typical verb phrases like 开门 \textit{kāi mén} (`open the door'), adverbs cannot be inserted between the verb and its
    object (e.g., 开偷偷地门 \textit{kāi tōutōu de mén} `open the door secretly' is ungrammatical).
    This suggests that the inability to insert an adverb is not unique to reduplication and may not prove its word-like property.
    
    %%%%%%%%%%%%%%%%%%%
    %%%%%%%%%%%%%%%%%%%
    \noindent
    \textbf{Answer:} The rationale for inserting an adverb in-between the reduplication is that a verb can be modified by an adverb, hence the modification test.
    In Mandarin Chinese, almost all adverbials are obligatorily pre-verbal \citep[50]{Ernst2014}, as illustrated in the example below (\ref{ex:adv}).
    \ea\label{ex:adv}
    \ea[]{\gll \objex{ta1} \objex{qiao1qiao1de} \objex{zou3-le}\\
    he quietly go-\textsc{pfv}.\\
    \glt `He quietly went away.'
    }
    \ex[*]{\gll \objex{ta1} \objex{zou3-le} \objex{qiao1qiao1de}.\\
    he go-\textsc{pfv} quietly\\
    }
    \z\z
    Therefore, the general explanation for the ungrammaticality of \obj{kai1} \obj{tou1tou1de} \obj{men2} `open secretly door' is that the adverb cannot occur after the verb.
    This is not a problem for the test in (41) (43 in the previous draft), as the adverb can, in principle, be interpreted as modifying the second verb.
  
  \citet[50]{Ernst2014} mentions that the only possible post-verbal adverbials are participant PPs (\textit{with somebody}),
  manner or resultative V-de constructions as well as 
  duration and frequency expressions.
  (\ref{ex:VPinsert}) shows a case of inserting a frequency expression into a VP.
  This suggests that it is, in principle, possible to insert adverbials into a VP.
  
  \ea\label{ex:VPinsert}
    \gll \objex{kai1}  \objex{san3} \objex{ci4} \objex{men2}\\
    open three time door\\
    \glt `open the door three times'
  \z
  
    
    \item The application of MRS within the HPSG framework represents a central focus of this paper. It might be helpful to explicitly acknowledge that the integration of MRS into the HPSG framework has been explored in prior research, such as \citet{Copestakeetal2005}. Addressing this would situate the current analysis more clearly within the broader theoretical landscape and strengthen its overall credibility.
    
    %%%%%%%%%%%%%%%%%%%
    %%%%%%%%%%%%%%%%%%%
    \noindent
    \textbf{Answer:} We added references and particularly mentioned that MRS is also used in other
    theoretical work and in computational implementations like the Grammar Matrix and CoreGram and
    we pointed the reader to the HPSG handbook, which has a section on MRS in the chapter on
    sematnics. We also point to Müller's forthcoming textbook that contains a chapter introducing
    MRS.
    
    
\end{enumerate}

\section{Shortening}
Since our manuscript exceeded the word limit of the journal, we moved the following parts into the appendices:
\begin{itemize}
    \item The analysis based on \citet{Tsai2008} (p. 25--27 in the last version of the manuscript), because \citet{YangWei2017} did not spell out the exact analysis, but only suggested that reduplication can be analyzed as an aspect marker in the aspectual system proposed by \citet{Tsai2008}. The analysis presented here is inferef by us based on \citet{Tsai2008}.
    
    \item The possible historical and phonological explanations for only \obj{yi} and \obj{le} appearing in between reduplication (footnote 17 in the last version of the manuscript).
\end{itemize}
    

{\sloppy
\printbibliography[heading=subbibliography,notkeyword=this]
}
\end{document}



%      <!-- Local IspellDict: en_US-w_accents -->

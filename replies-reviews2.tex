%% -*- coding:utf-8 -*-
\documentclass[fleqn,twoside]{article}


\usepackage[
style = langsci-unified,
backend = biber,
natbib = true,
]{biblatex}
\addbibresource{Bib.bib}

\usepackage{xcolor}
\definecolor{lsDOIGray}{cmyk}{0,0,0,0.45}

\usepackage{csquotes}

%% Language
\usepackage[ngerman, english]{babel}
	

\usepackage{microtype}


%%%
%% Tables, lists, columns
% Text in columns: \begin{}{n} \columnbreak \end{multicols}
\usepackage{tabularx}
\usepackage{multicol}
%	\setlength{\columnsep}{.5cm}	


\usepackage{float}

\usepackage{subcaption}



%%%
%% Linguistic graphs
% tikz
\usepackage{tikz}
\usetikzlibrary{patterns, matrix}

% tree
%\usepackage{forest}
% Needed for the "actual forest version"
%\useforestlibrary{linguistics}
%\forestapplylibrarydefaults{linguistics}
\usepackage{langsci-forest-setup}

% avm
\usepackage{langsci-avm}




	
%% hyperref & url
\usepackage[
bookmarksnumbered, %For numbered bookmarks in PDF
hidelinks, %For links without colored borders
%breaklinks=true,
%pdfborder={0 0 0}
]{hyperref}
%\hypersetup{pdfborder={0 0 0}}
% line break url
%\usepackage{breakurl}


%hyphenation
\usepackage{german}
\selectlanguage{USenglish}
\usepackage{hyphenat}


\usepackage{etoolbox}

% examples and glossing
\usepackage{refcount}
\usepackage{xstring}

\usepackage{gb4e}
\exewidth{(222)}
\judgewidth{**}

\usepackage{jambox}


% disambiguate bib names
\newrobustcmd{\disambiguate}[3]{#2~#3}

% refs
\newcommand{\sectref}[1]{Section~\ref{#1}}
\newcommand{\figref}[1]{Figure~\ref{#1}}
\newcommand{\cioktegen}[2][]{\citeauthor{#2}'s (\citeyear*[#1]{#2})}

%%% Linguistics graphs related
%% indexed box
\newlength{\MyetagLength}
\settowidth{\MyetagLength}{{$\scriptstyle 1$}}

%% indexed box
\newcommand{\iboxt}[1]{{%
        \setlength{\fboxsep}{1.25pt}%
        \fbox{$\scriptstyle #1$}%
}}


\newcommand{\ibox}[1]{%
    %  \iboxt{#1}\,%       why this extra space???? 20.02.2018
    \iboxt{#1}%
}

\newcommand{\iboxb}[1]{(\,\iboxt{#1}\,)}

%% empty box
\newcommand{\etag}{\ibox{\rule{0ex}{1.1ex}\hspace{\MyetagLength}}\xspace}

%% type
\newcommand{\type}[1]{{\normalfont\itshape #1\/}}

%% list of phons
\newcommand{\phonliste}[1]{%
    \mbox{%
        $%
        %
        \left\langle \mbox{\normalfont\itshape#1} \right\rangle%
        $%
        %\\[-1.5mm]
    }%
}

\newcommand{\sliste}[1]{%
    \mbox{%
        $\left\langle\mbox{\upshape #1}\right\rangle$}%
}

\newcommand{\impl}{$\Rightarrow$\xspace}

%\usepackage{fontspec}

%\newtoggle{draft}\togglefalse{draft}
\newtoggle{draft}\toggletrue{draft}



%\usepackage{hyperref}
%\hypersetup{colorlinks=false, pdfborder={0 0 0}}

%\usepackage{unified-biblatex}

%\let\citew\citealp
%\newcommand{\page}{}
%\bibliography{Bib,bib-abbr,biblio}


%\usepackage{langsci-forest-setup}

%\usepackage{german}%

%\selectlanguage{USenglish}

%\usepackage{abbrev,merkmalstruktur,article-ex,makros.2e,mycommands,eng-date,my-xspace}
%\usepackage{my-gb4e-article}

%\usepackage{soul}

%\usepackage{forest}


\let\citew\cite

\newcommand{\is}[1]{}

\begin{document}
\noindent
{\large\bf Replies to 2nd round of reviews of \emph{An HPSG account for verbal reduplication in Mandarin Chinese}}

\noindent
Dear editors,

Thank you very much again for the positive reviews. The following text is a repetition of the points the
reviewers raised and an explanation of how we addressed open issues.

\section{Reviewer 1}\label{sec:1}

\begin{enumerate}
%%%%% YL
\item p. 2-3, ex. (3). Regarding the verb 来往 \textit{lai-wang}, it actually refers to two different verbs: (1) `come
and go,' a coordinate verb, and (2) `communicate.' In some dictionaries, the second syllable of the
latter (`communicate') is marked as having a neutral tone. The first verb, which has a coordinate
structure, reduplicates as AABB, while the second reduplicates as ABAB. In principle, the verb
meaning `communicate' should not reduplicate as AABB, just as the verb meaning `come and go'
should not reduplicate as ABAB.

%%%%%%%%%%%%%%%%%%%
%%%%%%%%%%%%%%%%%%%
\noindent
\textbf{Answer:}

\item p. 7, lines 29--31: ``Finally, \citet[322]{SuiHu2016} claim that a reduplicated verb cannot combine with
a quantized object.''

This statement holds true only for accomplishments, not activities. With accomplishments, verbs
become telic when followed by a quantized object (see \citealt{Verkuyl1972}). This is not the case for
activities. The examples provided in (15) do not contain accomplishment verbs, so their acceptability
is expected. Please double-check the statement by  \citet[322]{SuiHu2016}.

%%%%%%%%%%%%%%%%%%%
%%%%%%%%%%%%%%%%%%%
\noindent
\textbf{Answer:}

\item p. 14, lines 11--14: ``The reduplication does not seem to cancel the \textit{telos} of achievement'' and related
answer to reviewer1’s comment.

It appears the author(s) misunderstood the reviewer’s comment. What the reviewer meant was that in
certain contexts, achievements can act like activities (as if the \textit{telos} is cancelled), allowing them to be
reduplicated. The reviewer did not suggest that reduplication cancels the \textit{telos}; on the contrary,
reduplication sets a boundary for the event, which is why it is mainly compatible with activities. This
explains why reduplicated verbs are not compatible with ``for-X-time'' expressions, as both
reduplication and ``for-X-time'' expressions set boundaries on the event, making them incompatible
together. In fact, while the base verb can appear with ``for-X-time'' expressions, such as 有一次,我看了一天的书,既累又困 (BCC corpus), but the reduplicated verb cannot: *我看了看一天的书.
The author(s) correctly state on p. 16, lines 21: ``Reduplication is incompatible with expressions
quantifying the duration or extent of the event expressed in the sentence (\citealt[83--84]{Li1998}; \citealt[114--115]{Chen2005})''. Indeed, both reduplication and duration expressions set boundaries on the event.
The author(s) conclude in lines 37--38: ``These tests suggest that reduplicated verbs are indeed
achievements, rather than being used as activities.'' I fail to see why reduplicated verbs should be
considered activities, given their semantics. Also, note that in example (26), the verb \textit{mai} (`buy') is
not an achievement but rather an accomplishment. As for example (26b), \textit{liang tian hou} (`two days
later') is not an ``in X-time'' expression. An ``in X-time'' expression would be something like \textit{he read
three books \ul{in two days}}. Therefore, (26b) is not a good example to illustrate the author(s)' point.
This passage (from line 11 to line 38) must be revised or perhaps eliminated altogether.

%%%%%%%%%%%%%%%%%%%
%%%%%%%%%%%%%%%%%%%
\noindent
\textbf{Answer:}

\item p. 15, lines 25--28: ``\textit{Le} is compatible with the reduplication, because its dynamicity can relate to not
only the termination or instantiation of an event (a point of change), but also the process of the
situation, just like that of the reduplication (see Section 2.3.1)''. This passage is unclear. Please revise
it for clarity.

%%%%%%%%%%%%%%%%%%%
%%%%%%%%%%%%%%%%%%%
\noindent
\textbf{Answer:}

\item P. 16, lines 31--34: ``A reviewer also notes that the reduplication appears frequently in imperative
(example 25a) and conditional sentences (31) as well as causative sentences with \pinyin{rang4/jiao4/shi3}
`let/let/make' (examples 21, 22, 25c).''
This is not exactly what the reviewer stated. The reviewer was not referring to reduplicated verbs in
general. Rather, s/he noted that, as described in various grammars, verbs depicting events not
controlled by an agent (which normally cannot be reduplicated) sometimes undergo reduplication,
especially in imperative, conditional, or causative sentences. I am not sure if reduplication, in general,
tends to appear specifically in these sentence types. It seems to me that it appears frequently in other
contexts as well. The claim ``This explains the general tendency for reduplication to appear in
conditional clauses, imperative, or causative contexts, frequently describing future situations'' (p. 17,
lines 4--6) should be reconsidered.

%%%%%%%%%%%%%%%%%%%
%%%%%%%%%%%%%%%%%%%
\noindent
\textbf{Answer:}

\item p. 19, lines 29--31: ``Tense inflections in English such as -ed change neither the category nor the
valency of the input verb''.

Since this is an inflectional affix, it is not appropriate to make this comparison. Inflectional affixes
are obligatory, and they constitute a different phenomenon.

%%%%%%%%%%%%%%%%%%%
%%%%%%%%%%%%%%%%%%%
\noindent
\textbf{Answer:}

\item p. 20, lines 23--27: ``In any case, since reduplication is not compounding (\citealt[149--150]{Sui2018}; \citealt{GaoEtAl2021} provide psycholinguistic evidence), and the patterns discussed here constitute a different
process than the AABB pattern (see the discussions above, also \citealt[Sec. 4.3]{Deng2013}, \citealt[Sec. 2]{SuiHu2016}, \citealt{Sui2018} and \citealt{Wang2023}), it is not surprising that le occurs at a different position''.
However, le insertion represents a case of lexical integrity violation (meaning the reduplicated verb
does not form a word), unless it is considered as an instance of infixation.

%%%%%%%%%%%%%%%%%%%
%%%%%%%%%%%%%%%%%%%
\noindent
\textbf{Answer:}

\item p. 21, examples in (44): It would be better to use a pronoun as the object in (44a), as it is undeniable
that pronoun objects appear in that position.

%%%%%%%%%%%%%%%%%%%
%%%%%%%%%%%%%%%%%%%
\noindent
\textbf{Answer:}

\item p. 21, lines 21--24: ``Cross-linguistically, verbal reduplication in Mandarin Chinese patterns more with
morphological reduplication (below as reduplication) in other languages than syntactic reduplication
(below as repetition; \citealt[31]{Gil2005}, \citealt[1--2]{Forza2016}).''

Be careful: stating that reduplication is a syntactic phenomenon does not mean it is an instance of
repetition. Arguing that diminishing reduplication is a syntactic phenomenon affecting the vP domain
and proposing a syntactic analysis (as in \citealt{Arcodiaetal2014} and \citealt{BascianoMelloni2017}) does not imply that it is syntactic repetition; they are two different phenomena.

%%%%%%%%%%%%%%%%%%%
%%%%%%%%%%%%%%%%%%%
\noindent
\textbf{Answer:}

\item p.23, lines 29--31: ``Third, idioms (52a) lose their idiomatic meaning when used with verbal classifiers
(52b) but maintain their idiomatic meaning with reduplications (52c) \citep[230--231]{YangWei2017}.''
It would be useful to specify that this applies except when 一下 \textit{yixia} is used, which indicates short
duration.

%%%%%%%%%%%%%%%%%%%
%%%%%%%%%%%%%%%%%%%
\noindent
\textbf{Answer:}

\item p. 25, lines 37--41: ``This suggests that the reduced acceptability of reduplicated achievement and
stative verbs is semantic rather than structural. Their use is possible in specific contexts and should
not be ruled out syntactically. Consequently, this proposal does not seem to offer an appropriate
account for reduplication.''

This depends on whether such verbs can, in certain circumstances, behave differently, acquiring some
aspectual properties of other verb classes. This can happen also with adjctives. For example, the
adjective 高兴 `happy', which typically reduplicates as 高高兴兴 (AABB, adjectival increasing
reduplication), can under certain conditions describe a process, acquiring verbal features, and
reduplicate as ABAB: e.g., 让他高兴高兴 (BCC corpus). It is not the case that all adjectives can
undergo diminishing ABAB reduplication.

%%%%%%%%%%%%%%%%%%%
%%%%%%%%%%%%%%%%%%%
\noindent
\textbf{Answer:}

\item p. 26, line 26: \textit{zhe} is durative rather than progressive.

%%%%%%%%%%%%%%%%%%%
%%%%%%%%%%%%%%%%%%%
\noindent
\textbf{Answer:}

\item P. 26, lines 27--28: ``He observes that \textit{zhe} `PROG', \textit{le} `PFV' and \pinyin{wan2} `COMPL' necessarily occur
with additional information about the event denoted by the sentence (53), while \pinyin{zai4} `dur' and \pinyin{guo4}
`EXP' can occur without further information (54).''

This passage is not completely clear. Please revise and clarify it. Also, the aspect marker 过 \textit{guo}
should be toneless.

%%%%%%%%%%%%%%%%%%%
%%%%%%%%%%%%%%%%%%%
\noindent
\textbf{Answer:}

\item p. 27, lines 37--40: Pleasc explain the mismatch more thoroughly.

%%%%%%%%%%%%%%%%%%%
%%%%%%%%%%%%%%%%%%%
\noindent
\textbf{Answer:}

\item p. 29, line 29: Indeed, AAB reduplication is basically a reduplication of a monosyllabic verb followed
by an object, so it is productive as AA.

%%%%%%%%%%%%%%%%%%%
%%%%%%%%%%%%%%%%%%%
\noindent
\textbf{Answer:}


\end{enumerate}

\section{Reviewer 2}\label{sec:2}

\begin{enumerate}
    %%%%% YL
    \item Example (11) is not entirely convincing in its attempt to challenge \citegen{SuiHu2016} proposal. Sui and Hu suggest that verb reduplication can occur in relative clauses when licensed by modal verbs, as these verbs check the relevant features enabling reduplication. Furthermore, they note that certain verbs, such as 打算 \textit{dǎsuàn} (`plan'), 让 \textit{ràng} (`let'), and 叫 \textit{jiào} (`ask') can also license verb
    reduplication. In example (11), the verb 苦留 \textit{kǔliú} (`trying hard to make stay')
    appears to function in a similar licensing role. If this interpretation is accurate, the example does not effectively challenge Su and Hu’s argument. It may be helpful for the authors to clarify why 苦留 \textit{kǔliú} is distinct from other licensing verbs or to provide additional examples that counter Sui and Hu's claims.
    
    %%%%%%%%%%%%%%%%%%%
    %%%%%%%%%%%%%%%%%%%
    \noindent
    \textbf{Answer:}
    
    \item Using example (14), the authors suggest that negation can modify verb reduplication and challenge prior studies that emphasize constraints on their co-occurrence. However, it is important to note that example (14b) is situated in an interrogative context. As \citet{SuiHu2016} highlight, negation modifying verb reduplication is not acceptable in \textit{affirmative} sentences. Similar constraints are reported in earlier literature, such as \citet{Shen1995} and \citet{Liu1983} (see \citealt{Xie2018} for a review).
    For instance, the sentence 你不复习复习功课 \textit{Nǐ bù fùxí fùxí gōngkè} (`You do not review your lessons’) is not acceptable, whereas 你不复习复习功课吗? \textit{Nǐ
    bù fùxí fùxí gōngkè ma}? (`Aren't you going to review your lessons?') is acceptable. This suggests that reduplication with negation is permitted in certain contexts, such as interrogative or emphatic sentences, but not in other contexts, like affirmative statements. The authors should consider addressing these contextual nuances to strengthen their analysis.
    
    In summary, while Section 2.2 effectively demonstrates that the distribution of reduplicated verbs is broader than previously reported, it is essential to acknowledge that some examples represent special cases.
    
    %%%%%%%%%%%%%%%%%%%
    %%%%%%%%%%%%%%%%%%%
    \noindent
    \textbf{Answer:}
    
    \item On Page 20, the authors use the modification test (\citep{Dai1992} to argue that verb reduplication behaves like words rather than phrases. Example (43) illustrates that adverbs cannot be inserted within a verb reduplication. However, this argument could be strengthened by addressing the following point: The rationale for inserting the adverb within the reduplication is not very clear. Even for typical verb phrases like 开门 \textit{kāi mén} (`open the door'), adverbs cannot be inserted between the verb and its
    object (e.g., 开偷偷地门 \textit{kāi tōutōu de mén} `open the door secretly' is ungrammatical).
    This suggests that the inability to insert an adverb is not unique to reduplication and may not prove its word-like property.
    
    %%%%%%%%%%%%%%%%%%%
    %%%%%%%%%%%%%%%%%%%
    \noindent
    \textbf{Answer:}
    
    \item The application of MRS within the HPSG framework represents a central focus of this paper. It might be helpful to explicitly acknowledge that the integration of MRS into the HPSG framework has been explored in prior research, such as \citet{Copestakeetal2005}. Addressing this would situate the current analysis more clearly within the broader theoretical landscape and strengthen its overall credibility.
    
    %%%%%%%%%%%%%%%%%%%
    %%%%%%%%%%%%%%%%%%%
    \noindent
    \textbf{Answer:}
    \todo[inline]{Stefan, say something}
    
    
\end{enumerate}

    

{\sloppy
\printbibliography[heading=subbibliography,notkeyword=this]
}
\end{document}



%      <!-- Local IspellDict: en_US-w_accents -->
